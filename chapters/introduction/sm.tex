\chapter{The \acl{SM}}
\label{chap:intro:sm}

% TODO references

The \acf{SM} is a quantum field theory that describes the interactions of three 
of the four fundamental forces of nature, electromagnetism, the weak force, and 
the strong force, with a set of fundamental particles.
The \ac{SM} makes many predictions, from the spectral lines of the hydrogen  
atom to the rate of the \BsTomumu\ decay, and so far no experimental results 
have provided conclusive proof that such predictions are, fundamentally, 
incorrect.
The absence of a description of the fourth fundamental force, gravity, the 
non-zero masses of neutrinos, and several astronomical observations indicate 
that, at the very least, the \ac{SM} is not a complete description of the 
universe.
It does not describe the observed matter-antimatter asymmetry, nor can it 
account for the apparent presence of dark matter.
The purpose of the \ac{LHC} is to provide experiments with enough data to be 
able to find minute deviations from the predictions of the \ac{SM}, which could 
point to particular extensions of theory.
In this \namecref{chap:intro:sm}, the forces and particles described by the 
\ac{SM} are summarised.
Particular emphasis is given to the phenomenology of the weak force, which the 
\lhcb\ detector is optimised to study.

The \ac{SM} is a powerful predictive model partly due to its use of symmetries, 
% TODO citation for Noether's theorem?
which, through Noether's theorem, give rise to conserved currents.
For example, the symmetry of physical laws to rotations in space and 
translations in time leads to the conservation of total angular momentum and 
energy, respectively.
In a quantum field theory, each of these currents are quantised in units of a 
quantum charge.
An interaction with a field changes the value of the corresponding quantum 
charge of a particular particle by integer multiples, and the total charge of 
the system is invariant.
Interactions with a field only occur if the particles are charged under that 
field, such as the familiar electric charge for the electromagnetic field.

Interactions with fields are mediated via the exchange of bosons, fundamental 
particles with integer spin, such as the photon, the force carrier of the 
electromagnetic field, and the Higgs, the particle responsible for giving the 
fundamental particles mass.
Fermions are particles with half-integer spins; the quarks, leptons, and 
neutrinos constitute the fundamental fermions.

The strong force acts on particles with colour charge, quarks and gluons, via 
the exchange of gluons.
The fact that gluons are charged under the force they mediate significantly 
complicates the theory of strong interactions.
At low energies, the self interaction leads to the phenomenon of confinement, 
whereby free colour-charged objects cannot be observed, as the attractive force 
between two colour-charged objects is approximately constant at large 
separations, and so an infinite amount of energy is required to separate such 
objects completely.
Conversely, at higher energies, these particles become asymptotically free, 
eventually forming a quark-gluon plasma.
The energy scale at which the transition between these two regimes occurs is 
denoted \qcdscale.

Theoretical studies of the strong force use the framework of \ac{QCD}, which 
can either be used perturbatively, when the energy scale of the process under 
consideration is much larger than \qcdscale, or non-perturbatively, when the 
opposite is true.
It is feasible to make predictions using perturbative \ac{QCD}, although the 
computations become more challenging as the order in the perturbative 
expansions increases.
% TODO mention lattice QCD
Predictions requiring non-perturbative \ac{QCD} are generally not possible to 
perform analytically with today's understanding.
In the cases, experimental input is necessary to constrain the theory.
One such example of a non-perturbative problem is the prediction of \pp\ 
cross-sections at the \ac{LHC}, which will be discussed in more detail in 
\cref{chap:prod:theory}.

Six quarks (\Pup, \Pdown, \Pcharm, \Pstrange, \Ptop, and \Pbottom) are only 
observed experimentally in colourless bound states of two or more quarks called 
hadrons.
The behaviour of hadrons can be studied with a particle detector to infer the 
behaviour of the quarks and gluons.
In particular, the \lhcb\ detector is optimised to select and reconstruct the 
decays of mesons and baryons in order to probe the nature of the weak force.

The weak force interacts with particles with a non-zero weak isospin \wisospin, 
and the carriers are the two charged \PWpm bosons and the neutral \PZ boson.
It is particularly interesting to study because it violates several symmetries 
that were historically considered to be exact, and is the only force to have 
been observed to do so.
One example is symmetry under the parity transformation \Ptransform, which 
changes the sign of the spatial coordinates.
The weak force violates this symmetry in the largest possible way by only 
coupling to particles with left-handed chirality and anti-particles with 
right-handed chirality.
The discovery of \Ptransform\ violation~\cite{Wu:1957my} was soon followed by 
the discovery of \CP\ violation~\cite{Christenson:1964fg}, the breaking of the 
symmetry of the \CP\ transformation which simultaneously changes the handedness 
of particles (\Ptransform) and transforms particles into their anti-particles.  
(\Ctransform).

\section{\texorpdfstring{\CP}{CP} violation}
\label{chap:intro:sm:cp}

As the measurement of \CP\ violation in charm and beauty hadron decays is a 
cornerstone of the \lhcb\ physics programme, and indeed \cref{chap:cpv} of the 
thesis, its mechanism will be describe here in more detail.

The weak interaction eigenstates $(\Pdown', \Pstrange', \Pbottom')$ are 
different to the quark mass eigenstates $(\Pdown, \Pstrange, \Pbottom)$.
The weak eigenstates are a superposition of the mass eigenstates, the linear 
coefficients of which are given by the \ac{CKM} matrix
\begin{equation}
  \begin{pmatrix} \Pdown' \\ \Pstrange' \\ \Pbottom' \end{pmatrix}
  =
  V_{\text{CKM}}\begin{pmatrix} \Pdown \\ \Pstrange \\ \Pbottom \end{pmatrix}
  =
  \begin{pmatrix}
    \Vud & \Vus & \Vub \\
    \Vcd & \Vcs & \Vcb \\
    \Vtd & \Vts & \Vtb
  \end{pmatrix}
  \begin{pmatrix} \Pdown \\ \Pstrange \\ \Pbottom \end{pmatrix},
\end{equation}
where $V_{ij}$ is the coupling of the $i$ to $j$ transition, e.g.\ $|\Vud|^{2}$ 
is the relative probability of the transition \decay{\Pdown}{\Pup}.
The values of the \ac{CKM} matrix elements are not predicted by the \ac{SM} and 
so must be determined experimentally.
By exploiting the predicated unitarity of the \ac{CKM} matrix, it can be 
represented using only three mixing angles and a complex phase $\delta$
\begin{equation}
  V_{\textrm{CKM}} =
  \begin{pmatrix}
    c_{12}c_{13} & s_{12}c_{13} & s_{13}e^{-i\delta} \\
    -s_{12}c_{23} - c_{12}s_{23}s_{13}e^{-i\delta} & c_{12}c_{23} - 
    s_{12}s_{23}s_{13}e^{-i\delta} & s_{23}c_{13} \\
    s_{12}c_{23} - c_{12}c_{23}s_{13}e^{-i\delta} & -c_{12}s_{23} - 
    s_{12}c_{23}s_{13}e^{-i\delta} & c_{23}c_{13}
  \end{pmatrix},
\end{equation}
where $s_{ij} = \sin{\theta_{ij}}$ and $c_{ij} = \cos{\theta_{ij}}$.
In alternative formulations~\cite{Wolfenstein:1983yz}, the three angles are 
parameterised as $\alpha$, $\beta$ and $\gamma$.
One of the key measurements of \lhcb\ is that of the angle $\gamma$, which is 
the least well-measured of the three angles, with the precision on the world 
average around \SI{9}{\percent}~\cite{LHCb-CONF-2016-001}; in comparison the 
\ac{SM} precision is estimated to be up to \SI{4}{\degree}, depending on the 
assumptions~\cite{Brod:2013sga,Brod:2014bfa}.

It is the phase in the \ac{CKM} matrix that permits \CP\ violation.
Consider a decay \decay{P}{f} and the \CP\ conjugate process
\decay{\bar{P}}{\bar{f}}.
The corresponding decay amplitudes $\mathcal{M}_{f}$ and 
$\bar{\mathcal{M}}_{\bar{f}}$ can have some phase $\phi$, and the \ac{CKM} 
mechanism can add an additional phase $\theta$ which changes sign under \CP, 
giving
\begin{equation}
  \mathcal{M}_{f}             = |a| e^{i(\phi + \theta)}, \quad
  \bar{\mathcal{M}}_{\bar{f}} = |a| e^{i(\phi - \theta)},
  \label{eqn:intro:sm:weak:amplitudes_one}
\end{equation}
where $|a|$ is the magnitude of matrix element.
The observable rates of these processes, $|M_{f}|^{2}$ and 
$|\bar{M}_{\bar{f}}|^{2}$, are equal, and so the decay is symmetric under 
\CP\@.
However, for a process that can proceed via two channels, for example, such as 
in \cref{fig:intro:sm:weak_feynman}, the total matrix elements are 
$\mathcal{M}_{f} = \mathcal{M}_{f,1} + \mathcal{M}_{f,2}$ and 
$\bar{\mathcal{M}}_{\bar{f}} = \bar{\mathcal{M}}_{\bar{f},1} + 
\bar{\mathcal{M}}_{\bar{f},2}$ with
\begin{align*}
  \mathcal{M}_{f,1} &= |a_{1}| e^{i(\phi_{1} + \theta_{1})}, &
  \mathcal{M}_{f,2} &= |a_{2}| e^{i(\phi_{2} + \theta_{2})},\\
  \bar{\mathcal{M}}_{\bar{f},1} &= |a_{1}| e^{i(\phi_{1} - \theta_{1})}, &
  \bar{\mathcal{M}}_{\bar{f},2} &= |a_{2}| e^{i(\phi_{2} - \theta_{2})}.
\end{align*}
It then follows that
\begin{equation*}
  |\mathcal{M}_{f}|^{2} - |\bar{\mathcal{M}}_{\bar{f}}|^{2} =
  -4|\mathcal{M}_{1}||\mathcal{M}_{2}|\sin(\phi_{1} - \phi_{2})\sin(\theta_{1} 
  - \theta_{2}).
\end{equation*}
This is non-zero, provided that $\theta_{1} \neq \theta_{2}$ and $\phi_{1} \neq 
\phi_{2}$, and hence introduces a possible \CP\ asymmetry.
It is then the interference between channels that permits \CP\ symmetry 
breaking in weak interactions.

The breaking of the \CP\ symmetry was first observed in kaon decays in 
1964~\cite{Christenson:1964fg}, and in 2011 \CP\ violation was first observed 
in beauty decays~\cite{Aubert:2001nu,Abe:2001xe}.
Observing \CP\ violation in charm decays would provide a more complete picture 
of what is possible in the \ac{SM}, but it will not solve the problem of the 
observed baryon asymmetry in the universe, which the \ac{SM} cannot account 
for.
In the \ac{LHC} era, it is hoped that signs of physics \acl{BSM} will point to 
specific, new theories, that themselves may have explanations for this great 
mystery.

\begin{figure}
  % The relative widths of the subfigures here have been
  % roughly tweaked to produce equally sized fonts
  \begin{subfigure}{0.40\textwidth}
    \centering
    \includegraphics[width=\textwidth]{introduction/weak_tree}
    \caption{Tree diagram}
    \label{fig:intro:sm:weak_feynman:tree}
  \end{subfigure}
  \begin{subfigure}{0.55\textwidth}
    \centering
    \includegraphics[width=\textwidth]{introduction/weak_penguin}
    \caption{Penguin diagram}
    \label{fig:intro:sm:weak_feynman:penguin}
  \end{subfigure}
  \caption{%
    Two possible Feynman diagrams for the decay \decay{\PBzero}{\PKplus\Ppiminus}.
  }
  \label{fig:intro:sm:weak_feynman}
\end{figure}
