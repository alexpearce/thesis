\chapter{Results}
\label{chap:prod:results}

Double differential open charm cross-section measurements are made in bins of 
charm hadron transverse momentum and rapidity with the relation in 
\cref{eqn:prod:introduction:differential_cross_section}, repeated here for 
reference
\begin{equation*}
  % There is a "\left." here, which is invisible, to balance the "\right\rvert"
  \left.\frac{\dif^{2}{\xsec(\PHc)}}{\dif{\pT}\dif{\rapidity}}\right\rvert_{i}
    = \frac{1}{\Delta\pT\Delta\rapidity}
      \frac{%
        N_{i}(\HcTof)
      }{%
        \eff_{i}(\HcTof)\cdot\bfrac(\HcTof)\cdot\intlumi
      }.
\end{equation*}
The prompt signal yields $N_{i}(\HcTof)$ are measured with the maximum 
likelihood fit described in \cref{chap:prod:fitting}, and the measurements of 
the acceptance, reconstruction, and total selection efficiency 
$\eff_{i}(\HcTof)$ are detailed in \cref{chap:prod:effs}.
The branching fractions \bfrac\ are listed in 
\cref{tab:prod:introduction:branching_ratios}, and the integrated luminosity 
\intlumi\ is \SI{\xsectotlumi}{\per\pico\barn}, as given in 
\cref{eqn:prod:xsectotlumi}.
The estimation of the associated systematic uncertainties on these quantities 
is described in \cref{chap:prod:syst}.
Combining all inputs, the measurements are presented in 
\cref{fig:prod:results:double_differential:D0_Dp,fig:prod:results:double_differential:Ds_Dst}.
They are compared with the three set of predictions, where available for a 
given meson and \pTy\ bin, described in \cref{chap:prod:theory:comparisons}.
A comparison of the data with the predictions is given in 
\cref{chap:prod:results:discussion}.
Preceding that, measurements of charm production ratios between proton-proton 
centre-of-mass energies and between charm mesons are presented.

\section{Production ratios}
\label{chap:prod:results:ratios}

The predicted ratios of prompt charm production cross-sections between 
different centre-of-mass energies can be much more precise than absolute 
predictions, as several common sources of uncertainties 
cancel~\cite{Gauld:2015yia,Cacciari:2015fta,Kniehl:2012ti}.
Using the present results obtained at \sqrtseq{13} and the corresponding 
results at \sqrtseq{7}~\cite{LHCb-PAPER-2012-041}, the ratios 
\resultratio{13}{7} are measured for \PDzero, \PDplus, \PDsplus, and \PDstarp\ 
mesons.
The \sqrtseq{13} measurements are re-binned to match the binning used in the 
\sqrtseq{7} results and the production ratios are presented for \pTrange{0}{8} 
and \yrange{2}{4.5}.
In the calculation of the uncertainties on the ratios, those due to the use of 
the branching fractions cancel.
Correlations of \SI{30}{\percent} and \SI{50}{\percent} are assumed for the 
uncertainties on the luminosity measurements and tracking efficiency 
corrections, respectively, and all other uncertainties are assumed to be 
uncorrelated.
\Cref{fig:prod:results:ratio_7tev} shows the measured ratios compared with 
predictions from theory 
calculations~\cite{Gauld:2015yia,Cacciari:2015fta,Kniehl:2012ti}.

Cross-section ratios between different charm mesons are also computed.
It is generally assumed that the fragmentation fractions $f(\cToHc)$, as in 
\cref{eqn:prod:introduction:ccbar_cross_section}, are universal, in that they 
do not depend upon the hard process that created the charm 
quark~\cite{PDG2008,Lisovyi:2015uqa}, and comparing cross-section meson ratios 
measured at different colliders can test this assumption.
Measurements of open charm meson production have been made using \epem\ 
collision data by the \babar, \belle, and \cleo\ 
collaborations~\cite{Artuso:2004pj,Seuster:2005tr,Aubert:2002ue}, collectively 
`\bfactories', and so comparisons of meson ratios are made with those results.
The \bfactory\ measurements of \PDzero, \PDplus, \PDsplus, and \PDstarp\ 
production were made using the same set of final states as the analysis 
described here, and so more precise comparisons can be made by computing the 
ratios of cross-sections multiplied by the ratio of respective branching 
fractions \xsectimesbfrac, for example
\begin{equation*}
  \resultratio{\PDplus}{\PDzero} = \frac{%
    \xsec(\PDplus)
  }{%
    \xsec(\PDzero)
  }\frac{%
    \bfrac(\DpToKpipi)
  }{%
    \bfrac(\DzToKpi)
  }.
\end{equation*}
The uncertainty due to the branching fractions does not enter in this ratio.
The meson ratios, differential in \pT\ and \rapidity\, are presented in 
\cref{fig:prod:results:ratio_mesons_a,fig:prod:results:ratio_mesons_b}, along 
with comparisons to the ratios of the integrated measurements made by the 
\bfactories.
For most ratios, the \belle\ and \cleo\ results are shown for comparison with 
the \lhcb\ results.
However, \cleo\ did not measure the \PDsplus production cross-section, and so 
for ratios including \PDsplus the relevant \cleo\ results are combined with the 
\babar\ \PDsplus results.

\section{Integrated cross-sections}
\label{chap:prod:results:integrated}

Integrated production cross-sections $\xsec(\ppToHc{}X)$ for each charm meson 
are computed as the sum of the per-bin measurements, where the uncertainty on 
the sum takes into account the correlations between bins discussed in 
\cref{chap:prod:syst}.
For \PDsplus and \PDstarp\ mesons, the kinematic region considered is 
\pTyrange{1}{8}{2}{4.5} due to insufficient data below $\pT = \SI{1}{\GeVc}$, 
while for \PDzero and \PDplus the same kinematic region as for the ratio 
measurements is used.
The upper limit is chosen to coincide with that of the \lhcb\ measurements at 
\sqrtseq{7}.

The \PDzero and \PDstarp\ cross-section results contain bins in which a 
measurement was not possible, and so the initial cross-section integration 
requires a correction factor to account for the missing measurements.
A method based on theory calculations is chosen, wherein a multiplicative 
correction factor is computed as the ratio between the predicted integrated 
cross-section within the considered kinematic region and the sum of all per-bin 
cross-section predictions for bins for which a measurement exists.
This method uses the \nnpdfl\ predictions~\cite{Gauld:2015yia} for \PDzero and 
the \fonll\ predictions~\cite{Cacciari:2015fta} for \PDstarp.
The uncertainty on the extrapolation factor is taken as the difference between 
factors computed using the upper and lower bounds of the theory predictions, 
and is propagated to the integrated cross-sections as a systematic uncertainty.
\Cref{tab:prod:results:integrated_double_differential} gives the integrated 
cross-sections for \PDzero, \PDplus, \PDsplus, and \PDstarp\ mesons.

The integrated total charm cross-section $\xsec(\ppToccbarX)$ is calculated as 
\begin{equation}
  \xsec(\ppToccbarX) = \frac{\xsec(D)}{(2f(\cToHc))},
\end{equation}
for each decay mode.
The term $f(\cToHc)$ is the quark to hadron transition probability, and the 
factor 2 accounts for the inclusion of charge conjugate (anti-charm) states in 
the measurement.
The transition probabilities have been computed using measurements at \epem\ 
colliders operating at a centre-of-mass energy close to the \PUpsilonFourS 
resonance~\cite{PDG2008}, and are listed in 
\cref{tab:prod:introduction:fragmentation_fractions}.
The fragmentation fraction $f(\decay{\Pcharm}{\PDzero})$ has an overlapping 
contribution
from $f(\decay{\Pcharm}{\PDstarp})$.

The precision on the \ccbar\ cross-section can be improved by combining the 
individual estimates.
For this, the \PDsplus and \PDstarp\ inputs are neglected, as the \PDzero and 
\PDplus measurements are considerably more precise.
The \PDzero\ and \PDplus\ inputs are combined using the method of the 
\emph{best linear unbiased estimator}, or \blue\ method~\cite{Lyons:1988rp}.
This procedure combines a set of $N$ measurements $m_{i}$ as an average $m$ 
using per-measurement weights $\alpha_{i}$
\begin{equation}
  m = \sum_{i}^{N} \alpha_{i}m_{i}.
  \label{eqn:results:blue}
\end{equation}
This linear relation is where the `linear' name comes from.
The `unbiased' name comes from the constraint on the set of weights that 
$\sum_{i}\alpha_{i} = 1$, and `best' comes from the choice of the weights being 
such that the covariance $\sigma^{2}$ of the average is minimal
\begin{equation}
  \sigma^{2} = \vec{\alpha}^{\text{T}}C\vec{\alpha},
  \label{eqn:results:blue_variance}
\end{equation}
where $C$ is the covariance matrix of the input measurements, and 
$\vec{\alpha}$ is the vector of $N$ weights, each $\alpha_{i}$.
The vector of weights optimising this equation can be computed from $C$ as
\begin{equation}
  \alpha = \frac{C^{-1}\vec{u}}{\vec{u}^{\text{T}}C^{-1}\vec{u}}
  \label{eqn:results:blue_weights}
\end{equation}
where $\vec{u}$ is a vector of length $N$ whose components are all 
unity~\cite{Lyons:1988rp}.
The combination of the \PDzero and \PDplus measurements in this manner gives
\begin{equation*}
  \xsec{(\ppToccbarX)}_{\pT\,<\,8\,\si{\GeVc},\,2.0\,<\,y\,<\,4.5} =
    \SI[parse-numbers=false]{2840 \pm 3 \pm 170 \pm 150}{\micro\barn},
\end{equation*}
where the uncertainties are statistical, systematic, and from the uncertainty 
on the fragmentation fractions.

A comparison of the \ccbar\ cross-sections with predictions in the \pT\ range 
\pTrange{0}{8} is given in 
\cref{fig:prod:results:integrated_double_differential}.
The same Figure also shows a comparison of $\xsec(\ppToccbarX)$ for 
\pTrange{1}{8} computed using each of the four open charm mesons.

Ratios of the integrated \xsectimesbfrac\ measurements are given in 
Table~\ref{tab:prod:results:integrated_meson_ratios}.

\section{Interpretation and comparison with theory}
\label{chap:prod:results:discussion}

At the highest level of validation, 
\cref{fig:prod:results:integrated_double_differential:1_8} shows that 
integrated cross-sections agree very well between all four charm mesons of 
which measurements were made.
In comparison, the central value of the \ccbar\ cross-section is higher in the 
\pTrange{0}{1} region as measured with \PDzero than with \PDplus, as shown in 
\cref{fig:prod:results:integrated_double_differential:0_8}, although the 
measurements are in agreement within the $\pm1\,\sigma$ uncertainties.
The two absolute predictions for the \ccbar\ cross-section shown on 
\cref{fig:prod:results:integrated_double_differential:0_8} agree with the data 
within their very large uncertainties, however the most precise prediction, the 
``scaled'' number from the \nnpdfl\ group, does not agree with the \lhcb\ 
measurement.
This particular prediction was made be predicting the ratio \resultratio{13}{7} 
and then scaling it with the \lhcb\ measurements at \sqrtseq{7}.
The disagreement between the data and this predictions suggests a systematic 
mis-modelling that scales with the centre-of-mass energy.

Comparing the double differential cross-section results with the predictions, 
presented in 
\cref{fig:prod:results:double_differential:D0_Dp,fig:prod:results:double_differential:Ds_Dst}, 
the data consistently lie at or above the upper limit of the $\pm1\,\sigma$ 
uncertainty bands on the predictions.
The shape in both \pT\ and \rapidity\ is well-modelled, but worsens with 
increasing \pT\ and decreasing rapidity.
The \gmvfns\ prediction provides the truest description of data, although it is 
unable to make any predictions below $\pT = \SI{3}{\GeVc}$ due to large 
theoretical uncertainties.
The \fonll\ prediction set is similarly restricted in the low \pT\ regions, 
with very low \pT\ bins having associated uncertainties spanning over two 
orders of magnitude.
These factors point to a difficulty in predicting the overall scale of double 
differential charm cross-sections.
The measurements presented here can be used to constrain the \aclp{QCDPDF} 
further, which may help improve the agreement between future measurements and 
predictions.

The agreement between the ratios of cross-sections at $\sqrts = 13$ and 
\SI{7}{\TeV} shown in \cref{fig:prod:results:ratio_7tev} is good, although the 
possibility of interpretation is limited by the large uncertainties on the 
data, which are driven by the large per-bin uncertainties on the \SI{7}{\TeV} 
measurement.
Such ratios are particularly useful for tuning theoretical predictions, due to 
aforementioned cancellation of input uncertainties.
% TODO might be able to add the 5 TeV paper here
Additional high-precision measurements of open charm production cross-sections 
at \lhcb\ with different centre-of-mass energies would be useful constraints.

The measurements of ratios of cross-sections times branching fractions 
\xsectimesbfrac\ presented in 
\cref{fig:prod:results:ratio_mesons_a,fig:prod:results:ratio_mesons_b} agree 
with those made at the \bfactories, indicting that the assumption of 
universality of the fragmentation fractions $f(\cToHc)$ is valid.
There is a weak trend in \pT\ in some ratios, such in 
\cref{fig:prod:results:ratio_mesons:Ds_D0,fig:prod:results:ratio_mesons:Dst_D0}, 
which is consistent with heavier mesons having a harder \pT\ 
spectrum~\cite{PDG2008}.

\begin{figure}
  \begin{subfigure}[b]{\textwidth}
    \centering
    \includegraphics[width=\textwidth]{production/results/D0_double_differential}
    \caption{\PDzero}
    \label{fig:prod:results:double_differential:D0}
  \end{subfigure}
  \begin{subfigure}[b]{\textwidth}
    \centering
    \includegraphics[width=\textwidth]{production/results/Dp_double_differential}
    \caption{\PDplus}
    \label{fig:prod:results:double_differential:Dp}
  \end{subfigure}
  \caption{%
    Measurements and predictions for the absolute prompt \PDzero 
    (\subref*{fig:prod:results:double_differential:D0}) and \PDplus 
    (\subref*{fig:prod:results:double_differential:Dp}) cross-sections at 
    \sqrtseq{13}.
    Each set of measurements and predictions in a given rapidity bin is offset 
    by a multiplicative factor $10^{-m}$, where the factor $m$ is shown on the 
    plots.
    The boxes indicate the $\pm1\sigma$ uncertainty band on the theory 
    predictions.
    In cases where this band spans more than two orders of magnitude only its 
    upper edge is indicated.
  }
  \label{fig:prod:results:double_differential:D0_Dp}
\end{figure}

\begin{figure}
  \begin{subfigure}[b]{\textwidth}
    \centering
    \includegraphics[width=\textwidth]{production/results/Ds_double_differential}
    \caption{\PDsplus}
    \label{fig:prod:results:double_differential:Ds}
  \end{subfigure}
  \begin{subfigure}[b]{\textwidth}
    \centering
    \includegraphics[width=\textwidth]{production/results/Dst_double_differential}
    \caption{\PDstarp}
    \label{fig:prod:results:double_differential:Dst}
  \end{subfigure}
  \caption{%
    Measurements and predictions for the absolute prompt \PDsplus 
    (\subref*{fig:prod:results:double_differential:Ds}) and \PDstarp 
    (\subref*{fig:prod:results:double_differential:Dst}) cross-sections at 
    \sqrtseq{13}.
    Each set of measurements and predictions in a given rapidity bin is offset 
    by a multiplicative factor $10^{-m}$, where the factor $m$ is shown on the 
    plots.
    The boxes indicate the $\pm1\sigma$ uncertainty band on the theory 
    predictions.
    In cases where this band spans more than two orders of magnitude only its 
    upper edge is indicated.
  }
  \label{fig:prod:results:double_differential:Ds_Dst}
\end{figure}

\begin{figure}
  \begin{subfigure}[b]{0.5\textwidth}
    \centering
    \includegraphics[width=\textwidth]{production/results/D0_ratio_with_2010}
    \caption{\PDzero}
    \label{fig:prod:results:ratio_7tev:D0}
  \end{subfigure}
  \begin{subfigure}[b]{0.5\textwidth}
    \centering
    \includegraphics[width=\textwidth]{production/results/Dp_ratio_with_2010}
    \caption{\PDplus}
    \label{fig:prod:results:ratio_7tev:Dp}
  \end{subfigure}
  \begin{subfigure}[b]{0.5\textwidth}
    \centering
    \includegraphics[width=\textwidth]{production/results/Ds_ratio_with_2010}
    \caption{\PDsplus}
    \label{fig:prod:results:ratio_7tev:Ds}
  \end{subfigure}
  \begin{subfigure}[b]{0.5\textwidth}
    \centering
    \includegraphics[width=\textwidth]{production/results/Dst_ratio_with_2010}
    \caption{\PDstarp}
    \label{fig:prod:results:ratio_7tev:Dst}
  \end{subfigure}
  \caption{%
    Measurements and predictions of the prompt \PDzero 
    (\subref*{fig:prod:results:ratio_7tev:D0}), \PDplus 
    (\subref*{fig:prod:results:ratio_7tev:Dp}), \PDsplus 
    (\subref*{fig:prod:results:ratio_7tev:Ds}), and \PDstarp 
    (\subref*{fig:prod:results:ratio_7tev:Dst}) cross-section ratios.
    Each set of measurements and predictions in a given rapidity bin is offset 
    by an additive factor $m$ shown on the plots.
    The dash-dotted lines indicate the unit ratio for each of the rapidity 
    intervals and the dashed lines indicate a ratio of two.
    Each set of measurements and predictions in a given rapidity bin is 
    offset by an additive constant $m$, which is shown on the plot.
    No prediction is available for the \PDsplus ratio.
  }
  \label{fig:prod:results:ratio_7tev}
\end{figure}

\begin{figure}
  \begin{subfigure}[b]{0.5\textwidth}
    \centering
    \includegraphics[width=\textwidth]{production/results/Dp_ratio_with_D0}
    \caption{\resultratio{\PDplus}{\PDzero}}
    \label{fig:prod:results:ratio_mesons:Dp_D0}
  \end{subfigure}
  \begin{subfigure}[b]{0.5\textwidth}
    \centering
    \includegraphics[width=\textwidth]{production/results/Ds_ratio_with_D0}
    \caption{\resultratio{\PDsplus}{\PDzero}}
    \label{fig:prod:results:ratio_mesons:Ds_D0}
  \end{subfigure}
  \begin{subfigure}[b]{0.5\textwidth}
    \centering
    \includegraphics[width=\textwidth]{production/results/Dst_ratio_with_D0}
    \caption{\resultratio{\PDstarp}{\PDzero}}
    \label{fig:prod:results:ratio_mesons:Dst_D0}
  \end{subfigure}
  \caption{%
    Ratios of cross-section times branching fraction measurements of \PDplus 
    (\subref*{fig:prod:results:ratio_mesons:Dp_D0}), \PDsplus 
    (\subref*{fig:prod:results:ratio_mesons:Ds_D0}), and \PDstarp\ 
    (\subref*{fig:prod:results:ratio_mesons:Dst_D0}) mesons with respect to the 
    \PDzero measurements.
    The bands indicate the corresponding ratios computed using measurements 
    from \epem\ collider 
    experiments~\cite{Artuso:2004pj,Seuster:2005tr,Aubert:2002ue}.
    The ratios are given as a function of \pT\ and different symbols indicate 
    different ranges in \rapidity.
    The notation $\xsec(\HcTof)$ on each $y$-axis is shorthand for 
    \xsectimesbfrac.
  }
  \label{fig:prod:results:ratio_mesons_a}
\end{figure}

\begin{figure}
  \begin{subfigure}[b]{0.5\textwidth}
    \centering
    \includegraphics[width=\textwidth]{production/results/Dst_ratio_with_Dp}
    \caption{\resultratio{\PDstarp}{\PDplus}}
    \label{fig:prod:results:ratio_mesons:Dst_Dp}
  \end{subfigure}
  \begin{subfigure}[b]{0.5\textwidth}
    \centering
    \includegraphics[width=\textwidth]{production/results/Ds_ratio_with_Dp}
    \caption{\resultratio{\PDsplus}{\PDplus}}
    \label{fig:prod:results:ratio_mesons:Ds_Dp}
  \end{subfigure}
  \begin{subfigure}[b]{0.5\textwidth}
    \centering
    \includegraphics[width=\textwidth]{production/results/Ds_ratio_with_Dst}
    \caption{\resultratio{\PDstarp}{\PDsplus}}
    \label{fig:prod:results:ratio_mesons:Ds_Dst}
  \end{subfigure}
  \caption{%
    Ratios of cross-section times branching fraction measurements for 
    $\PDstarp/\PDplus$ (\subref*{fig:prod:results:ratio_mesons:Dst_Dp}), 
    $\PDsplus/\PDplus$ (\subref*{fig:prod:results:ratio_mesons:Ds_Dp}), and 
    $\PDstarp\PDsplus$ (\subref*{fig:prod:results:ratio_mesons:Ds_Dst}).
    The bands indicate the corresponding ratios computed using measurements 
    from \epem\ collider 
    experiments~\cite{Artuso:2004pj,Seuster:2005tr,Aubert:2002ue}.
    The ratios are given as a function of \pT\ and different symbols indicate 
    different ranges in \rapidity.
    The notation $\xsec(\HcTof)$ on each $y$-axis is shorthand for 
    \xsectimesbfrac.
  }
  \label{fig:prod:results:ratio_mesons_b}
\end{figure}

\begin{figure}
  \begin{subfigure}[b]{0.5\textwidth}
    \centering
    \includegraphics[width=\textwidth]{production/results/integrated_double_differential_0_8}
    \caption{\pTrange{0}{8}}
    \label{fig:prod:results:integrated_double_differential:0_8}
  \end{subfigure}
  \begin{subfigure}[b]{0.5\textwidth}
    \centering
    \includegraphics[width=\textwidth]{production/results/integrated_double_differential_1_8}
    \caption{\pTrange{1}{8}}
    \label{fig:prod:results:integrated_double_differential:1_8}
  \end{subfigure}
  \caption{%
    Integrated cross-sections (black diamonds), their average (black circle and 
    blue band) and theory predictions (red 
    squares)~\cite{Gauld:2015yia,Cacciari:2015fta} are shown based on the 
    \PDzero and \PDplus for \pTrange{0}{8} 
    (\subref*{fig:prod:results:integrated_double_differential:0_8}) and for 
    measurements based on all four mesons for \pTrange{1}{8} 
    (\subref*{fig:prod:results:integrated_double_differential:1_8}).
    The ``absolute'' predictions are based on calculations of the \SI{13}{\TeV} 
    cross-section, while the ``scaled'' predictions are based on calculations 
    of the $13$ to \SI{7}{\TeV} ratio multiplied with the \lhcb\ measurement at 
    \SI{7}{\TeV}~\cite{LHCb-PAPER-2012-041}.
  }
  \label{fig:prod:results:integrated_double_differential}
\end{figure}

\begin{table}
  \caption{%
    Prompt charm production cross-sections integrated over \yrange{2}{4.5} and 
    the \pT\ ranges given.
    The computation of the extrapolation factors is described in the text.
    The first uncertainty on the cross-section is statistical, and the second 
    is systematic and includes the contribution from the extrapolation factor.
    No extrapolation factor is given for $\PD^{+}_{(\Pstrange)}$ as a 
    measurement is available in every bin of the integrated phase space.
  }
  \label{tab:prod:results:integrated_double_differential}
  \centering
  \begin{tabular}{ccccr}
  \toprule
           &                  &                   & Extrapolation factor & Cross-section~(\si{\micro\barn}) \\
  \midrule
  \PDzero  & $\pTrange{0}{8}$ & $\yrange{2}{4.5}$ & $1.0004 \pm 0.0009$  & $3240 \pm \phantom{1}4 \pm 190$  \\
  \PDplus  & $\pTrange{0}{8}$ & $\yrange{2}{4.5}$ & -                    & $1290 \pm \phantom{1}8 \pm 190$  \\
  \midrule
  \PDzero  & $\pTrange{1}{8}$ & $\yrange{2}{4.5}$ & $1.0005 \pm 0.0009$  & $2470 \pm \phantom{1}3 \pm 130$  \\
  \PDplus  & $\pTrange{1}{8}$ & $\yrange{2}{4.5}$ & -                    & $1000 \pm \phantom{1}3 \pm 110$  \\
  \PDsplus & $\pTrange{1}{8}$ & $\yrange{2}{4.5}$ & -                    & $460 \pm 13 \pm 100$             \\
  \PDstarp & $\pTrange{1}{8}$ & $\yrange{2}{4.5}$ & $1.0004 \pm 0.0023$  & $880 \pm \phantom{1}5 \pm 140$   \\
  \bottomrule
\end{tabular}

\end{table}

\begin{table}
  \caption{%
    Ratios of integrated cross-section times branching fraction measurements in 
    the kinematic range \pTrange{1}{8} and \yrange{2}{4.5}.
    The first uncertainty on the ratio is statistical and the second is 
    systematic. The notation $\xsec(\HcTof)$ is shorthand for \xsectimesbfrac.
  }
  \label{tab:prod:results:integrated_meson_ratios}
  \centering
  \renewcommand{\arraystretch}{1.3}
\begin{tabular}{lc}
  \toprule
  Quantity                               & Measurement                                    \\
  \midrule
  $\xsec(\DpToKpipi)/\xsec(\DzToKpi)$    & $0.953 ^{+0.003}_{-0.003}$$^{+0.060}_{-0.054}$ \\
  $\xsec(\DspTophipi)/\xsec(\DzToKpi)$   & $0.106 ^{+0.003}_{-0.003}$$^{+0.009}_{-0.010}$ \\
  $\xsec(\DstToDzpi)/\xsec(\DzToKpi)$    & $0.242 ^{+0.001}_{-0.001}$$^{+0.027}_{-0.026}$ \\
  \midrule
  $\xsec(\DspTophipi)/\xsec(\DpToKpipi)$ & $0.112^{+0.004}_{-0.004}$$^{+0.006}_{-0.009}$  \\
  $\xsec(\DstToDzpi)/\xsec(\DpToKpipi)$  & $0.254^{+0.001}_{-0.001}$$^{+0.016}_{-0.017}$  \\
  \midrule
  $\xsec(\DspTophipi)/\xsec(\DstToDzpi)$ & $0.444^{+0.013}_{-0.013}$$^{+0.042}_{-0.052}$  \\
  \bottomrule
\end{tabular}

\end{table}
