\chapter{Efficiency evaluation}
\label{chap:prod:effs}

% TODO make sure every section defines how the uncertainties are computed

The efficiency chain defines the fraction of produced prompt signal candidates 
that survive the full reconstruction and selection procedure.
The chain is defined in steps grouped by physical requirements and those 
imposed by software, with each step being relative to the previous one.
In the order of steps by which candidates are lost, this starts with the efficiency of a signal charm hadron to decay within the 
\lhcb\ acceptance having been produced in a \pp\ collision, \effacc.
The efficiency of such a decay to be fully reconstructed is the reconstruction 
efficiency is \effreco, and then the efficiency of the reconstructed decay to 
be triggered through \lzero, \hltone, and \hlttwo is the trigger efficiency 
\efftrig.
There is also the efficiency of triggered candidates to pass through the 
offline selection \effoffline, and finally there is the signal window 
efficiency \effsigwin\ due to the requirement imposed before the \lnipchisq\ 
fit.
The total efficiency, \eff, is the product of these, and is what enters in 
\cref{eqn:prod:introduction:differential_cross_section}.

For most steps in the efficiency chain, the candidates not passing each step 
are lost and cannot be processed further, and so methods are needed to assess the 
efficiency without having access to the original candidates.
There are two methods that shall be used in this \namecref{chap:prod:effs}: 
efficiency estimation from simulated \acf{MC} data, described in 
\cref{chap:prod:data:mc}, where true information is available before and after 
all steps; and methods of calibration, where proxy data selected without the 
use of particular information can be used to assess the effects of using that information in the 
analysis data.
Efficiency evaluation with \ac{MC} can be simple, but differences between the 
data and the \ac{MC} must be accounted for, which may not be obvious.
Calibration techniques use real data, and so can be more robust than \ac{MC}, 
but obtaining clean calibration samples can be challenging.

In \lhcb, it is known that the \ac{PID} selection efficiencies are not 
well-modelled in the \ac{MC}, due to an under-estimation of the detector 
occupancies.
The same is true, albeit to a lesser extent, for the reconstruction 
efficiencies.
Kinematic variables, such as charm hadron and final state momenta, are 
comparatively well-modelled.
This motivates a two-step procedure for evaluating the \emph{selection} 
efficiency, whereby the \ac{PID} efficiency is computed using calibration 
techniques, described in \cref{chap:prod:effs:pid}, and the efficiency of the remaining 
requirements is made using \ac{MC}, detailed in \cref{chap:prod:effs:sel}.
The reconstruction efficiency will be computed using \ac{MC}, but corrected 
with a factor obtained from calibration samples, described in \cref{chap:prod:effs:acc}.
The extent to which these techniques do not accurately model the efficiencies 
will be discussed in the context of systematic uncertainties in 
\cref{chap:prod:syst:mc}.

The matching of true \ac{MC} particles to reconstructed objects, the process of 
truth-matching as described in \cref{chap:prod:data:mc}, is not \SI{100}{\%} 
efficient, and so a correction factor \efftruth is be computed to account for 
the deficit.

To summarise: the total efficiency $\eff_{i}(\decay{\PHc}{f})$, for charm candidates in the $i$th \pTy\ bin, is 
the product of the individual efficiencies and correction factors
% TODO should revisit this to include how the sections actually flow; could 
% have the L0 efficiency explicitly, for example
\begin{align}
  \eff_{i}(\decay{\PHc}{f}) = \effacc &\times \effreco \times \efftruth \times \efftracking \nonumber\\
                                      &\times \efftrig \times \effoffline \times \effpid \times \effsigwin.
  \label{eqn:prod:effs:total_eff}
\end{align}
where each efficiency \eff\ is dependent on the charm hadron \PHc\ and the 
final state $f$, and is conditional on the previous step.

In this \namecref{chap:prod:effs} the details of the efficiency computation for each step is 
given, along with example efficiency tables in \pTy\ for the two-body \DzToKpi\ 
and three-body \DpToKpipi\ decay.
The notation for conditional yields will be used throughout, where $N_{B|A}$ 
denotes the prompt signal yields after process $B$ given that process $A$ 
preceded $B$.

\section{Detector acceptance}
\label{chap:prod:effs:acc}

The acceptance efficiency is due to the finite spatial acceptance of the \lhcb\ 
detector.
By instrumenting the forward region, in the pseudorapidity range $2 < \Eta < 
5$, any particles flying outside this region are not detected.
As there is no way to access these particles in data, it is necessary to use 
simulated data.
The acceptance is modelled by a set of cuts requiring that all stable charged 
particles in the final state are within $10 < \theta < 
\SI{400}{\milli\radian}$, where $\theta$ is the polar angle of the particle 
momentum vector, and that all the particles in the final state point in the 
positive $z$ direction, as defined in \cref{chap:prod:data:mc}.
These requirements are applied at the \emph{generator} level, before the 
detector simulation, on the true kinematics of the particles.

By counting the number of charm hadrons passing and failing the cut, the 
acceptance efficiency is defined as
\begin{equation}
  \effacc = \frac{%
    N_{\text{Accepted}|\text{Generated}}
  }
  {%
    N_{\text{\text{Generated}}}
  }.
  \label{eqn:prod:effs:acc}
\end{equation}
The input dataset to this computation is the generator-level set described in 
\cref{chap:prod:data:mc}.
% TODO should mention that we're using AC intervals, stating why and what they
% are
The acceptance efficiencies for \DzToKpi\ and \DpToKpipi, in \pTy\ bins, are 
given in \cref{tab:prod:effs:acc:dztokpi,tab:prod:effs:acc:dptokpipi}.

\section{Reconstruction}
\label{chap:prod:effs:reco}

The reconstruction efficiency \effreco\ parameterises the fraction of charm meson decays passing the acceptance requirements that are also fully reconstructed as tracks, for the final state particles, and vertices, for the charm mesons.
This folds in several effects: whether the final state particles have a large enough momentum not to be bent out of the detector acceptance by the magnetic field; whether the final state particles leave enough hits in the tracking system to be \emph{reconstructible} (that is, above the minimum threshold at which the tracking system \emph{could} reconstruct a track); whether, given that enough hits were deposited by a final state particle, a track was actually \emph{reconstructed}; and whether a vertex can be formed, given that all final state particles have associated tracks.

The reconstruction efficiency is defined as the ratio of fully reconstructed 
decays to those passing the acceptance requirements
\begin{equation}
  \effreco = \frac{%
    N_{\text{Reconstructed}|\text{Accepted}}
  }
  {%
    N_{\text{\text{Accepted}}}
  }.
  \label{eqn:prod:effs:reco}
\end{equation}
The reconstruction efficiencies for \DzToKpi\ and \DpToKpipi, in \pTy\ bins, are 
given in \cref{tab:prod:effs:reco:dztokpi,tab:prod:effs:reco:dptokpipi}.

\subsection{Truth matching inefficiency}
\label{chap:prod:effs:truth}

The evaluation of \cref{eqn:prod:effs:reco} requires that the reconstructed objects, tracks and vertices, be correctly associated back to the `truth-level' information, that is the \ac{MC} objects which were generated and then propagated through the detector simulation.
The truth-matching procedure is described in \cref{chap:prod:data:mc}.
The background category information is used to filter the \ac{MC} such that the 
number of remaining candidates are counted to obtain the input to some 
efficiency computation, such as \cref{eqn:prod:effs:reco}.
However, the requirement that a track is matched to \SI{70}{\percent} of the hits created by an \ac{MC} particle is not \SI{100}{\percent} efficient at accepting signal.
This results in an underestimation of the number of signal particles, and hence an overestimation of the cross-section.
The magnitude of the size of the inefficiency can be determined by counting the 
number of signal decays that fail the truth matching requirement.
\cref{fig:prod:effs:truth:categories} shows the \PDzero and \PDplus mass 
distributions for the various background categories.
The `ghost' category exhibits a peak identical in nature to that seen in the 
`signal' category.

Given that the number of correctly matched signal decays is $N$, and the number 
of incorrectly labelled signal decays is $U$, the selection efficiency defined 
in \cref{eqn:prod:effs:reco} is incorrect by a factor
\begin{equation}
  \efftruth = \frac{U+N}{N}.
  \label{eqn:prod:effs:truth}
\end{equation}
The number of decays passing the signal requirement is used for $N$, and $U$ is 
measured with a maximum likelihood fit to mass distribution of the decays not 
passing that requirement.
As shown in \cref{fig:prod:effs:truth:categories}, the fraction of incorrectly 
identified signal decays is small, and so a single fit is performed with the 
data integrated across all \pTy\ bins.
The details of the fit are identical to those for the mass fits used in the 
prompt charm yield extraction with real data, described in 
\cref{chap:prod:fitting:mass}.
The result of the fits to the \DzToKpi\ and the \DpToKpipi\ simulated datasets 
are shown in \cref{fig:prod:effs:truth:fit}.
Table~\ref{tab:prod:effs:truth_matching} lists the obtained efficiencies.

\begin{figure}
  \begin{subfigure}[b]{0.5\textwidth}
    \centering
    \includegraphics[width=\textwidth]{production/efficiencies/D0ToKpi_BKGCAT}
    \caption{\PDzero}
    \label{fig:prod:effs:truth:categories:D0ToKpi}
  \end{subfigure}
  \begin{subfigure}[b]{0.5\textwidth}
    \centering
    \includegraphics[width=\textwidth]{production/efficiencies/DpToKpipi_BKGCAT}
    \caption{\PDplus}
    \label{fig:prod:effs:truth:categories:DpToKpipi}
  \end{subfigure}
  \caption{%
    Mass distributions of \ac{MC} 
    \PDzero~(\subref*{fig:prod:effs:truth:categories:D0ToKpi}) and 
    \PDplus~(\subref*{fig:prod:effs:truth:categories:DpToKpipi}) candidates for 
    various background categories, as indicated in the legends.
    The most prominent features are the signal-like peak in the `ghost' 
    distribution, indicative of a truth-matching inefficiency, and the long 
    tail extending below the mass peak in the `other' distribution, which 
    mostly consists of partially reconstructed physics backgrounds such as 
    \decay{\PDzero}{\PKminus\Ppiplus\Ppizero} and 
    \decay{\PDplus}{\PKminus\Ppiplus\Ppiplus\Ppizero}.
  }
  \label{fig:prod:effs:truth:categories}
\end{figure}

\begin{figure}
  \begin{subfigure}[b]{0.5\textwidth}
    \centering
    \includegraphics[width=\textwidth]{production/efficiencies/D0ToKpi_BKGCAT_fit}
    \caption{\DzToKpi}
    \label{fig:prod:effs:truth:fit:D0ToKpi}
  \end{subfigure}
  \begin{subfigure}[b]{0.5\textwidth}
    \centering
    \includegraphics[width=\textwidth]{production/efficiencies/DpToKpipi_BKGCAT_fit}
    \caption{\DpToKpipi}
    \label{fig:prod:effs:truth:fit:DpToKpipi}
  \end{subfigure}
  \caption{%
    Mass distributions of \ac{MC} 
    \PDzero~(\subref*{fig:prod:effs:truth:fit:D0ToKpi}) and 
    \PDplus~(\subref*{fig:prod:effs:truth:fit:DpToKpipi}) candidates not 
    passing the `signal' truth-matching requirement, that is the sum of the 
    ``Ghost'' and ``Other'' categories shown in 
    \cref{fig:prod:effs:truth:categories}.
    Overlaid on each distribution is the result of a maximum likelihood fit, 
    with the total model shown as a blue curve, and the background model as a 
    red dotted line.
  }
  \label{fig:prod:effs:truth:fit}
\end{figure}

\subsection{Tracking efficiency correction}
\label{chap:prod:effs:tracking}

The track reconstruction efficiency, or \emph{tracking efficiency}, measures 
the efficiency for a charged particle to be reconstructed as a track, and is 
part of the definition of the reconstruction efficiency in 
\cref{eqn:prod:effs:reco}.
The \ac{MC} may not accurately model the tracking efficiency, and so a 
data-driven technique is used to compute correction factors based on the 
observed differences between the data and the \ac{MC}.

Only \emph{long} tracks are used in this analysis, which requires that the 
particles create hits in the \velo\ and in the T stations.
Hits in the \ttracker\ are not required to form long tracks, though the 
information can be used to improve the momentum resolution.
To probe the efficiency of long track creation, \JpsiTomumu\ decays are 
reconstructed using a `tag-and-probe' technique, where one particle is fully 
reconstructed as a long track, as usual, and the other track is created using 
track segments made only using information from the muon stations.
The momentum of the latter track, the `probe', can be inferred by treating the 
magnetic field as an optical system that imparts a momentum `kick' in the $x$ 
direction.
By assuming the muon track segment was made by a particle originating from the 
\ac{PV}, the probe segment can be extrapolated back from the muon stations to 
the $yz$ plane where the kick is modelled, and then the difference in slopes 
between the \ac{PV}-plane line and the probe track gives the probe momentum.
The invariant mass of the dimuon system can then be computed, the distribution 
of which for many tag-and-probe candidates can be used to distinguish between 
backgrounds and true \PJpsi decays.
The mass resolution of \JpsiTomumu\ decays reconstructed using long+muon tracks 
is about \SI{200}{\MeVcc}, to be compared with a resolution of about 
\SI{16}{\MeVcc} using long+long tracks.
To improve the mass resolution, the muon tracks are combined with hits in the 
\ttracker, improving the resolution to around 
\SI{57}{\MeVcc}~\cite{Aaij:2014pwa,DeCian:2013zua}.

The tracking efficiency is evaluated as the ratio of \PJpsi mesons 
reconstructed using the long+muon+\ttracker\ method, with those that can 
\emph{also} be reconstructed using the long+long method
\begin{equation}
  \eff_{\text{Tracking}} = 
  \frac{%
    N_{\text{long+long}|\text{long+muon+TT}}
  }{%
    N_{\text{long+muon+TT}}
  }.
  \label{eqn:prod:effs:tracking_eff}
\end{equation}
A long track an event is considered as `matched' to the probe track if there is 
a \SI{70}{\percent} overlap of their hits in the muon stations.
The tag-and-probe method is performed using real data and \ac{MC}, and a ratio 
is formed
\begin{equation}
  \efftracking = \frac{%
    \eff_{\text{Tracking}}^{\text{Data}}
  }{%
    \eff_{\text{Tracking}}^{\text{MC}}
  }.
  \label{eqn:prod:effs:tracking_ratio}
\end{equation}
This ratio is evaluated in bins of probe track momentum \ptot\ and 
pseudorapidity \Eta, as the tracking efficiency is found to vary as a function 
of these variables, and is shown in \cref{fig:prod:effs:tracking_table}.
As it is also found to vary as a function of the total number of reconstructed 
long tracks in the event, the input tracks in the \ac{MC} are weighted such 
that event multiplicity distribution matches that in the data.

The correction factor in \cref{eqn:prod:effs:tracking_ratio} can be applied to 
the tracking efficiency found using any decay, as it is only a function of 
\ptot\ and \Eta.
It is then applied to the reconstruction efficiency in 
\cref{eqn:prod:effs:reco}, where for a single decay the total correction factor 
is the product of the correction factors for all the tracks comprising the 
final states.
Tracking efficiency correction factors for \DzToKpi\ and \DpToKpipi\ are given 
in \cref{tab:prod:effs:tracking:dztokpi,tab:prod:effs:tracking:dptokpipi}.

The correction factors given do not account for the differences between muon 
and hadron interactions with the detector.
These differences will result in different tracking efficiencies, which may not 
be the same between data and \ac{MC}.
The effect of these differences will be discussed in the context of systematic 
uncertainties in \cref{chap:prod:syst:tracking}.

\begin{figure}
  \centering
  \includegraphics[width=\textwidth]{production/efficiencies/tracking_correction_table}
  \caption{%
    Tracking efficiency correction factors \efftracking\ in bins of track 
    momentum and pseudorapidity.
    The value is labelled in each bin, along with the statistical uncertainty 
    on those values due to the finite size of the calibration datasets.
    Bins in which a measurement could not be made are empty.
  }
  \label{fig:prod:effs:tracking_table}
\end{figure}

Combining the truth-matching efficiency and tracking efficiency correction 
factors, the `corrected reconstruction' efficiency $\effreco'$ can be 
considered to be
\begin{equation}
  \effreco' = \efftruth\times\efftracking\times\effreco.
  \label{eqn:prod:effs:reco_corrected}
\end{equation}

\section{Selection}
\label{chap:prod:effs:sel}

At this stage, the efficiency chain includes the effects of the detector 
acceptance, of the reconstruction, of the truth-matching used to evaluate the 
reconstruction efficiency, and of the data-\ac{MC} differences in the tracking 
efficiency.
What it left is the selection efficiencies due to the online and offline 
selections, as described in \cref{chap:prod:sel}.
For simplicity, the `online' and `offline' selection are treated as a single 
requirement, with the caveat that \ac{PID} requirements are not included.
The \ac{PID} efficiency can be computed using data-driven methods, and will be 
described in \cref{chap:prod:effs:pid}.

The \lzero\ trigger selects bunch-bunch crossings randomly at a predetermined 
rate, and so the \lzero\ efficiency is just the ratio of accepted crossings to 
the total number, or equivalently
\begin{equation}
  \efflzero = \frac{%
    R_{\nobias}
  }{%
    \ncollbunches\revfreq
  },
  \label{eqn:prod:effs:sel:l0}
\end{equation}
where $R_{\nobias}$ is the rate at which the \lzero\ trigger was programmed to 
accept crossings; \ncollbunches\ is the number of bunches in the \ac{LHC} which 
collided at the \lhcb\ \acl{LHCIP}; and \revfreq\ is the bunch revolution 
frequency of \SI{11.246}{\kilo\hertz}.
Both $R_{\nobias}$ and \ncollbunches\ can vary as a function of the \ac{LHC} 
fill, and hence so can $\eff_{\nobias}$.
The corresponding values are tabulated in 
\cref{tab:prod:sel:online:l0_nobias_rateeff}.

The remaining \hltone, \hlttwo, and offline selection efficiency is computed as 
the ratio of the number of truth-matched decays passing the full selection (but 
without \ac{PID} requirements) to the number of truth-matched, reconstructed 
decays, giving
\begin{equation}
  \efftrig \times \effoffline = \effselection = \efflzero \times \frac{%
    N_{\text{Selected}|\text{Reconstructed}}
  }{%
    N_{\text{Reconstructed}}
  }.
  \label{eqn:prod:effs:sel}
\end{equation}
Selection efficiencies for \DzToKpi\ and \DpToKpipi\ are given in 
\cref{tab:prod:effs:sel:dztokpi,tab:prod:effs:sel:dptokpipi}.

\section{Particle identification}
\label{chap:prod:effs:pid}

The efficiency of a decay to pass all requirements on the variables that are 
discriminatory between charged particle species is the \ac{PID} efficiency 
\effpid.
The variables are those in the \acf{DLL} family, of which this analysis uses 
the \dllkpi\ variable to discriminate between kaon and pion tracks, with the 
exception of the soft pion in the \DstToDzpi decay on which no \ac{PID} 
requirement is made.
This \namecref{chap:prod:effs:pid} shall describe how the \effpid\ is computed 
using a calibration technique.

It is assumed that the efficiency of a \ac{PID} cut for a given particle type 
can be parameterised as a function of three variables: the track momentum 
\ptot, the track pseudorapidity \Eta, and a measure of the detector occupancy.
These particular variables are chosen as the performance of the \rich\ 
detectors is known to depend them~\cite{Adinolfi:2012qfa}.
As a measure of the detector occupancy, the number \nspd\ of hits in the \spd\ 
detector is used.
If a model can be constructed of the efficiency for a particular \ac{PID} 
requirement as a function of \ptot, \Eta, and \nspd, the model can then be used 
to determine the \ac{PID} efficiency of that requirement on any sample.
The purpose of the calibration technique to build such a model.

A clean sample of \DzToKpi\ decays can be obtained without the use of \ac{PID} 
information.
This sample will be referred to here as the \emph{calibration} sample, and a 
has a \SI{20}{\percent} overlap with the sample used for the cross-section 
measurement, due to the different selection strategy that is required when 
\ac{PID} information is ignored.
To compute the efficiency of a given \ac{PID} requirement on either a kaon or a 
pion track, kaons or pions from the calibration sample are partitioned in 
\ptot, \Eta, and \nspd\ as a three-dimensional histogram.
The number of signal tracks $C_{i}$ in the $i$th bin is computed, and then the 
number of signal decays after the \ac{PID} requirement $C_{i}'$ is computed, 
such that the efficiency in the $i$th bin is
\begin{equation}
  \eff_{i} = \frac{C_{i}'}{C_{i}}.
\end{equation}
It is assumed that the \ptotetanspd\ partitions are sufficiently small that the 
\ac{PID} efficiency within a bin is single-valued.
The normalised three-dimensional histogram of efficiencies is an estimate of 
the \ac{PDF} of the efficiency model for the given \ac{PID} requirement on the 
given particle.

One method to obtain the track counts $C_{i}$ and $C_{i}'$ is to perform two 
maximum likelihood fits to the $\PKminus\Ppiplus$ invariant mass distributions, 
one before and one after the \ac{PID} requirement, where the signal \PDzero 
decays can be modelled as a peaking structure and the combinatorial background 
can be modelled as a linear or exponential function.
This can take a substantial amount of time, however, as many fits must be 
performed and checked.
Instead, a single fit is run on the entire calibration sample, and weights are 
computed that can be used to statistically remove the background contribution 
in variable distributions that are uncorrelated with the invariant 
mass.\footnotemark
As \ptot, \Eta, and \nspd\ are found to be uncorrelated with 
$m(\PKminus\Ppiplus)$, these weights can be used to compute $C_{i}$ by summing 
the weights of the tracks that fall in the $i$th bin.
They can also be used to compute $C_{i}'$ in the same way, as the \dll\ 
variables are also found to be uncorrelated with $m(\PKminus\Ppiplus)$, and a 
cut on a variable is the same as partitioning data into bins.
Once the efficiency in each \ptotetanspd\ bin is computed, the \ac{PID}\ 
efficiency for the same cut can be computed for a track from \emph{any} source 
by looking up which bin it falls into, and assigning the corresponding 
efficiency to it.

\footnotetext{%
  These weights are called \emph{sWeights}~\cite{Pivk:2004ty}.
  % TODO should reference a specific chapter where sWeights are described
  A more complete description of their computation, properties, and use cases 
  will be given in \cref{chap:cpv}.
}

For the calibration sample, the average efficiency across the parameterisation 
space can be expressed as
\begin{equation}
  \effpid = \frac{%
    \displaystyle\sum_{i \in \textnormal{bins}} C_{i}\eff_{i}
    }{%
    \displaystyle\sum_{i \in \textnormal{bins}} C_{i}
    }
\end{equation}
For some other sample of tracks, referred to here as the \emph{reference} 
sample,\footnotemark\ the contents of each bin can be scaled by a weight 
$w_{i}$ based on the fraction of the total sample size in the $i$th bin 
relative to that of the calibration sample
\begin{equation}
  w_{i} = \frac{%
    R_{i}
  }{%
    C_{i}
  }\frac{%
    C
  }{%
    R
  },
  \label{eqn:prod:effs:pid:ref_weight}
\end{equation}
where $C_{i}$ ($R_{i}$) is the number of signal calibration (reference) tracks 
in the $i$th bins, as before, and $C$ ($R$) is the total number of signal 
calibration (reference) tracks in the sample.
The average \ac{PID} efficiency for the weighted reference sample is then
\begin{equation}
  \effpid = \frac{%
    \displaystyle\sum_{i \in \textnormal{bins}} w_{i}C_{i}\eff_{i}
  }{%
    \displaystyle\sum_{i \in \textnormal{bins}} w_{i}C_{i}
  }.
  \label{eqn:prod:effs:pid:single_track_ref_eff}
\end{equation}

\footnotetext{%
  It is the ``reference'' sample because it does not have to be equal to the 
  sample of tracks in data one wishes to know the efficiencies for.
  It can be a sample of simulated data that describes the kinematics of the 
  decay well, or even a different decay mode with similar properties (often 
  called a control channel or a reference channel).
}

For this cross-section analysis, all the final states have multiple charged 
tracks, of different species, upon which \ac{PID} requirements are made.
\Cref{eqn:prod:effs:pid:single_track_ref_eff} gives the average \ac{PID} 
efficiency for a final state where a \ac{PID} requirement is made only on one 
track, and so the formalism must be extended to the multi-track case.
The average efficiency is first modified by substituting 
\cref{eqn:prod:effs:pid:ref_weight} into 
\cref{eqn:prod:effs:pid:single_track_ref_eff} to give
\begin{equation}
  \effpid = \frac{%
    \displaystyle\sum_{i \in \textnormal{bins}} R_{i}\eff_{i}
  }{%
    \displaystyle\sum_{i \in \textnormal{bins}} R_{i}
  }
  = \frac{1}{R}\sum_{i \in \textnormal{bins}} R_{i}\eff_{i},
\end{equation}
and then expanding in terms of $a_{ij}$, the \emph{weight} of the $j$th 
reference track in the $i$th \ptotetanspd\ bin
\begin{equation}
  \effpid = \frac{1}{R}
            \sum_{i \in \textnormal{bins}}\sum_{j} a_{ij}\eff_{i}.
  \label{eqn:prod:effs:pid:expanded_ref_eff}
\end{equation}
The weight term accounts for the possibility of an impure reference sample, 
where, for example, it can be the same type of background-subtracting weight as 
used for the creation the efficiency histogram from the calibration sample.
\Cref{eqn:prod:effs:pid:expanded_ref_eff} can be expressed equivalently as the 
sum over all tracks
\begin{equation}
  \effpid = \frac{1}{R}
            \sum_{t \in \textnormal{tracks}} a_{t}\eff_{t}.
  \label{eqn:prod:effs:pid:expanded_ref_eff_tracks}
\end{equation}
With this, the \ac{PID} efficiency calculation can be thought of as either of 
two equivalent methods: weighting the calibration to look like the reference 
sample, as in \cref{eqn:prod:effs:pid:single_track_ref_eff}, or assigning 
per-track efficiencies to the reference sample based on \ptotetanspd\ bins they 
fall into, as in \cref{eqn:prod:effs:pid:expanded_ref_eff_tracks}.
\Cref{eqn:prod:effs:pid:expanded_ref_eff_tracks} can be generalised to 
multi-track efficiencies by summing over all \emph{decays} in the reference 
sample, replacing the per-track efficiency by a per-event efficiency 
\effpiddecay
\begin{equation}
  \effpid = \frac{1}{R}\sum_{d\in\text{decays}} a_{d}\effpiddecay,
\end{equation}
where $a_{d}$ is the total weight of the decay, and \effpiddecay\ is the 
\ac{PID} efficiency for that decay.
The per-decay weight $a_{d}$ and efficiency \effpiddecay\ for the case in which 
a \ac{PID} requirement is made on only a single track are $a_{t}$ and 
\effpidtrack, whilst for a multi-track decay they are
\begin{equation}
  a_{d} = \sum_{t \in \textnormal{tracks}} a_{t},
  \quad\textnormal{and}\quad
  \effpiddecay = \prod_{t \in \textnormal{tracks}} \effpidtrack
\end{equation}

In this analysis, the reference sample used for the calibration is the fully 
selected data, which includes \ac{PID} cuts applied both in the trigger and 
offline.
The use of a reference sample that includes \ac{PID} cuts is a valid approach 
under the assumption that no efficiency bins have a reference sample size of 
zero~\cite{Anderlini:2202412}.
The benefit is that one does not rely on a proxy or \ac{MC} dataset modelling 
the track kinematics, and the inter-track correlations, correctly.

\subsection{\ac{PID} calibration uncertainties}
\label{chap:prod:effs:pid:uncertainties}

The uncertainty $\unc{\effpidtrack}$ on the per-track efficiency \effpidtrack\ 
is the binomial uncertainty of $C_{i}'$ tracks passing the \ac{PID} selection 
out of the initial $C_{i}$ tracks in the calibration sample.
For the multi-track case, the uncertainty $\unc{\effpiddecay}$ on the \ac{PID} 
efficiency \effpiddecay\ can be derived using Gaussian error propagation
\begin{equation}
  \unc{\effpiddecay} = \sqrt{%
    \sum_{t\phantom{'}\in\text{tracks}}\sum_{t'\in\text{tracks}}
      \rho_{tt'}
      \frac{%
        \unc{\effpidtrack}
      }{%
        \effpidtrack
      }
      \frac{%
        % Bit gross as this assumes we know what the \effpidtrack macro is
        \unc{\eff_{t'}}
      }{%
        \eff_{t'}
      }
  }.
  \label{eqn:prod:effs:pid:decay_uncertainty}
\end{equation}
The correlation term $\rho_{tt'}$ accounts for cases where pairs of tracks in 
the reference decay are of the same species and fall in the same \ptotetanspd\ 
bin, in which case it is equal to one otherwise it is zero.

As the reference sample is of a finite size, they is an additional, associated 
uncertainty on the per-event \ac{PID} efficiency.
However, the reference sample is identical to the sample entering the fits 
described in \cref{chap:prod:fitting}, and so this uncertainty is already 
accounted for in the uncertainty on the fitted signal yields.
The uncertainty due to the finite calibration sample size, given in 
\cref{eqn:prod:effs:pid:decay_uncertainty}, is treated as a systematic 
uncertainty on the cross-section measurement, and is discussed further in 
\cref{chap:prod:syst:pid}.

\subsection{Optimisation on efficiency binning scheme}
\label{chap:prod:effs:pid:binning}

An assumption in the method of \ac{PID} calibration previously described is 
that the \ac{PID} efficiency within a \ptotetanspd\ bin is single-valued.
This can be guaranteed by employing an infinitely fine binning.
In reality, this cannot be used due to the finite size of the calibration 
sample.
There is then a balance between decreasing the bin size such that the 
single-value assumption is satisfied, and increasing the bin size so as to 
minimise the uncertainty on the \ac{PID} efficiency computed within that bin.
This Section will discuss how the binning for the \ac{PID} calibration was 
chosen for this analysis.

\Cref{} illustrates the dependence of the efficiency of the kaon \ac{PID} 
requirement on kaon momentum.
It is noted firstly that the efficiency does strongly vary with track momentum, 
and secondly that the equal-probability binning can lead to neighbouring bins 
where the efficiency is near-constant, as in the 14--\SI{25}{\GeVc} region, 
where a single bin would be sufficient.
In addition, the relative softness of the reference sample distribution 
(black), compared to that of the calibration sample (red), leads to a large 
fraction of the reference data being covered by a single bin ($p < 
\SI{14}{\GeVc}$).
If the efficiency varies strongly within the first bin, the average value of 
the \ac{PID} efficiency in that bin is different for the reference and 
calibration samples, and so estimating the reference efficiency with that 
obtained from the calibration samples will give a biased result.

In an attempt to minimise potential biases in the evaluation of the \ac{PID} 
efficiency on the reference sample, for each dimension of \ptotetanspd\ the 
following iterative procedure is used, assuming each dimension is initially 
partitioned as a single bin:
\begin{enumerate}
  \item Divide each bin into ten equal-probability sub-bins.
  \item For each bin:
    \begin{enumerate}
      \item Compute the \ac{PID} efficiency using the calibration sample in the 
        bin \eff\ and in the ten sub-bins $\eff_{i}$, and evaluate a \chisq\ 
        statistic as
        \begin{equation}
          \chisq = \sum_{i}^{10} \frac{{(\eff - \eff_{i})}^{2}}{\unc{\eff_{i}}^{2}},
          \label{eqn:prod:effs:pid:bin_chisq}
        \end{equation}
      \item Compute the $p$-value of the \chisq\ value given 9 degrees of 
        freedom. If $p < 0.001$, the null hypothesis that the \ac{PID} 
        efficiency within that bin is constant is rejected, and so the bin 
        should be divided in two. Otherwise, the bin is not split.
    \end{enumerate}
  \item For each bin that should be divided in two:
    \begin{enumerate}
      \item Divide the bin into twenty equal probability sub-bins.
      \item For each of the 19 new boundaries, compute the \chisq\ statistic in 
        the union of the sub-bins below the boundary and in the union of the 
        sub-bins above the boundary using \cref{eqn:prod:effs:pid:bin_chisq}, 
        and record the sum of the two \chisq\ values.
      \item Record the boundary with the smallest \chisq\ sum.
      \item If the two sub-bin unions either side of the recorded boundary do 
        not each contain at least 2000 tracks, the original bin is not split.
        Otherwise, the original bin is split in two at the recorded boundary.
    \end{enumerate}
  \item If no additional bins were created in this iteration, stop.  Otherwise, 
    repeat from Step 1
\end{enumerate}
The iteration stops when either all bins satisfy the $p$-value requirement or 
no further divisions are possible due to the lower sample size limit.
The sample size limit of 2000 tracks was chosen to reduce the possibility that 
the three one-dimensional binning schemes could be combined to form a 
three-dimensional binning which contained empty bins.
\Cref{fig:prod:pid:binning:kaon,fig:prod:pid:binning:pion} shows the resulting 
\ptotetanspd\ binning (dark red) for kaon and pion tracks.
It can be seen that there are still bins where the finer binning (blue) shows a 
large variation in the efficiency.
The effect of this on the cross-section measurement is considered as a 
systematic uncertainty in \cref{chap:prod:syst:pid}.

The \acl{PID} efficiencies computed using the discussed calibration procedure 
are given in \cref{tab:prod:effs:pid:dztokpi,tab:prod:effs:pid:dptokpipi} for 
\DzToKpi\ and \DpToKpipi.
The uncertainties given are those due to the finite size of the calibration 
sample.

\section{Signal window}
\label{chap:prod:effs:signal_window}

As described in Section~\ref{chap:prod:fitting}, the number of prompt signal 
candidates is measured by fitting the \lnipchisq\ distribution of candidates in 
a signal window defined in the mass distribution.
This mass window requirement removes some signal candidates.
The efficiency of the requirement is evaluated by computing the fractional 
integral of the signal component of the model $f_{\text{Sig.}}(m)$ fitted to 
the mass distribution
\begin{equation}
  \effsigwin = \frac{%
    \int_{M - \delta_{M}}^{M +\delta_{M}} f_{\text{Sig.}}(m) \dif{m}
  }{%
    \int_{-\infty}^{\infty} f_{\text{Sig.}}(m) \dif{m}
  }
  = \int_{M - \delta_{M}}^{M +\delta_{M}} f_{\text{Sig.}}(m) \dif{m},
    \label{eqn:prod:effs:signal_window}
\end{equation}
where $M$ is the nominal charm meson mass, on which the signal window is 
centred, and $\delta_{M}$ is half the width of the signal window.
In the case of the \PDstarp measurement, there is an additional requirement 
that the candidates used in the \lnipchisq\ fit fall in a signal window defined 
in the \deltam\ distribution.
The efficiency of the \deltam\ signal window requirement is computed in the 
same way as for the mass signal window requirement, except the signal component 
of the model fitted to the \deltam\ distribution $g_{\text{Sig}}(\deltam)$ is 
used.
The total signal window for the \PDstarp\ measurement is then
\begin{equation}
  \effsigwin^{\PDstarp} =
    \int_{\deltam_{0} - \delta_{\deltam}}^{\deltam_{0} +\delta_{\deltam}} g_{\text{Sig.}}(\deltam) \dif{\deltam}
    \int_{M - \delta_{M}}^{M +\delta_{M}} f_{\text{Sig.}}(m) \dif{m},
    \label{eqn:prod:effs:signal_window:dst}
\end{equation}
where $\deltam_{0}$ is the nominal \PDstarp-\PDzero mass difference, and $\delta_{\deltam}$ is half the width of the \deltam\ window.

The integral of a probability density function, as in \cref{eqn:prod:effs:signal_window,eqn:prod:effs:signal_window:dst}, is a number.
To assign an uncertainty to this number, the fractional integral is recomputed 
in a series of 500 pseudo-experiments.
Within each one, each shape parameter is assigned a value sampled from a 
normal distribution with a mean as the `nominal' value of that parameter, the 
value found by the fit, and with a width equal to the uncertainty on that nominal 
value. Correlations between the different shape parameters are taken into account.
The variance on the deviation of the integrals, with respect to the integral 
computed using the nominal shape parameter values, is taken as the square of 
the uncertainty on the nominal integral value.

As the shape of some mass signal models are \pTy\ bin dependent, as described 
in Section~\ref{chap:prod:fitting:details}, the signal window efficiency is computed 
separately in each \pTy\ bin.
\Cref{tab:prod:effs:sigwin:dztokpi,tab:prod:effs:sigwin:dptokpipi}
give the per-bin signal window efficiencies for each mode.

\section{Combination of efficiencies and their uncertainties}
\label{chap:prod:effs:tot}

% TODO describe how we actually compute the reco+sel eff as one step, and how
% uncertainties are treated (i.e. not using the standard binomial approach)
% TODO normalise tables. D+ tables use [0, 1000[, D0 use [0, 1000]

For the efficiencies computed by counting the number $N$ of decays before and the number $k$ after some selection, the estimate of the true efficiency is $k/N$.
The application of the selection on the $N$ events is a binomial process, with the probability $P$ to observe $k$ counts given the true efficiency \eff\ and the number of initial $N$ counts given by
\begin{equation}
  P(k | \eff, N) = \begin{pmatrix}n\\k\end{pmatrix}\eff^{k}{(1 - \eff)}^{N - k}.
\end{equation}
% TODO need to say that the standard variance is broken
To exploit the \acl{MC} error propagation technique discussed in \cref{}, the posterior \ac{PDF} $P(\eff|k, N)$ must be obtained.
This can be done using Bayes' theorem as
\begin{equation}
  P(\eff|k,n) \propto P(k|\eff,n)P(\eff|n).
\end{equation}
The normalisation of this Equation depends on the choice of prior $P(\eff|n)$, which is chosen to be the uniform distribution and gives
\begin{equation}
  P(\eff|k,n) \propto \eff^{k}{(1 - \eff)}^{n - k}.
\end{equation}
% TODO finish

% These tables are big and ugly, so stick them all at the end of the Chapter
\begin{table}
  \caption{%
    Acceptance efficiencies \effacc\ for \DzToKpi\ measured in \PDzero \pTy\ 
    bins.
  }
  \label{tab:prod:effs:acc:dztokpi}
  \centering
  \renewcommand{\arraystretch}{1.3}
\begin{tabular}{lr@{\hskip+0.2em}c@{\hskip+0.2em}r@{\hskip+0.2em}c@{\hskip+0.2em}rr@{\hskip+0.2em}c@{\hskip+0.2em}r@{\hskip+0.2em}c@{\hskip+0.2em}rr@{\hskip+0.2em}c@{\hskip+0.2em}r@{\hskip+0.2em}c@{\hskip+0.2em}rr@{\hskip+0.2em}c@{\hskip+0.2em}r@{\hskip+0.2em}c@{\hskip+0.2em}rr@{\hskip+0.2em}c@{\hskip+0.2em}r@{\hskip+0.2em}c@{\hskip+0.2em}r}
\toprule&\multicolumn{25}{c}{$\text{$y$}$}\\
$\text{$p_{\text{T}}$ [\text{MeV}/c]}$ & \multicolumn{5}{c}{$[2,2.5]$} & \multicolumn{5}{c}{$[2.5,3]$} & \multicolumn{5}{c}{$[3,3.5]$} & \multicolumn{5}{c}{$[3.5,4]$} & \multicolumn{5}{c}{$[4,4.5]$} \\
\midrule$[14000,15000]$ & $100.00$ & $^+_-$ & $^{0.00}_{1.67}$ & &  & $100.00$ & $^+_-$ & $^{0.00}_{2.07}$ & &  & $100.00$ & $^+_-$ & $^{0.00}_{2.35}$ & &  & $100.00$ & $^+_-$ & $^{0.00}_{5.69}$ & &  & $93.75$ & $^+_-$ & $^{4.30}_{9.43}$ & &  \\
$[13000,14000]$ & $100.00$ & $^+_-$ & $^{0.00}_{1.32}$ & &  & $100.00$ & $^+_-$ & $^{0.00}_{1.91}$ & &  & $100.00$ & $^+_-$ & $^{0.00}_{2.67}$ & &  & $100.00$ & $^+_-$ & $^{0.00}_{3.33}$ & &  & $100.00$ & $^+_-$ & $^{0.00}_{3.86}$ & &  \\
$[12000,13000]$ & $100.00$ & $^+_-$ & $^{0.00}_{0.88}$ & &  & $100.00$ & $^+_-$ & $^{0.00}_{1.37}$ & &  & $100.00$ & $^+_-$ & $^{0.00}_{1.23}$ & &  & $100.00$ & $^+_-$ & $^{0.00}_{2.04}$ & &  & $100.00$ & $^+_-$ & $^{0.00}_{4.13}$ & &  \\
$[11000,12000]$ & $97.65$ & $^+_-$ & $^{0.94}_{1.50}$ & &  & $100.00$ & $^+_-$ & $^{0.00}_{0.81}$ & &  & $100.00$ & $^+_-$ & $^{0.00}_{1.13}$ & &  & $100.00$ & $^+_-$ & $^{0.00}_{1.40}$ & &  & $100.00$ & $^+_-$ & $^{0.00}_{2.50}$ & &  \\
$[10000,11000]$ & $99.65$ & $^+_-$ & $^{0.25}_{0.60}$ & &  & $100.00$ & $^+_-$ & $^{0.00}_{0.63}$ & &  & $100.00$ & $^+_-$ & $^{0.00}_{0.70}$ & &  & $100.00$ & $^+_-$ & $^{0.00}_{0.99}$ & &  & $100.00$ & $^+_-$ & $^{0.00}_{1.50}$ & &  \\
$[9000,10000]$ & $99.28$ & $^+_-$ & $^{0.33}_{0.57}$ & &  & $100.00$ & $^+_-$ & $^{0.00}_{0.36}$ & &  & $100.00$ & $^+_-$ & $^{0.00}_{0.45}$ & &  & $100.00$ & $^+_-$ & $^{0.00}_{0.62}$ & &  & $99.19$ & $^+_-$ & $^{0.58}_{1.38}$ & &  \\
$[8000,9000]$ & $98.48$ & $^+_-$ & $^{0.43}_{0.60}$ & &  & $100.00$ & $^+_-$ & $^{0.00}_{0.26}$ & &  & $100.00$ & $^+_-$ & $^{0.00}_{0.30}$ & &  & $100.00$ & $^+_-$ & $^{0.00}_{0.41}$ & &  & $98.51$ & $^+_-$ & $^{0.68}_{1.16}$ & &  \\
$[7000,8000]$ & $98.62$ & $^+_-$ & $^{0.33}_{0.44}$ & &  & $99.87$ & $^+_-$ & $^{0.09}_{0.21}$ & &  & $100.00$ & $^+_-$ & $^{0.00}_{0.17}$ & &  & $99.80$ & $^+_-$ & $^{0.14}_{0.34}$ & &  & $99.69$ & $^+_-$ & $^{0.23}_{0.53}$ & &  \\
$[6000,7000]$ & $96.75$ & $^+_-$ & $^{0.41}_{0.46}$ & &  & $100.00$ & $^+_-$ & $^{0.00}_{0.09}$ & &  & $99.90$ & $^+_-$ & $^{0.07}_{0.16}$ & &  & $99.66$ & $^+_-$ & $^{0.15}_{0.27}$ & &  & $99.26$ & $^+_-$ & $^{0.30}_{0.48}$ & &  \\
$[5000,6000]$ & $95.83$ & $^+_-$ & $^{0.37}_{0.40}$ & &  & $99.87$ & $^+_-$ & $^{0.06}_{0.10}$ & &  & $99.64$ & $^+_-$ & $^{0.11}_{0.17}$ & &  & $99.56$ & $^+_-$ & $^{0.14}_{0.20}$ & &  & $98.10$ & $^+_-$ & $^{0.38}_{0.47}$ & &  \\
$[4000,5000]$ & $94.34$ & $^+_-$ & $^{0.33}_{0.35}$ & &  & $99.32$ & $^+_-$ & $^{0.12}_{0.14}$ & &  & $99.73$ & $^+_-$ & $^{0.08}_{0.11}$ & &  & $99.19$ & $^+_-$ & $^{0.16}_{0.19}$ & &  & $96.22$ & $^+_-$ & $^{0.40}_{0.44}$ & &  \\
$[3500,4000]$ & $91.66$ & $^+_-$ & $^{0.45}_{0.48}$ & &  & $98.56$ & $^+_-$ & $^{0.20}_{0.23}$ & &  & $99.70$ & $^+_-$ & $^{0.09}_{0.13}$ & &  & $98.51$ & $^+_-$ & $^{0.24}_{0.29}$ & &  & $94.09$ & $^+_-$ & $^{0.56}_{0.61}$ & &  \\
$[3000,3500]$ & $89.45$ & $^+_-$ & $^{0.44}_{0.45}$ & &  & $98.31$ & $^+_-$ & $^{0.19}_{0.21}$ & &  & $98.93$ & $^+_-$ & $^{0.16}_{0.18}$ & &  & $97.07$ & $^+_-$ & $^{0.29}_{0.33}$ & &  & $91.56$ & $^+_-$ & $^{0.56}_{0.60}$ & &  \\
$[2500,3000]$ & $87.75$ & $^+_-$ & $^{0.40}_{0.42}$ & &  & $97.14$ & $^+_-$ & $^{0.21}_{0.23}$ & &  & $98.70$ & $^+_-$ & $^{0.15}_{0.17}$ & &  & $95.95$ & $^+_-$ & $^{0.29}_{0.32}$ & &  & $89.27$ & $^+_-$ & $^{0.54}_{0.57}$ & &  \\
$[2000,2500]$ & $84.59$ & $^+_-$ & $^{0.38}_{0.39}$ & &  & $96.22$ & $^+_-$ & $^{0.21}_{0.22}$ & &  & $98.30$ & $^+_-$ & $^{0.15}_{0.16}$ & &  & $95.79$ & $^+_-$ & $^{0.26}_{0.27}$ & &  & $87.13$ & $^+_-$ & $^{0.50}_{0.51}$ & &  \\
$[1500,2000]$ & $81.59$ & $^+_-$ & $^{0.36}_{0.37}$ & &  & $94.56$ & $^+_-$ & $^{0.22}_{0.23}$ & &  & $97.60$ & $^+_-$ & $^{0.15}_{0.17}$ & &  & $94.42$ & $^+_-$ & $^{0.26}_{0.27}$ & &  & $83.44$ & $^+_-$ & $^{0.47}_{0.48}$ & &  \\
$[1000,1500]$ & $78.50$ & $^+_-$ & $^{0.34}_{0.35}$ & &  & $93.35$ & $^+_-$ & $^{0.22}_{0.22}$ & &  & $96.34$ & $^+_-$ & $^{0.17}_{0.18}$ & &  & $93.02$ & $^+_-$ & $^{0.25}_{0.26}$ & &  & $80.22$ & $^+_-$ & $^{0.44}_{0.45}$ & &  \\
$[0,1000]$ & $76.41$ & $^+_-$ & $^{0.29}_{0.29}$ & &  & $91.57$ & $^+_-$ & $^{0.20}_{0.20}$ & &  & $95.30$ & $^+_-$ & $^{0.15}_{0.16}$ & &  & $91.45$ & $^+_-$ & $^{0.22}_{0.22}$ & &  & $77.80$ & $^+_-$ & $^{0.35}_{0.36}$ & &  \\
\bottomrule\end{tabular}

\end{table}

\begin{table}
  \caption{%
    Acceptance efficiencies \effacc\ for \DpToKpipi\ measured in \PDplus \pTy\ 
    bins.
  }
  \label{tab:prod:effs:acc:dptokpipi}
  \centering
  \renewcommand{\arraystretch}{1.3}
\begin{tabular}{lr@{\hskip+0.2em}c@{\hskip+0.2em}r@{\hskip+0.2em}c@{\hskip+0.2em}rr@{\hskip+0.2em}c@{\hskip+0.2em}r@{\hskip+0.2em}c@{\hskip+0.2em}rr@{\hskip+0.2em}c@{\hskip+0.2em}r@{\hskip+0.2em}c@{\hskip+0.2em}rr@{\hskip+0.2em}c@{\hskip+0.2em}r@{\hskip+0.2em}c@{\hskip+0.2em}rr@{\hskip+0.2em}c@{\hskip+0.2em}r@{\hskip+0.2em}c@{\hskip+0.2em}r}
\toprule&\multicolumn{25}{c}{$\text{$y$}$}\\
$\text{$p_{\text{T}}$ [\text{MeV}/c]}$ & \multicolumn{5}{c}{$[2,2.5]$} & \multicolumn{5}{c}{$[2.5,3]$} & \multicolumn{5}{c}{$[3,3.5]$} & \multicolumn{5}{c}{$[3.5,4]$} & \multicolumn{5}{c}{$[4,4.5]$} \\
\midrule
$[0,1000]$ & $67.05$ & $^+_-$ & $^{0.34}_{0.34}$ & &  & $88.70$ & $^+_-$ & $^{0.24}_{0.24}$ & &  & $93.18$ & $^+_-$ & $^{0.19}_{0.20}$ & &  & $83.11$ & $^+_-$ & $^{0.31}_{0.32}$ & &  & $59.35$ & $^+_-$ & $^{0.44}_{0.44}$ & &  \\
$[1000,1500]$ & $70.61$ & $^+_-$ & $^{0.40}_{0.41}$ & &  & $91.98$ & $^+_-$ & $^{0.25}_{0.26}$ & &  & $95.14$ & $^+_-$ & $^{0.21}_{0.22}$ & &  & $88.42$ & $^+_-$ & $^{0.33}_{0.34}$ & &  & $69.86$ & $^+_-$ & $^{0.53}_{0.53}$ & &  \\
$[1500,2000]$ & $75.54$ & $^+_-$ & $^{0.42}_{0.43}$ & &  & $94.67$ & $^+_-$ & $^{0.23}_{0.24}$ & &  & $97.06$ & $^+_-$ & $^{0.18}_{0.19}$ & &  & $91.77$ & $^+_-$ & $^{0.32}_{0.34}$ & &  & $76.51$ & $^+_-$ & $^{0.56}_{0.57}$ & &  \\
$[2000,2500]$ & $80.04$ & $^+_-$ & $^{0.45}_{0.46}$ & &  & $96.60$ & $^+_-$ & $^{0.21}_{0.23}$ & &  & $98.23$ & $^+_-$ & $^{0.16}_{0.18}$ & &  & $94.44$ & $^+_-$ & $^{0.32}_{0.33}$ & &  & $82.68$ & $^+_-$ & $^{0.59}_{0.60}$ & &  \\
$[2500,3000]$ & $83.81$ & $^+_-$ & $^{0.49}_{0.50}$ & &  & $98.08$ & $^+_-$ & $^{0.18}_{0.20}$ & &  & $98.42$ & $^+_-$ & $^{0.18}_{0.20}$ & &  & $96.32$ & $^+_-$ & $^{0.30}_{0.32}$ & &  & $88.16$ & $^+_-$ & $^{0.59}_{0.62}$ & &  \\
$[3000,3500]$ & $87.14$ & $^+_-$ & $^{0.50}_{0.52}$ & &  & $98.86$ & $^+_-$ & $^{0.16}_{0.18}$ & &  & $99.01$ & $^+_-$ & $^{0.16}_{0.19}$ & &  & $96.96$ & $^+_-$ & $^{0.32}_{0.35}$ & &  & $91.10$ & $^+_-$ & $^{0.60}_{0.64}$ & &  \\
$[3500,4000]$ & $89.87$ & $^+_-$ & $^{0.53}_{0.55}$ & &  & $99.01$ & $^+_-$ & $^{0.17}_{0.21}$ & &  & $99.31$ & $^+_-$ & $^{0.15}_{0.19}$ & &  & $97.82$ & $^+_-$ & $^{0.31}_{0.35}$ & &  & $93.66$ & $^+_-$ & $^{0.61}_{0.67}$ & &  \\
$[4000,5000]$ & $92.32$ & $^+_-$ & $^{0.41}_{0.43}$ & &  & $99.59$ & $^+_-$ & $^{0.09}_{0.12}$ & &  & $99.49$ & $^+_-$ & $^{0.11}_{0.15}$ & &  & $98.35$ & $^+_-$ & $^{0.24}_{0.28}$ & &  & $95.20$ & $^+_-$ & $^{0.48}_{0.53}$ & &  \\
$[5000,6000]$ & $95.96$ & $^+_-$ & $^{0.38}_{0.42}$ & &  & $99.76$ & $^+_-$ & $^{0.09}_{0.13}$ & &  & $99.56$ & $^+_-$ & $^{0.13}_{0.19}$ & &  & $98.64$ & $^+_-$ & $^{0.28}_{0.36}$ & &  & $97.23$ & $^+_-$ & $^{0.47}_{0.57}$ & &  \\
$[6000,7000]$ & $96.79$ & $^+_-$ & $^{0.44}_{0.50}$ & &  & $100.00$ & $^+_-$ & $^{0.00}_{0.10}$ & &  & $99.50$ & $^+_-$ & $^{0.18}_{0.28}$ & &  & $99.15$ & $^+_-$ & $^{0.27}_{0.39}$ & &  & $97.21$ & $^+_-$ & $^{0.61}_{0.78}$ & &  \\
$[7000,8000]$ & $98.68$ & $^+_-$ & $^{0.33}_{0.44}$ & &  & $99.87$ & $^+_-$ & $^{0.10}_{0.23}$ & &  & $99.51$ & $^+_-$ & $^{0.22}_{0.38}$ & &  & $99.58$ & $^+_-$ & $^{0.22}_{0.43}$ & &  & $99.67$ & $^+_-$ & $^{0.24}_{0.56}$ & &  \\
$[8000,9000]$ & $99.46$ & $^+_-$ & $^{0.24}_{0.42}$ & &  & $99.80$ & $^+_-$ & $^{0.15}_{0.35}$ & &  & $100.00$ & $^+_-$ & $^{0.00}_{0.32}$ & &  & $98.31$ & $^+_-$ & $^{0.62}_{0.94}$ & &  & $98.91$ & $^+_-$ & $^{0.58}_{1.11}$ & &  \\
$[9000,10000]$ & $99.46$ & $^+_-$ & $^{0.29}_{0.55}$ & &  & $100.00$ & $^+_-$ & $^{0.00}_{0.36}$ & &  & $100.00$ & $^+_-$ & $^{0.00}_{0.46}$ & &  & $99.42$ & $^+_-$ & $^{0.42}_{0.98}$ & &  & $98.25$ & $^+_-$ & $^{0.94}_{1.78}$ & &  \\
$[10000,11000]$ & $99.60$ & $^+_-$ & $^{0.29}_{0.68}$ & &  & $100.00$ & $^+_-$ & $^{0.00}_{0.64}$ & &  & $100.00$ & $^+_-$ & $^{0.00}_{0.67}$ & &  & $99.12$ & $^+_-$ & $^{0.64}_{1.49}$ & &  & $100.00$ & $^+_-$ & $^{0.00}_{1.72}$ & &  \\
$[11000,12000]$ & $100.00$ & $^+_-$ & $^{0.00}_{0.75}$ & &  & $100.00$ & $^+_-$ & $^{0.00}_{0.88}$ & &  & $100.00$ & $^+_-$ & $^{0.00}_{1.27}$ & &  & $100.00$ & $^+_-$ & $^{0.00}_{1.37}$ & &  & $100.00$ & $^+_-$ & $^{0.00}_{2.61}$ & &  \\
$[12000,13000]$ & $100.00$ & $^+_-$ & $^{0.00}_{1.05}$ & &  & $100.00$ & $^+_-$ & $^{0.00}_{1.05}$ & &  & $100.00$ & $^+_-$ & $^{0.00}_{1.62}$ & &  & $100.00$ & $^+_-$ & $^{0.00}_{2.61}$ & &  & $100.00$ & $^+_-$ & $^{0.00}_{4.79}$ & &  \\
$[13000,14000]$ & $100.00$ & $^+_-$ & $^{0.00}_{1.50}$ & &  & $100.00$ & $^+_-$ & $^{0.00}_{1.91}$ & &  & $100.00$ & $^+_-$ & $^{0.00}_{2.04}$ & &  & $100.00$ & $^+_-$ & $^{0.00}_{3.86}$ & &  & $100.00$ & $^+_-$ & $^{0.00}_{3.53}$ & &  \\
$[14000,15000]$ & $100.00$ & $^+_-$ & $^{0.00}_{2.40}$ & &  & $100.00$ & $^+_-$ & $^{0.00}_{2.40}$ & &  & $100.00$ & $^+_-$ & $^{0.00}_{3.00}$ & &  & $100.00$ & $^+_-$ & $^{0.00}_{4.99}$ & &  & $100.00$ & $^+_-$ & $^{0.00}_{7.95}$ & &  \\
\bottomrule\end{tabular}

\end{table}

\begin{table}
  \caption{%
    Reconstruction efficiencies \effreco\ for \DzToKpi\ measured in \PDzero 
    \pTy\ bins.
  }
  \label{tab:prod:effs:reco:dztokpi}
  \centering
  \renewcommand{\arraystretch}{1.3}
\begin{tabular}{lccccc}
\toprule&\multicolumn{5}{c}{$\text{$y$}$}\\
$\text{$p_{\text{T}}$ [\text{MeV}/c]}$ & $[2,2.5]$ & $[2.5,3]$ & $[3,3.5]$ & $[3.5,4]$ & $[4,4.5]$ \\
\midrule
$[0,1000]$ & $18.02^{+0.13}_{-0.13}$ & $38.33^{+0.15}_{-0.15}$ & $47.13^{+0.16}_{-0.16}$ & $45.23^{+0.18}_{-0.18}$ & $32.30^{+0.23}_{-0.23}$ \\
$[1000,1500]$ & $19.83^{+0.15}_{-0.15}$ & $39.65^{+0.17}_{-0.17}$ & $48.14^{+0.18}_{-0.18}$ & $45.81^{+0.19}_{-0.19}$ & $32.59^{+0.21}_{-0.21}$ \\
$[1500,2000]$ & $21.07^{+0.16}_{-0.15}$ & $42.66^{+0.18}_{-0.18}$ & $50.40^{+0.19}_{-0.19}$ & $47.87^{+0.20}_{-0.20}$ & $34.09^{+0.22}_{-0.22}$ \\
$[2000,2500]$ & $23.73^{+0.17}_{-0.17}$ & $45.20^{+0.20}_{-0.20}$ & $52.59^{+0.21}_{-0.21}$ & $50.76^{+0.22}_{-0.22}$ & $37.56^{+0.25}_{-0.25}$ \\
$[2500,3000]$ & $25.73^{+0.20}_{-0.20}$ & $47.72^{+0.22}_{-0.22}$ & $55.08^{+0.23}_{-0.23}$ & $53.32^{+0.25}_{-0.25}$ & $40.25^{+0.28}_{-0.28}$ \\
$[3000,3500]$ & $28.43^{+0.23}_{-0.23}$ & $50.24^{+0.25}_{-0.25}$ & $56.76^{+0.26}_{-0.26}$ & $55.78^{+0.29}_{-0.29}$ & $43.01^{+0.33}_{-0.33}$ \\
$[3500,4000]$ & $30.34^{+0.27}_{-0.27}$ & $51.59^{+0.29}_{-0.29}$ & $58.42^{+0.31}_{-0.31}$ & $56.82^{+0.33}_{-0.33}$ & $44.93^{+0.39}_{-0.39}$ \\
$[4000,5000]$ & $32.73^{+0.24}_{-0.24}$ & $53.86^{+0.26}_{-0.26}$ & $60.06^{+0.27}_{-0.27}$ & $58.29^{+0.30}_{-0.30}$ & $46.44^{+0.35}_{-0.35}$ \\
$[5000,6000]$ & $35.49^{+0.32}_{-0.32}$ & $55.24^{+0.34}_{-0.34}$ & $61.62^{+0.36}_{-0.37}$ & $60.30^{+0.40}_{-0.41}$ & $48.14^{+0.48}_{-0.48}$ \\
$[6000,7000]$ & $36.87^{+0.43}_{-0.42}$ & $56.76^{+0.45}_{-0.45}$ & $62.26^{+0.48}_{-0.49}$ & $62.14^{+0.54}_{-0.55}$ & $49.30^{+0.66}_{-0.66}$ \\
$[7000,8000]$ & $40.13^{+0.56}_{-0.56}$ & $59.07^{+0.59}_{-0.60}$ & $63.58^{+0.63}_{-0.64}$ & $63.37^{+0.72}_{-0.73}$ & $49.26^{+0.91}_{-0.91}$ \\
$[8000,9000]$ & $44.84^{+0.72}_{-0.72}$ & $63.71^{+0.74}_{-0.74}$ & $67.73^{+0.79}_{-0.80}$ & $66.13^{+0.92}_{-0.93}$ & $53.24^{+1.13}_{-1.13}$ \\
$[9000,10000]$ & $42.14^{+0.21}_{-0.21}$ & $58.50^{+0.23}_{-0.23}$ & $62.46^{+0.25}_{-0.25}$ & $61.69^{+0.29}_{-0.30}$ & $50.16^{+0.38}_{-0.38}$ \\
$[10000,11000]$ & $43.57^{+0.22}_{-0.22}$ & $59.30^{+0.23}_{-0.23}$ & $62.32^{+0.26}_{-0.26}$ & $61.79^{+0.31}_{-0.31}$ & $49.33^{+0.39}_{-0.39}$ \\
$[11000,12000]$ & $44.30^{+0.26}_{-0.26}$ & $59.05^{+0.29}_{-0.29}$ & $60.99^{+0.32}_{-0.32}$ & $62.21^{+0.38}_{-0.38}$ & $50.12^{+0.50}_{-0.50}$ \\
$[12000,13000]$ & $44.79^{+0.32}_{-0.32}$ & $58.79^{+0.35}_{-0.35}$ & $62.00^{+0.39}_{-0.39}$ & $61.65^{+0.47}_{-0.47}$ & $49.14^{+0.63}_{-0.63}$ \\
$[13000,14000]$ & $45.09^{+0.38}_{-0.38}$ & $58.05^{+0.42}_{-0.42}$ & $61.85^{+0.47}_{-0.48}$ & $61.14^{+0.58}_{-0.58}$ & $50.36^{+0.77}_{-0.77}$ \\
$[14000,15000[$ & $44.90^{+0.45}_{-0.45}$ & $58.84^{+0.49}_{-0.50}$ & $61.51^{+0.57}_{-0.57}$ & $61.41^{+0.69}_{-0.70}$ & $46.16^{+0.94}_{-0.94}$ \\
\bottomrule\end{tabular}

\end{table}

\begin{table}
  \caption{%
    Reconstruction efficiencies \effreco\ for \DpToKpipi\ measured in \PDplus 
    \pTy\ bins.
  }
  \label{tab:prod:effs:reco:dptokpipi}
  \centering
  \renewcommand{\arraystretch}{1.3}
\begin{tabular}{l|r@{\hskip+0.2em}c@{\hskip+0.2em}r@{\hskip+0.2em}c@{\hskip+0.2em}rr@{\hskip+0.2em}c@{\hskip+0.2em}r@{\hskip+0.2em}c@{\hskip+0.2em}rr@{\hskip+0.2em}c@{\hskip+0.2em}r@{\hskip+0.2em}c@{\hskip+0.2em}rr@{\hskip+0.2em}c@{\hskip+0.2em}r@{\hskip+0.2em}c@{\hskip+0.2em}rr@{\hskip+0.2em}c@{\hskip+0.2em}r@{\hskip+0.2em}c@{\hskip+0.2em}r}
\toprule&\multicolumn{25}{c}{$\text{$y$}$}\\
$\text{$p_{\text{T}}$ [\text{MeV}/c]}$ & \multicolumn{5}{c}{$[2,2.5]$} & \multicolumn{5}{c}{$[2.5,3]$} & \multicolumn{5}{c}{$[3,3.5]$} & \multicolumn{5}{c}{$[3.5,4]$} & \multicolumn{5}{c}{$[4,4.5]$} \\
\midrule$[14000,15000]$ & $29.32$ & $^+_-$ & $^{0.37}_{0.37}$ & &  & $44.56$ & $^+_-$ & $^{0.44}_{0.44}$ & &  & $48.73$ & $^+_-$ & $^{0.52}_{0.52}$ & &  & $47.59$ & $^+_-$ & $^{0.63}_{0.63}$ & &  & $33.91$ & $^+_-$ & $^{0.79}_{0.78}$ & &  \\
$[13000,14000]$ & $29.96$ & $^+_-$ & $^{0.32}_{0.31}$ & &  & $45.02$ & $^+_-$ & $^{0.38}_{0.38}$ & &  & $49.19$ & $^+_-$ & $^{0.43}_{0.43}$ & &  & $47.71$ & $^+_-$ & $^{0.52}_{0.52}$ & &  & $33.37$ & $^+_-$ & $^{0.65}_{0.64}$ & &  \\
$[12000,13000]$ & $28.86$ & $^+_-$ & $^{0.26}_{0.26}$ & &  & $45.33$ & $^+_-$ & $^{0.31}_{0.31}$ & &  & $49.59$ & $^+_-$ & $^{0.36}_{0.36}$ & &  & $49.01$ & $^+_-$ & $^{0.43}_{0.43}$ & &  & $33.18$ & $^+_-$ & $^{0.53}_{0.52}$ & &  \\
$[11000,12000]$ & $28.20$ & $^+_-$ & $^{0.22}_{0.22}$ & &  & $45.31$ & $^+_-$ & $^{0.26}_{0.26}$ & &  & $48.61$ & $^+_-$ & $^{0.29}_{0.29}$ & &  & $48.78$ & $^+_-$ & $^{0.35}_{0.35}$ & &  & $33.95$ & $^+_-$ & $^{0.42}_{0.42}$ & &  \\
$[10000,11000]$ & $26.73$ & $^+_-$ & $^{0.18}_{0.17}$ & &  & $44.57$ & $^+_-$ & $^{0.21}_{0.21}$ & &  & $49.27$ & $^+_-$ & $^{0.24}_{0.24}$ & &  & $47.65$ & $^+_-$ & $^{0.29}_{0.29}$ & &  & $33.79$ & $^+_-$ & $^{0.35}_{0.35}$ & &  \\
$[9000,10000]$ & $31.30$ & $^+_-$ & $^{0.74}_{0.73}$ & &  & $50.40$ & $^+_-$ & $^{0.79}_{0.79}$ & &  & $56.39$ & $^+_-$ & $^{0.80}_{0.81}$ & &  & $52.83$ & $^+_-$ & $^{0.83}_{0.83}$ & &  & $37.42$ & $^+_-$ & $^{0.90}_{0.89}$ & &  \\
$[8000,9000]$ & $24.06$ & $^+_-$ & $^{0.62}_{0.61}$ & &  & $44.13$ & $^+_-$ & $^{0.76}_{0.75}$ & &  & $49.20$ & $^+_-$ & $^{0.85}_{0.85}$ & &  & $49.41$ & $^+_-$ & $^{0.98}_{0.98}$ & &  & $33.08$ & $^+_-$ & $^{1.14}_{1.12}$ & &  \\
$[7000,8000]$ & $23.55$ & $^+_-$ & $^{0.48}_{0.48}$ & &  & $41.90$ & $^+_-$ & $^{0.59}_{0.59}$ & &  & $48.49$ & $^+_-$ & $^{0.65}_{0.65}$ & &  & $48.03$ & $^+_-$ & $^{0.73}_{0.73}$ & &  & $34.60$ & $^+_-$ & $^{0.86}_{0.85}$ & &  \\
$[6000,7000]$ & $20.46$ & $^+_-$ & $^{0.36}_{0.35}$ & &  & $40.01$ & $^+_-$ & $^{0.45}_{0.45}$ & &  & $46.55$ & $^+_-$ & $^{0.50}_{0.50}$ & &  & $45.80$ & $^+_-$ & $^{0.56}_{0.56}$ & &  & $31.61$ & $^+_-$ & $^{0.62}_{0.61}$ & &  \\
$[5000,6000]$ & $17.90$ & $^+_-$ & $^{0.26}_{0.26}$ & &  & $38.05$ & $^+_-$ & $^{0.33}_{0.33}$ & &  & $45.64$ & $^+_-$ & $^{0.37}_{0.37}$ & &  & $45.53$ & $^+_-$ & $^{0.41}_{0.41}$ & &  & $31.44$ & $^+_-$ & $^{0.45}_{0.45}$ & &  \\
$[4000,5000]$ & $15.34$ & $^+_-$ & $^{0.18}_{0.18}$ & &  & $35.45$ & $^+_-$ & $^{0.25}_{0.25}$ & &  & $44.07$ & $^+_-$ & $^{0.27}_{0.27}$ & &  & $43.21$ & $^+_-$ & $^{0.30}_{0.30}$ & &  & $29.37$ & $^+_-$ & $^{0.32}_{0.32}$ & &  \\
$[3500,4000]$ & $13.20$ & $^+_-$ & $^{0.20}_{0.20}$ & &  & $32.40$ & $^+_-$ & $^{0.28}_{0.28}$ & &  & $42.36$ & $^+_-$ & $^{0.31}_{0.31}$ & &  & $41.58$ & $^+_-$ & $^{0.34}_{0.34}$ & &  & $26.97$ & $^+_-$ & $^{0.35}_{0.35}$ & &  \\
$[3000,3500]$ & $11.56$ & $^+_-$ & $^{0.17}_{0.17}$ & &  & $30.52$ & $^+_-$ & $^{0.24}_{0.24}$ & &  & $40.04$ & $^+_-$ & $^{0.27}_{0.27}$ & &  & $39.10$ & $^+_-$ & $^{0.29}_{0.29}$ & &  & $24.56$ & $^+_-$ & $^{0.29}_{0.29}$ & &  \\
$[2500,3000]$ & $9.67$ & $^+_-$ & $^{0.14}_{0.14}$ & &  & $27.95$ & $^+_-$ & $^{0.20}_{0.20}$ & &  & $37.96$ & $^+_-$ & $^{0.23}_{0.23}$ & &  & $36.31$ & $^+_-$ & $^{0.25}_{0.25}$ & &  & $22.32$ & $^+_-$ & $^{0.25}_{0.24}$ & &  \\
$[2000,2500]$ & $8.05$ & $^+_-$ & $^{0.12}_{0.12}$ & &  & $24.99$ & $^+_-$ & $^{0.18}_{0.17}$ & &  & $34.40$ & $^+_-$ & $^{0.20}_{0.20}$ & &  & $32.92$ & $^+_-$ & $^{0.21}_{0.21}$ & &  & $19.53$ & $^+_-$ & $^{0.21}_{0.21}$ & &  \\
$[1500,2000]$ & $6.45$ & $^+_-$ & $^{0.10}_{0.10}$ & &  & $21.87$ & $^+_-$ & $^{0.15}_{0.15}$ & &  & $31.21$ & $^+_-$ & $^{0.18}_{0.18}$ & &  & $29.58$ & $^+_-$ & $^{0.19}_{0.19}$ & &  & $16.69$ & $^+_-$ & $^{0.18}_{0.18}$ & &  \\
$[1000,1500]$ & $5.24$ & $^+_-$ & $^{0.09}_{0.09}$ & &  & $19.01$ & $^+_-$ & $^{0.14}_{0.14}$ & &  & $28.08$ & $^+_-$ & $^{0.16}_{0.16}$ & &  & $26.46$ & $^+_-$ & $^{0.17}_{0.17}$ & &  & $14.25$ & $^+_-$ & $^{0.17}_{0.16}$ & &  \\
$[0,1000]$ & $4.03$ & $^+_-$ & $^{0.07}_{0.07}$ & &  & $16.32$ & $^+_-$ & $^{0.12}_{0.12}$ & &  & $25.00$ & $^+_-$ & $^{0.14}_{0.14}$ & &  & $23.22$ & $^+_-$ & $^{0.15}_{0.15}$ & &  & $12.17$ & $^+_-$ & $^{0.14}_{0.14}$ & &  \\
\bottomrule\end{tabular}

\end{table}

\begin{table}
  \centering
  \caption{%
    Efficiency $1/\efftruth$ of
    the truth matching requirement applied to the simulated data.
  }
  \label{tab:prod:effs:truth_matching}
  \begin{tabular}{lc}
  \toprule
  Mode                 & $1/\efftruth$ (\%) \\
  \midrule
  \DzToKpi             & $98.68 \pm 0.06$   \\
  \DpToKpipi           & $96.53 \pm 0.58$   \\
  \DspTophipi          & $98.53 \pm 0.13$   \\
  \DstToDzpi, \DzToKpi & $96.85 \pm 0.09$   \\
  \bottomrule
\end{tabular}

\end{table}

\begin{table}
  \caption{%
    Tracking efficiency correction factors \efftracking\ for \DzToKpi\ measured 
    in \PDzero in \pTy\ bins.
  }
  \label{tab:prod:effs:tracking:dztokpi}
  \centering
  \renewcommand{\arraystretch}{1.0}
\begin{tabular}{lccccc}
\toprule&\multicolumn{5}{c}{$\text{$y$}$}\\
$\text{$p_{\text{T}}$ [\text{MeV}/c]}$ & $[2,2.5]$ & $[2.5,3]$ & $[3,3.5]$ & $[3.5,4]$ & $[4,4.5]$ \\
\midrule
$[0,1000]$ & $110.26 \pm 0.03$ & $107.14 \pm 0.03$ & $104.04 \pm 0.03$ & $100.75 \pm 0.03$ & $99.40 \pm 0.03$ \\
$[1000,1500]$ & $109.14 \pm 0.05$ & $105.85 \pm 0.04$ & $102.95 \pm 0.03$ & $100.30 \pm 0.03$ & $99.03 \pm 0.03$ \\
$[1500,2000]$ & $108.04 \pm 0.06$ & $105.26 \pm 0.04$ & $102.58 \pm 0.03$ & $100.17 \pm 0.03$ & $99.00 \pm 0.04$ \\
$[2000,2500]$ & $107.21 \pm 0.06$ & $104.85 \pm 0.05$ & $102.24 \pm 0.03$ & $99.95 \pm 0.03$ & $99.16 \pm 0.03$ \\
$[2500,3000]$ & $106.94 \pm 0.06$ & $104.58 \pm 0.03$ & $102.02 \pm 0.03$ & $99.70 \pm 0.03$ & $99.52 \pm 0.03$ \\
$[3000,3500]$ & $106.48 \pm 0.07$ & $104.32 \pm 0.04$ & $101.76 \pm 0.04$ & $99.48 \pm 0.03$ & $99.93 \pm 0.03$ \\
$[3500,4000]$ & $106.47 \pm 0.05$ & $104.05 \pm 0.04$ & $101.44 \pm 0.03$ & $99.38 \pm 0.03$ & $100.30 \pm 0.04$ \\
$[4000,5000]$ & $106.12 \pm 0.04$ & $103.79 \pm 0.03$ & $101.14 \pm 0.03$ & $99.40 \pm 0.03$ & $100.78 \pm 0.03$ \\
$[5000,6000]$ & $105.82 \pm 0.04$ & $103.59 \pm 0.04$ & $100.87 \pm 0.04$ & $99.87 \pm 0.04$ & $101.40 \pm 0.04$ \\
$[6000,7000]$ & $105.35 \pm 0.04$ & $103.45 \pm 0.05$ & $100.69 \pm 0.06$ & $100.53 \pm 0.05$ & $102.00 \pm 0.04$ \\
$[7000,8000]$ & $104.99 \pm 0.04$ & $103.37 \pm 0.05$ & $100.84 \pm 0.05$ & $100.91 \pm 0.06$ & $102.26 \pm 0.05$ \\
$[8000,9000]$ & $104.53 \pm 0.06$ & $103.36 \pm 0.06$ & $101.12 \pm 0.07$ & $101.38 \pm 0.06$ & $102.61 \pm 0.07$ \\
$[9000,10000]$ & $104.37 \pm 0.02$ & $103.25 \pm 0.02$ & $101.47 \pm 0.03$ & $101.73 \pm 0.02$ & $102.83 \pm 0.03$ \\
$[10000,11000]$ & $104.15 \pm 0.02$ & $103.14 \pm 0.02$ & $101.73 \pm 0.02$ & $101.97 \pm 0.02$ & $103.00 \pm 0.03$ \\
$[11000,12000]$ & $103.98 \pm 0.02$ & $103.12 \pm 0.02$ & $102.01 \pm 0.03$ & $102.18 \pm 0.03$ & $103.13 \pm 0.04$ \\
$[12000,13000]$ & $103.81 \pm 0.02$ & $103.09 \pm 0.02$ & $102.18 \pm 0.03$ & $102.36 \pm 0.03$ & $103.30 \pm 0.04$ \\
$[13000,14000]$ & $103.65 \pm 0.03$ & $103.17 \pm 0.03$ & $102.37 \pm 0.03$ & $102.48 \pm 0.04$ & $103.45 \pm 0.05$ \\
$[14000,15000]$ & $103.55 \pm 0.03$ & $103.17 \pm 0.03$ & $102.54 \pm 0.04$ & $102.74 \pm 0.05$ & $103.52 \pm 0.08$ \\
\bottomrule\end{tabular}

\end{table}

\begin{table}
  \caption{%
    Tracking efficiency correction factors \efftracking\ for \DpToKpipi\ 
    measured in \PDplus in \pTy\ bins.
  }
  \label{tab:prod:effs:tracking:dptokpipi}
  \centering
  \renewcommand{\arraystretch}{1.0}
\begin{tabular}{lr@{\hskip+0.2em}c@{\hskip+0.2em}r@{\hskip+0.2em}c@{\hskip+0.2em}rr@{\hskip+0.2em}c@{\hskip+0.2em}r@{\hskip+0.2em}c@{\hskip+0.2em}rr@{\hskip+0.2em}c@{\hskip+0.2em}r@{\hskip+0.2em}c@{\hskip+0.2em}rr@{\hskip+0.2em}c@{\hskip+0.2em}r@{\hskip+0.2em}c@{\hskip+0.2em}rr@{\hskip+0.2em}c@{\hskip+0.2em}r@{\hskip+0.2em}c@{\hskip+0.2em}r}
\toprule&\multicolumn{25}{c}{$\text{$y$}$}\\
$\text{$p_{\text{T}}$ [\text{MeV}/c]}$ & \multicolumn{5}{c}{$[2,2.5]$} & \multicolumn{5}{c}{$[2.5,3]$} & \multicolumn{5}{c}{$[3,3.5]$} & \multicolumn{5}{c}{$[3.5,4]$} & \multicolumn{5}{c}{$[4,4.5]$} \\
\midrule$[14000,15000]$ & \multicolumn{5}{c}{$106.96 \pm 0.04$} & \multicolumn{5}{c}{$105.40 \pm 0.04$} & \multicolumn{5}{c}{$101.98 \pm 0.06$} & \multicolumn{5}{c}{$101.93 \pm 0.06$} & \multicolumn{5}{c}{$104.16 \pm 0.09$} \\
$[13000,14000]$ & \multicolumn{5}{c}{$107.23 \pm 0.03$} & \multicolumn{5}{c}{$105.40 \pm 0.04$} & \multicolumn{5}{c}{$101.87 \pm 0.05$} & \multicolumn{5}{c}{$101.70 \pm 0.07$} & \multicolumn{5}{c}{$103.77 \pm 0.08$} \\
$[12000,13000]$ & \multicolumn{5}{c}{$107.60 \pm 0.03$} & \multicolumn{5}{c}{$105.58 \pm 0.03$} & \multicolumn{5}{c}{$101.66 \pm 0.05$} & \multicolumn{5}{c}{$101.18 \pm 0.04$} & \multicolumn{5}{c}{$103.50 \pm 0.06$} \\
$[11000,12000]$ & \multicolumn{5}{c}{$107.96 \pm 0.02$} & \multicolumn{5}{c}{$105.71 \pm 0.03$} & \multicolumn{5}{c}{$101.57 \pm 0.04$} & \multicolumn{5}{c}{$100.82 \pm 0.03$} & \multicolumn{5}{c}{$103.21 \pm 0.05$} \\
$[10000,11000]$ & \multicolumn{5}{c}{$108.43 \pm 0.02$} & \multicolumn{5}{c}{$105.94 \pm 0.03$} & \multicolumn{5}{c}{$101.44 \pm 0.03$} & \multicolumn{5}{c}{$100.35 \pm 0.03$} & \multicolumn{5}{c}{$102.75 \pm 0.04$} \\
$[9000,10000]$ & \multicolumn{5}{c}{$109.02 \pm 0.09$} & \multicolumn{5}{c}{$106.26 \pm 0.08$} & \multicolumn{5}{c}{$101.20 \pm 0.10$} & \multicolumn{5}{c}{$100.18 \pm 0.08$} & \multicolumn{5}{c}{$102.56 \pm 0.09$} \\
$[8000,9000]$ & \multicolumn{5}{c}{$109.76 \pm 0.07$} & \multicolumn{5}{c}{$106.61 \pm 0.09$} & \multicolumn{5}{c}{$101.47 \pm 0.11$} & \multicolumn{5}{c}{$99.53 \pm 0.10$} & \multicolumn{5}{c}{$101.82 \pm 0.13$} \\
$[7000,8000]$ & \multicolumn{5}{c}{$110.42 \pm 0.07$} & \multicolumn{5}{c}{$106.83 \pm 0.09$} & \multicolumn{5}{c}{$101.54 \pm 0.09$} & \multicolumn{5}{c}{$99.13 \pm 0.08$} & \multicolumn{5}{c}{$100.93 \pm 0.10$} \\
$[6000,7000]$ & \multicolumn{5}{c}{$111.42 \pm 0.07$} & \multicolumn{5}{c}{$107.27 \pm 0.07$} & \multicolumn{5}{c}{$101.61 \pm 0.09$} & \multicolumn{5}{c}{$98.86 \pm 0.06$} & \multicolumn{5}{c}{$100.27 \pm 0.08$} \\
$[5000,6000]$ & \multicolumn{5}{c}{$112.02 \pm 0.07$} & \multicolumn{5}{c}{$107.78 \pm 0.06$} & \multicolumn{5}{c}{$102.02 \pm 0.06$} & \multicolumn{5}{c}{$98.79 \pm 0.05$} & \multicolumn{5}{c}{$99.45 \pm 0.06$} \\
$[4000,5000]$ & \multicolumn{5}{c}{$112.93 \pm 0.07$} & \multicolumn{5}{c}{$108.17 \pm 0.06$} & \multicolumn{5}{c}{$102.31 \pm 0.05$} & \multicolumn{5}{c}{$98.88 \pm 0.05$} & \multicolumn{5}{c}{$98.69 \pm 0.05$} \\
$[3500,4000]$ & \multicolumn{5}{c}{$113.22 \pm 0.10$} & \multicolumn{5}{c}{$108.21 \pm 0.08$} & \multicolumn{5}{c}{$102.41 \pm 0.07$} & \multicolumn{5}{c}{$99.15 \pm 0.05$} & \multicolumn{5}{c}{$98.28 \pm 0.06$} \\
$[3000,3500]$ & \multicolumn{5}{c}{$113.43 \pm 0.11$} & \multicolumn{5}{c}{$108.25 \pm 0.08$} & \multicolumn{5}{c}{$102.43 \pm 0.06$} & \multicolumn{5}{c}{$99.18 \pm 0.06$} & \multicolumn{5}{c}{$97.96 \pm 0.06$} \\
$[2500,3000]$ & \multicolumn{5}{c}{$113.03 \pm 0.12$} & \multicolumn{5}{c}{$107.91 \pm 0.08$} & \multicolumn{5}{c}{$102.30 \pm 0.06$} & \multicolumn{5}{c}{$99.30 \pm 0.05$} & \multicolumn{5}{c}{$97.77 \pm 0.06$} \\
$[2000,2500]$ & \multicolumn{5}{c}{$112.79 \pm 0.13$} & \multicolumn{5}{c}{$107.53 \pm 0.07$} & \multicolumn{5}{c}{$102.13 \pm 0.06$} & \multicolumn{5}{c}{$99.17 \pm 0.06$} & \multicolumn{5}{c}{$97.79 \pm 0.06$} \\
$[1500,2000]$ & \multicolumn{5}{c}{$112.48 \pm 0.14$} & \multicolumn{5}{c}{$106.68 \pm 0.08$} & \multicolumn{5}{c}{$101.50 \pm 0.06$} & \multicolumn{5}{c}{$98.93 \pm 0.05$} & \multicolumn{5}{c}{$97.78 \pm 0.07$} \\
$[1000,1500]$ & \multicolumn{5}{c}{$112.18 \pm 0.16$} & \multicolumn{5}{c}{$105.73 \pm 0.08$} & \multicolumn{5}{c}{$100.62 \pm 0.06$} & \multicolumn{5}{c}{$98.51 \pm 0.06$} & \multicolumn{5}{c}{$97.82 \pm 0.07$} \\
$[0,1000]$ & \multicolumn{5}{c}{$112.17 \pm 0.15$} & \multicolumn{5}{c}{$104.73 \pm 0.08$} & \multicolumn{5}{c}{$99.51 \pm 0.06$} & \multicolumn{5}{c}{$98.03 \pm 0.06$} & \multicolumn{5}{c}{$97.96 \pm 0.07$} \\
\bottomrule\end{tabular}

\end{table}

\begin{table}
  \caption{%
    Selection efficiencies $\effselection/\efflzero$ for \DzToKpi, excluding 
    \ac{PID} requirements, measured in \PDzero \pTy\ bins.
  }
  \label{tab:prod:effs:sel:dztokpi}
  \centering
  \renewcommand{\arraystretch}{1.3}
\begin{tabular}{lccccc}
\toprule&\multicolumn{5}{c}{$\text{$y$}$}\\
$\text{$p_{\text{T}}$ [\text{MeV}/c]}$ & $[2,2.5]$ & $[2.5,3]$ & $[3,3.5]$ & $[3.5,4]$ & $[4,4.5]$ \\
\midrule
$[0,1000]$ & $3.99^{+0.16}_{-0.15}$ & $7.53^{+0.13}_{-0.13}$ & $8.21^{+0.13}_{-0.13}$ & $6.82^{+0.14}_{-0.14}$ & $3.66^{+0.17}_{-0.16}$ \\
$[1000,1500]$ & $4.82^{+0.18}_{-0.17}$ & $8.92^{+0.16}_{-0.16}$ & $9.33^{+0.15}_{-0.15}$ & $7.48^{+0.15}_{-0.15}$ & $4.19^{+0.16}_{-0.16}$ \\
$[1500,2000]$ & $6.00^{+0.20}_{-0.19}$ & $10.51^{+0.17}_{-0.17}$ & $11.20^{+0.17}_{-0.17}$ & $8.61^{+0.16}_{-0.16}$ & $4.75^{+0.17}_{-0.16}$ \\
$[2000,2500]$ & $7.63^{+0.23}_{-0.22}$ & $12.91^{+0.20}_{-0.20}$ & $13.22^{+0.19}_{-0.19}$ & $10.80^{+0.20}_{-0.19}$ & $6.35^{+0.20}_{-0.20}$ \\
$[2500,3000]$ & $9.89^{+0.27}_{-0.26}$ & $15.94^{+0.24}_{-0.24}$ & $16.82^{+0.24}_{-0.23}$ & $13.26^{+0.24}_{-0.23}$ & $7.09^{+0.24}_{-0.23}$ \\
$[3000,3500]$ & $11.60^{+0.31}_{-0.30}$ & $18.96^{+0.28}_{-0.28}$ & $19.41^{+0.28}_{-0.28}$ & $15.81^{+0.29}_{-0.28}$ & $8.28^{+0.28}_{-0.27}$ \\
$[3500,4000]$ & $13.98^{+0.37}_{-0.36}$ & $22.09^{+0.34}_{-0.33}$ & $21.65^{+0.34}_{-0.33}$ & $18.12^{+0.35}_{-0.34}$ & $7.90^{+0.32}_{-0.31}$ \\
$[4000,5000]$ & $16.42^{+0.33}_{-0.32}$ & $24.63^{+0.30}_{-0.30}$ & $24.52^{+0.31}_{-0.30}$ & $20.76^{+0.32}_{-0.32}$ & $5.37^{+0.23}_{-0.22}$ \\
$[5000,6000]$ & $18.76^{+0.44}_{-0.43}$ & $26.55^{+0.41}_{-0.41}$ & $28.94^{+0.44}_{-0.43}$ & $20.47^{+0.43}_{-0.43}$ & $2.83^{+0.24}_{-0.22}$ \\
$[6000,7000]$ & $22.24^{+0.61}_{-0.60}$ & $29.32^{+0.56}_{-0.55}$ & $30.84^{+0.59}_{-0.58}$ & $17.26^{+0.55}_{-0.53}$ & $0.54^{+0.16}_{-0.12}$ \\
$[7000,8000]$ & $23.69^{+0.78}_{-0.76}$ & $29.99^{+0.73}_{-0.72}$ & $30.35^{+0.77}_{-0.76}$ & $12.80^{+0.65}_{-0.62}$ & $0.00^{+0.08}_{-0.00}$ \\
$[8000,9000]$ & $20.59^{+0.89}_{-0.86}$ & $30.05^{+0.89}_{-0.88}$ & $27.36^{+0.93}_{-0.91}$ & $5.54^{+0.58}_{-0.53}$ & $0.00^{+0.12}_{-0.00}$ \\
$[9000,10000]$ & $23.94^{+0.28}_{-0.28}$ & $32.02^{+0.28}_{-0.28}$ & $26.22^{+0.29}_{-0.29}$ & $2.95^{+0.13}_{-0.13}$ & $0.00^{+0.01}_{-0.00}$ \\
$[10000,11000]$ & $23.84^{+0.28}_{-0.28}$ & $32.00^{+0.29}_{-0.29}$ & $22.44^{+0.28}_{-0.28}$ & $1.19^{+0.09}_{-0.08}$ & $0.00^{+0.02}_{-0.00}$ \\
$[11000,12000]$ & $25.63^{+0.35}_{-0.35}$ & $32.18^{+0.36}_{-0.36}$ & $17.53^{+0.32}_{-0.32}$ & $0.16^{+0.04}_{-0.04}$ & $0.00^{+0.02}_{-0.00}$ \\
$[12000,13000]$ & $24.78^{+0.42}_{-0.41}$ & $31.55^{+0.43}_{-0.43}$ & $12.67^{+0.35}_{-0.34}$ & $0.00^{+0.02}_{-0.00}$ & $0.00^{+0.04}_{-0.00}$ \\
$[13000,14000]$ & $24.90^{+0.50}_{-0.49}$ & $30.32^{+0.51}_{-0.51}$ & $9.03^{+0.36}_{-0.35}$ & $0.00^{+0.03}_{-0.00}$ & $0.00^{+0.06}_{-0.00}$ \\
$[14000,15000]$ & $25.72^{+0.60}_{-0.59}$ & $29.53^{+0.60}_{-0.59}$ & $6.23^{+0.37}_{-0.35}$ & $0.00^{+0.04}_{-0.00}$ & $0.00^{+0.09}_{-0.00}$ \\
\bottomrule\end{tabular}

\end{table}

\begin{table}
  \caption{%
    Selection efficiencies $\effselection/\efflzero$ for \DpToKpipi, excluding 
    \ac{PID} requirements, measured in \PDplus \pTy\ bins.
  }
  \label{tab:prod:effs:sel:dptokpipi}
  \centering
  \renewcommand{\arraystretch}{1.3}
\begin{tabular}{lr@{\hskip+0.2em}c@{\hskip+0.2em}r@{\hskip+0.2em}c@{\hskip+0.2em}rr@{\hskip+0.2em}c@{\hskip+0.2em}r@{\hskip+0.2em}c@{\hskip+0.2em}rr@{\hskip+0.2em}c@{\hskip+0.2em}r@{\hskip+0.2em}c@{\hskip+0.2em}rr@{\hskip+0.2em}c@{\hskip+0.2em}r@{\hskip+0.2em}c@{\hskip+0.2em}rr@{\hskip+0.2em}c@{\hskip+0.2em}r@{\hskip+0.2em}c@{\hskip+0.2em}r}
\toprule&\multicolumn{25}{c}{$\text{$y$}$}\\
$\text{$p_{\text{T}}$ [\text{MeV}/c]}$ & \multicolumn{5}{c}{$[2,2.5]$} & \multicolumn{5}{c}{$[2.5,3]$} & \multicolumn{5}{c}{$[3,3.5]$} & \multicolumn{5}{c}{$[3.5,4]$} & \multicolumn{5}{c}{$[4,4.5]$} \\
\midrule
$[0,1000]$ & $0.03$ & $^+_-$ & $^{0.06}_{0.02}$ & &  & $0.25$ & $^+_-$ & $^{0.04}_{0.04}$ & &  & $0.41$ & $^+_-$ & $^{0.04}_{0.04}$ & &  & $0.53$ & $^+_-$ & $^{0.06}_{0.05}$ & &  & $0.31$ & $^+_-$ & $^{0.08}_{0.06}$ & &  \\
$[1000,1500]$ & $0.87$ & $^+_-$ & $^{0.18}_{0.15}$ & &  & $2.72$ & $^+_-$ & $^{0.14}_{0.13}$ & &  & $3.93$ & $^+_-$ & $^{0.14}_{0.13}$ & &  & $3.96$ & $^+_-$ & $^{0.15}_{0.15}$ & &  & $2.91$ & $^+_-$ & $^{0.22}_{0.20}$ & &  \\
$[1500,2000]$ & $3.17$ & $^+_-$ & $^{0.29}_{0.27}$ & &  & $9.85$ & $^+_-$ & $^{0.24}_{0.23}$ & &  & $11.77$ & $^+_-$ & $^{0.22}_{0.22}$ & &  & $11.34$ & $^+_-$ & $^{0.24}_{0.24}$ & &  & $6.76$ & $^+_-$ & $^{0.30}_{0.29}$ & &  \\
$[2000,2500]$ & $9.19$ & $^+_-$ & $^{0.45}_{0.43}$ & &  & $18.92$ & $^+_-$ & $^{0.32}_{0.31}$ & &  & $21.18$ & $^+_-$ & $^{0.29}_{0.29}$ & &  & $19.92$ & $^+_-$ & $^{0.32}_{0.31}$ & &  & $13.35$ & $^+_-$ & $^{0.41}_{0.40}$ & &  \\
$[2500,3000]$ & $13.76$ & $^+_-$ & $^{0.53}_{0.51}$ & &  & $27.24$ & $^+_-$ & $^{0.38}_{0.38}$ & &  & $28.96$ & $^+_-$ & $^{0.35}_{0.35}$ & &  & $27.10$ & $^+_-$ & $^{0.38}_{0.38}$ & &  & $18.22$ & $^+_-$ & $^{0.49}_{0.48}$ & &  \\
$[3000,3500]$ & $18.38$ & $^+_-$ & $^{0.61}_{0.59}$ & &  & $30.83$ & $^+_-$ & $^{0.43}_{0.43}$ & &  & $32.71$ & $^+_-$ & $^{0.40}_{0.40}$ & &  & $30.52$ & $^+_-$ & $^{0.43}_{0.43}$ & &  & $21.20$ & $^+_-$ & $^{0.56}_{0.55}$ & &  \\
$[3500,4000]$ & $23.76$ & $^+_-$ & $^{0.70}_{0.69}$ & &  & $35.77$ & $^+_-$ & $^{0.50}_{0.50}$ & &  & $37.13$ & $^+_-$ & $^{0.47}_{0.46}$ & &  & $34.28$ & $^+_-$ & $^{0.51}_{0.50}$ & &  & $22.20$ & $^+_-$ & $^{0.63}_{0.62}$ & &  \\
$[4000,5000]$ & $27.50$ & $^+_-$ & $^{0.59}_{0.58}$ & &  & $39.06$ & $^+_-$ & $^{0.42}_{0.42}$ & &  & $40.65$ & $^+_-$ & $^{0.41}_{0.41}$ & &  & $37.21$ & $^+_-$ & $^{0.45}_{0.45}$ & &  & $21.44$ & $^+_-$ & $^{0.53}_{0.52}$ & &  \\
$[5000,6000]$ & $31.24$ & $^+_-$ & $^{0.74}_{0.73}$ & &  & $43.16$ & $^+_-$ & $^{0.55}_{0.55}$ & &  & $43.89$ & $^+_-$ & $^{0.55}_{0.55}$ & &  & $40.44$ & $^+_-$ & $^{0.60}_{0.60}$ & &  & $15.65$ & $^+_-$ & $^{0.63}_{0.61}$ & &  \\
$[6000,7000]$ & $34.11$ & $^+_-$ & $^{0.92}_{0.91}$ & &  & $47.20$ & $^+_-$ & $^{0.72}_{0.72}$ & &  & $47.79$ & $^+_-$ & $^{0.73}_{0.73}$ & &  & $37.78$ & $^+_-$ & $^{0.81}_{0.80}$ & &  & $9.86$ & $^+_-$ & $^{0.73}_{0.68}$ & &  \\
$[7000,8000]$ & $36.32$ & $^+_-$ & $^{1.13}_{1.11}$ & &  & $48.82$ & $^+_-$ & $^{0.92}_{0.92}$ & &  & $48.89$ & $^+_-$ & $^{0.94}_{0.93}$ & &  & $34.10$ & $^+_-$ & $^{1.00}_{0.99}$ & &  & $3.45$ & $^+_-$ & $^{0.60}_{0.52}$ & &  \\
$[8000,9000]$ & $37.10$ & $^+_-$ & $^{1.42}_{1.40}$ & &  & $48.45$ & $^+_-$ & $^{1.14}_{1.14}$ & &  & $49.11$ & $^+_-$ & $^{1.21}_{1.21}$ & &  & $28.72$ & $^+_-$ & $^{1.27}_{1.24}$ & &  & $1.22$ & $^+_-$ & $^{0.56}_{0.39}$ & &  \\
$[9000,10000]$ & $36.17$ & $^+_-$ & $^{1.37}_{1.35}$ & &  & $45.78$ & $^+_-$ & $^{1.11}_{1.11}$ & &  & $43.70$ & $^+_-$ & $^{1.07}_{1.07}$ & &  & $16.48$ & $^+_-$ & $^{0.86}_{0.83}$ & &  & $0.09$ & $^+_-$ & $^{0.16}_{0.07}$ & &  \\
$[10000,11000]$ & $38.90$ & $^+_-$ & $^{0.37}_{0.37}$ & &  & $51.04$ & $^+_-$ & $^{0.32}_{0.32}$ & &  & $46.46$ & $^+_-$ & $^{0.35}_{0.34}$ & &  & $13.76$ & $^+_-$ & $^{0.29}_{0.29}$ & &  & $0.00$ & $^+_-$ & $^{0.02}_{0.00}$ & &  \\
$[11000,12000]$ & $40.84$ & $^+_-$ & $^{0.45}_{0.44}$ & &  & $51.41$ & $^+_-$ & $^{0.39}_{0.39}$ & &  & $42.87$ & $^+_-$ & $^{0.42}_{0.42}$ & &  & $7.82$ & $^+_-$ & $^{0.27}_{0.26}$ & &  & $0.00$ & $^+_-$ & $^{0.03}_{0.00}$ & &  \\
$[12000,13000]$ & $39.70$ & $^+_-$ & $^{0.53}_{0.52}$ & &  & $50.90$ & $^+_-$ & $^{0.47}_{0.47}$ & &  & $38.34$ & $^+_-$ & $^{0.50}_{0.49}$ & &  & $4.96$ & $^+_-$ & $^{0.27}_{0.26}$ & &  & $0.00$ & $^+_-$ & $^{0.04}_{0.00}$ & &  \\
$[13000,14000]$ & $39.10$ & $^+_-$ & $^{0.62}_{0.61}$ & &  & $49.92$ & $^+_-$ & $^{0.57}_{0.57}$ & &  & $31.79$ & $^+_-$ & $^{0.57}_{0.57}$ & &  & $1.79$ & $^+_-$ & $^{0.21}_{0.19}$ & &  & $0.00$ & $^+_-$ & $^{0.07}_{0.00}$ & &  \\
$[14000,15000]$ & $39.81$ & $^+_-$ & $^{0.73}_{0.73}$ & &  & $49.61$ & $^+_-$ & $^{0.67}_{0.67}$ & &  & $26.61$ & $^+_-$ & $^{0.66}_{0.65}$ & &  & $1.40$ & $^+_-$ & $^{0.23}_{0.20}$ & &  & $0.00$ & $^+_-$ & $^{0.10}_{0.00}$ & &  \\
\bottomrule\end{tabular}

\end{table}

\begin{figure}
  \begin{subfigure}[b]{0.5\textwidth}
    \centering
    \includegraphics[width=\textwidth]{production/efficiencies/PID_binning_kaon_p}
    \caption{Track \ptot}
    \label{fig:prod:effs:pid:binning:kaon:p}
  \end{subfigure}
  \begin{subfigure}[b]{0.5\textwidth}
    \centering
    \includegraphics[width=\textwidth]{production/efficiencies/PID_binning_kaon_eta}
    \caption{Track \Eta}
    \label{fig:prod:effs:pid:binning:kaon:eta}
  \end{subfigure}
  \begin{subfigure}[b]{0.5\textwidth}
    \centering
    \includegraphics[width=\textwidth]{production/efficiencies/PID_binning_kaon_nspd}
    \caption{\nspd}
    \label{fig:prod:effs:pid:binning:kaon:nspd}
  \end{subfigure}
  \caption{%
    Binning scheme (dark red) used for computing kaon \ac{PID} efficiencies.
    A finer binning scheme (blue) is given to illustrate the varying efficiency 
    within the nominal binning scheme.
    The calibration sample distribution (red) and reference sample distribution 
    (black) are also shown.
    The distributions are kaon momentum 
    (\subref*{fig:prod:effs:pid:binning:kaon:p}) and pseudorapidity 
    (\subref*{fig:prod:effs:pid:binning:kaon:eta}) and the event multiplicity 
    (\subref*{fig:prod:effs:pid:binning:kaon:nspd}).
  }
  \label{fig:prod:pid:binning:kaon}
\end{figure}

\begin{figure}
  \begin{subfigure}[b]{0.5\textwidth}
    \centering
    \includegraphics[width=\textwidth]{production/efficiencies/PID_binning_pion_p}
    \caption{Track \ptot}
    \label{fig:prod:effs:pid:binning:pion:p}
  \end{subfigure}
  \begin{subfigure}[b]{0.5\textwidth}
    \centering
    \includegraphics[width=\textwidth]{production/efficiencies/PID_binning_pion_eta}
    \caption{Track \Eta}
    \label{fig:prod:effs:pid:binning:pion:eta}
  \end{subfigure}
  \begin{subfigure}[b]{0.5\textwidth}
    \centering
    \includegraphics[width=\textwidth]{production/efficiencies/PID_binning_pion_nspd}
    \caption{\nspd}
    \label{fig:prod:effs:pid:binning:pion:nspd}
  \end{subfigure}
  \caption{%
    Binning scheme (dark red) used for computing pion \ac{PID} efficiencies.
    A finer binning scheme (blue) is given to illustrate the varying efficiency 
    within the nominal binning scheme.
    The calibration sample distribution (red) and reference sample distribution 
    (black) are also shown.
    The distributions are pion momentum 
    (\subref*{fig:prod:effs:pid:binning:pion:p}) and pseudorapidity 
    (\subref*{fig:prod:effs:pid:binning:pion:eta}) and the event multiplicity 
    (\subref*{fig:prod:effs:pid:binning:pion:nspd}).
  }
  \label{fig:prod:pid:binning:pion}
\end{figure}

\begin{table}
  \caption{%
  Particle identification selection efficiencies \effpid\ for \DzToKpi, 
  measured in \PDzero \pTy\ bins.
  }
  \label{tab:prod:effs:pid:dztokpi}
  \centering
  \renewcommand{\arraystretch}{1.0}
\begin{tabular}{lccccc}
\toprule&\multicolumn{5}{c}{$\text{$y$}$}\\
$\text{$p_{\text{T}}$ [\text{MeV}/c]}$ & $[2,2.5]$ & $[2.5,3]$ & $[3,3.5]$ & $[3.5,4]$ & $[4,4.5]$ \\
\midrule
$[0,1000]$ & $82.79 \pm 0.18$ & $87.38 \pm 0.16$ & $92.27 \pm 0.10$ & $89.87 \pm 0.14$ & $80.50 \pm 0.23$ \\
$[1000,1500]$ & $82.94 \pm 0.18$ & $88.53 \pm 0.13$ & $92.36 \pm 0.10$ & $89.08 \pm 0.12$ & $75.97 \pm 0.19$ \\
$[1500,2000]$ & $82.67 \pm 0.18$ & $89.14 \pm 0.10$ & $92.69 \pm 0.08$ & $88.54 \pm 0.10$ & $70.86 \pm 0.19$ \\
$[2000,2500]$ & $82.55 \pm 0.17$ & $89.75 \pm 0.09$ & $92.89 \pm 0.06$ & $86.89 \pm 0.09$ & $65.13 \pm 0.21$ \\
$[2500,3000]$ & $82.82 \pm 0.16$ & $90.46 \pm 0.08$ & $93.13 \pm 0.05$ & $84.08 \pm 0.09$ & $62.42 \pm 0.23$ \\
$[3000,3500]$ & $83.27 \pm 0.15$ & $91.18 \pm 0.07$ & $93.15 \pm 0.05$ & $80.62 \pm 0.10$ & $59.19 \pm 0.25$ \\
$[3500,4000]$ & $83.90 \pm 0.14$ & $91.80 \pm 0.06$ & $92.95 \pm 0.05$ & $77.58 \pm 0.11$ & $56.20 \pm 0.25$ \\
$[4000,5000]$ & $85.25 \pm 0.13$ & $92.18 \pm 0.06$ & $91.55 \pm 0.05$ & $72.98 \pm 0.14$ & $50.63 \pm 0.25$ \\
$[5000,6000]$ & $86.12 \pm 0.13$ & $91.67 \pm 0.06$ & $88.16 \pm 0.08$ & $68.45 \pm 0.16$ & $43.58 \pm 0.25$ \\
$[6000,7000]$ & $85.85 \pm 0.14$ & $90.19 \pm 0.07$ & $83.81 \pm 0.11$ & $64.67 \pm 0.19$ & $43.96 \pm 0.27$ \\
$[7000,8000]$ & $83.73 \pm 0.16$ & $87.34 \pm 0.10$ & $80.01 \pm 0.15$ & $61.07 \pm 0.22$ & - \\
$[8000,9000]$ & $81.68 \pm 0.19$ & $83.08 \pm 0.15$ & $76.91 \pm 0.19$ & $57.29 \pm 0.26$ & - \\
$[9000,10000]$ & $78.78 \pm 0.22$ & $77.73 \pm 0.24$ & $74.72 \pm 0.22$ & $53.34 \pm 0.28$ & - \\
$[10000,11000]$ & $73.28 \pm 0.31$ & $74.36 \pm 0.29$ & $71.52 \pm 0.25$ & $56.66 \pm 0.45$ & - \\
$[11000,12000]$ & $72.42 \pm 0.37$ & $69.90 \pm 0.48$ & $68.96 \pm 0.30$ & $60.38 \pm 0.65$ & - \\
$[12000,13000]$ & $69.61 \pm 0.44$ & $64.19 \pm 0.66$ & $66.18 \pm 0.33$ & - & - \\
$[13000,14000]$ & $65.73 \pm 0.50$ & $58.92 \pm 0.96$ & $64.56 \pm 0.36$ & - & - \\
$[14000,15000]$ & $60.94 \pm 0.80$ & $57.57 \pm 1.20$ & $60.32 \pm 0.41$ & - & - \\
\bottomrule\end{tabular}

\end{table}

\begin{table}
  \caption{%
  Particle identification selection efficiencies \effpid\ for \DpToKpipi, 
  measured in \PDzero \pTy\ bins.
  }
  \label{tab:prod:effs:pid:dptokpipi}
  \centering
  \renewcommand{\arraystretch}{1.0}
\begin{tabular}{l|ccccc}
\toprule&\multicolumn{5}{c}{$\text{$y$}$}\\
$\text{$p_{\text{T}}$ [\text{MeV}/c]}$ & $[2,2.5[$ & $[2.5,3[$ & $[3,3.5[$ & $[3.5,4[$ & $[4,4.5[$ \\
\midrule$[14000,15000[$ & $64.81 \pm 0.41$ & $72.99 \pm 0.39$ & $72.70 \pm 0.26$ & $50.29 \pm 0.49$ & - \\
$[13000,14000[$ & $66.76 \pm 0.32$ & $74.77 \pm 0.22$ & $73.82 \pm 0.24$ & $50.41 \pm 0.46$ & - \\
$[12000,13000[$ & $68.33 \pm 0.27$ & $75.09 \pm 0.30$ & $75.25 \pm 0.21$ & $58.82 \pm 0.30$ & - \\
$[11000,12000[$ & $69.75 \pm 0.24$ & $77.95 \pm 0.25$ & $77.79 \pm 0.18$ & $58.54 \pm 0.26$ & - \\
$[10000,11000[$ & $71.57 \pm 0.22$ & $80.85 \pm 0.15$ & $79.59 \pm 0.16$ & $61.67 \pm 0.24$ & - \\
$[9000,10000[$ & $73.13 \pm 0.21$ & $83.65 \pm 0.11$ & $81.96 \pm 0.14$ & $65.24 \pm 0.21$ & $45.24 \pm 0.54$ \\
$[8000,9000[$ & $73.55 \pm 0.20$ & $85.55 \pm 0.09$ & $83.80 \pm 0.11$ & $67.11 \pm 0.19$ & $42.68 \pm 0.35$ \\
$[7000,8000[$ & $74.17 \pm 0.20$ & $86.97 \pm 0.08$ & $86.61 \pm 0.09$ & $69.86 \pm 0.17$ & $46.69 \pm 0.28$ \\
$[6000,7000[$ & $73.77 \pm 0.20$ & $87.10 \pm 0.09$ & $88.81 \pm 0.08$ & $72.86 \pm 0.15$ & $50.45 \pm 0.27$ \\
$[5000,6000[$ & $72.97 \pm 0.22$ & $86.40 \pm 0.10$ & $90.27 \pm 0.07$ & $76.76 \pm 0.13$ & $54.40 \pm 0.26$ \\
$[4000,5000[$ & $71.94 \pm 0.23$ & $85.04 \pm 0.11$ & $90.72 \pm 0.08$ & $79.72 \pm 0.12$ & $59.99 \pm 0.23$ \\
$[3500,4000[$ & $71.48 \pm 0.24$ & $83.44 \pm 0.13$ & $89.96 \pm 0.10$ & $82.02 \pm 0.12$ & $63.79 \pm 0.22$ \\
$[3000,3500[$ & $70.99 \pm 0.25$ & $82.05 \pm 0.15$ & $88.97 \pm 0.12$ & $83.04 \pm 0.13$ & $66.50 \pm 0.21$ \\
$[2500,3000[$ & $71.27 \pm 0.25$ & $80.84 \pm 0.17$ & $87.57 \pm 0.15$ & $83.36 \pm 0.16$ & $69.21 \pm 0.21$ \\
$[2000,2500[$ & $71.38 \pm 0.24$ & $80.27 \pm 0.19$ & $85.90 \pm 0.21$ & $83.11 \pm 0.21$ & $71.76 \pm 0.24$ \\
$[1500,2000[$ & $73.41 \pm 0.23$ & $80.45 \pm 0.22$ & $84.61 \pm 0.32$ & $82.24 \pm 0.29$ & $73.73 \pm 0.28$ \\
$[1000,1500[$ & $75.24 \pm 0.22$ & $81.55 \pm 0.24$ & $83.74 \pm 0.39$ & $81.50 \pm 0.38$ & $74.91 \pm 0.34$ \\
$[0,1000[$ & $79.65 \pm 0.24$ & $82.98 \pm 0.25$ & $85.74 \pm 0.32$ & $84.03 \pm 0.44$ & $74.82 \pm 0.47$ \\
\bottomrule\end{tabular}

\end{table}

\begin{table}
  \caption{%
    Signal window efficiencies \effsigwin\ for \DzToKpi, measured in \PDzero \pTy\ bins.
  }
  \label{tab:prod:effs:sigwin:dztokpi}
  \centering
  \renewcommand{\arraystretch}{1.0}
\begin{tabular}{lccccc}
\toprule&\multicolumn{5}{c}{$\text{$y$}$}\\
$\text{$p_{\text{T}}$ [\text{MeV}/c]}$ & $[2,2.5[$ & $[2.5,3[$ & $[3,3.5[$ & $[3.5,4[$ & $[4,4.5[$ \\
\midrule$[14000,15000[$ & $95.64 \pm 0.41$ & $95.22 \pm 0.42$ & $97.60 \pm 0.33$ & $97.79 \pm 0.04$ & $97.79 \pm 0.04$ \\
$[13000,14000[$ & $95.98 \pm 0.32$ & $96.24 \pm 0.26$ & $93.58 \pm 0.25$ & $97.79 \pm 0.04$ & $97.79 \pm 0.04$ \\
$[12000,13000[$ & $96.62 \pm 0.21$ & $96.09 \pm 0.23$ & $93.90 \pm 0.28$ & $97.79 \pm 0.04$ & $97.79 \pm 0.04$ \\
$[11000,12000[$ & $96.58 \pm 0.18$ & $96.53 \pm 0.16$ & $96.01 \pm 0.31$ & $96.30 \pm 0.10$ & $97.79 \pm 0.04$ \\
$[10000,11000[$ & $96.78 \pm 0.14$ & $96.73 \pm 0.12$ & $95.68 \pm 0.24$ & $92.90 \pm 1.08$ & $97.79 \pm 0.04$ \\
$[9000,10000[$ & $96.92 \pm 0.10$ & $96.78 \pm 0.09$ & $96.19 \pm 0.15$ & $93.76 \pm 0.16$ & $97.79 \pm 0.04$ \\
$[8000,9000[$ & $97.35 \pm 0.06$ & $96.84 \pm 0.07$ & $96.16 \pm 0.11$ & $94.62 \pm 0.45$ & $97.79 \pm 0.04$ \\
$[7000,8000[$ & $97.41 \pm 0.05$ & $97.09 \pm 0.06$ & $96.32 \pm 0.08$ & $95.28 \pm 0.23$ & $97.79 \pm 0.04$ \\
$[6000,7000[$ & $97.44 \pm 0.05$ & $97.20 \pm 0.05$ & $96.47 \pm 0.07$ & $95.53 \pm 0.14$ & $93.76 \pm 0.34$ \\
$[5000,6000[$ & $97.58 \pm 0.04$ & $97.41 \pm 0.04$ & $96.72 \pm 0.05$ & $95.72 \pm 0.09$ & $93.33 \pm 0.20$ \\
$[4000,5000[$ & $97.68 \pm 0.04$ & $97.53 \pm 0.04$ & $96.89 \pm 0.05$ & $95.66 \pm 0.07$ & $93.61 \pm 0.07$ \\
$[3500,4000[$ & $97.77 \pm 0.04$ & $97.60 \pm 0.04$ & $97.00 \pm 0.05$ & $95.82 \pm 0.07$ & $93.63 \pm 0.09$ \\
$[3000,3500[$ & $97.81 \pm 0.04$ & $97.68 \pm 0.04$ & $97.09 \pm 0.04$ & $95.93 \pm 0.07$ & $93.67 \pm 0.08$ \\
$[2500,3000[$ & $97.84 \pm 0.04$ & $97.72 \pm 0.04$ & $97.17 \pm 0.04$ & $95.93 \pm 0.06$ & $93.74 \pm 0.13$ \\
$[2000,2500[$ & $97.91 \pm 0.04$ & $97.76 \pm 0.04$ & $97.28 \pm 0.04$ & $96.11 \pm 0.06$ & $93.61 \pm 0.17$ \\
$[1500,2000[$ & $97.97 \pm 0.04$ & $97.82 \pm 0.04$ & $97.36 \pm 0.04$ & $96.26 \pm 0.06$ & $93.70 \pm 1.90$ \\
$[1000,1500[$ & $97.99 \pm 0.04$ & $97.86 \pm 0.04$ & $97.46 \pm 0.04$ & $96.60 \pm 0.06$ & $95.28 \pm 0.15$ \\
$[0,1000[$ & $98.02 \pm 0.04$ & $97.92 \pm 0.04$ & $97.56 \pm 0.04$ & $96.79 \pm 0.06$ & $95.38 \pm 0.17$ \\
\bottomrule\end{tabular}

\end{table}

\begin{table}
  \caption{%
    Signal window efficiencies \effsigwin\ for \DpToKpipi, measured in \PDplus \pTy\ bins.
  }
  \label{tab:prod:effs:sigwin:dptokpipi}
  \centering
  \renewcommand{\arraystretch}{1.0}
\begin{tabular}{lccccc}
\toprule&\multicolumn{5}{c}{$\text{$y$}$}\\
$\text{$p_{\text{T}}$ [\text{MeV}/c]}$ & $[2,2.5[$ & $[2.5,3[$ & $[3,3.5[$ & $[3.5,4[$ & $[4,4.5[$ \\
\midrule$[14000,15000[$ & $96.74 \pm 0.15$ & $96.14 \pm 0.22$ & $96.17 \pm 0.35$ & $96.58 \pm 1.32$ & $97.30 \pm 0.04$ \\
$[13000,14000[$ & $96.96 \pm 0.10$ & $96.69 \pm 0.12$ & $96.11 \pm 0.26$ & $97.09 \pm 0.65$ & $97.30 \pm 0.04$ \\
$[12000,13000[$ & $96.97 \pm 0.09$ & $96.63 \pm 0.11$ & $95.89 \pm 0.23$ & $96.34 \pm 0.72$ & $97.30 \pm 0.04$ \\
$[11000,12000[$ & $97.05 \pm 0.06$ & $96.90 \pm 0.07$ & $96.11 \pm 0.15$ & $96.01 \pm 0.57$ & $97.30 \pm 0.04$ \\
$[10000,11000[$ & $97.21 \pm 0.05$ & $96.97 \pm 0.06$ & $96.31 \pm 0.11$ & $95.67 \pm 0.36$ & $97.30 \pm 0.04$ \\
$[9000,10000[$ & $97.18 \pm 0.05$ & $97.01 \pm 0.05$ & $96.35 \pm 0.09$ & $95.78 \pm 0.24$ & $92.94 \pm 2.08$ \\
$[8000,9000[$ & $97.22 \pm 0.04$ & $97.10 \pm 0.04$ & $96.43 \pm 0.07$ & $95.58 \pm 0.17$ & $96.33 \pm 1.07$ \\
$[7000,8000[$ & $97.25 \pm 0.04$ & $97.19 \pm 0.04$ & $96.65 \pm 0.05$ & $95.31 \pm 0.13$ & $94.03 \pm 0.38$ \\
$[6000,7000[$ & $97.29 \pm 0.04$ & $97.22 \pm 0.04$ & $96.77 \pm 0.05$ & $95.43 \pm 0.09$ & $93.75 \pm 0.12$ \\
$[5000,6000[$ & $97.30 \pm 0.04$ & $97.25 \pm 0.04$ & $96.89 \pm 0.04$ & $95.76 \pm 0.06$ & $93.78 \pm 1.62$ \\
$[4000,5000[$ & $97.31 \pm 0.04$ & $97.28 \pm 0.04$ & $96.99 \pm 0.04$ & $95.94 \pm 0.05$ & $93.78 \pm 0.07$ \\
$[3500,4000[$ & $97.31 \pm 0.04$ & $97.29 \pm 0.04$ & $97.04 \pm 0.04$ & $96.16 \pm 0.05$ & $93.86 \pm 0.13$ \\
$[3000,3500[$ & $97.32 \pm 0.04$ & $97.30 \pm 0.04$ & $97.08 \pm 0.04$ & $96.28 \pm 0.05$ & $93.89 \pm 0.12$ \\
$[2500,3000[$ & $97.32 \pm 0.04$ & $97.31 \pm 0.04$ & $97.12 \pm 0.04$ & $96.28 \pm 0.05$ & $94.09 \pm 0.12$ \\
$[2000,2500[$ & $97.32 \pm 0.04$ & $97.31 \pm 0.04$ & $97.13 \pm 0.04$ & $96.26 \pm 0.05$ & $94.63 \pm 0.14$ \\
$[1500,2000[$ & $97.32 \pm 0.04$ & $97.30 \pm 0.04$ & $97.08 \pm 0.04$ & $96.34 \pm 0.06$ & $93.75 \pm 0.08$ \\
$[1000,1500[$ & $97.33 \pm 0.04$ & $97.30 \pm 0.04$ & $97.06 \pm 0.04$ & $96.14 \pm 0.09$ & $93.61 \pm 0.09$ \\
$[0,1000[$ & $97.15 \pm 0.50$ & $97.28 \pm 0.05$ & $97.03 \pm 0.08$ & $96.29 \pm 0.23$ & $94.67 \pm 0.88$ \\
\bottomrule\end{tabular}

\end{table}
