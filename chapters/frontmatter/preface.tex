\chapter{Preface}

% To quote regulation 4.2.b:
%   It is helpful, particularly to external examiners, if a brief statement is 
%   included giving the candidate's degree(s) and research experience, even if 
%   the latter consists only of the work done for this thesis. This may be 
%   untitled or it may be headed 'Preface' or 'The Author' or similar.

The work presented in this thesis was performed in the context of the \lhcb\ 
experiment, a collaboration of around 800 scientists and engineers.
As such, it would not have been possible to perform without the often implicit 
help of many.
The Author worked specifically on the analysis of the data after it was 
collected by the detector and processed by software common to most other 
analyses.

The charm production measurements presented in \cref{chap:prod} are the result 
of the work of one Master's student, Christopher Burr, two PhD students, 
Dominik M\"{u}ller and the Author, and four post-doctoral researchers, Sajan 
Easo, Marco Gersabeck, Vava Gligorov, and Patrick Spradlin.
The bulk of the analysis work was performed by the three students, with the 
Author co-ordinating the team as a whole.
The Author was primarily responsible for the event and candidate selection, the 
yield extraction, and some particular efficiency evaluation and systematic 
studies, although the measurement was truly the work of everyone, with all 
parts being touched by most analysts at some point.
The measurements were published in a peer-reviewed journal in 
2016~\cite{Aaij:2015bpa}.
Similar measurements using data taken at \sqrtseq{5} were performed in 2016 by 
Dominik M\"{u}ller and the Author~\cite{Aaij:2016jht}.

The \CP\ violation search presented in \cref{chap:cpv} is solely the work of 
the Author, with supervision from Sajan Easo, George Lafferty, and Patrick 
Spradlin.

In addition to the work detailed in this thesis, the Author also: implemented a 
web-based monitoring tool for the \velo\ sub-detector, the architecture of 
which was adopted for the central web-based detector monitoring tool used 
during \runtwo; helped found the \lhcb\ Starterkit group, who aim to teach 
young researchers software literacy and the basics of the \lhcb\ software suite 
through three- and four-day workshops; and contributed to a public note on the 
method of \acl*{PID} calibration widely used within the 
collaboration~\cite{Anderlini:2202412}.
