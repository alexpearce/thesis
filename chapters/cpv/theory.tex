\chapter{Formalism}
\label{chap:cpv:theory}

The origin of \CP\ violation in weak interactions is described in 
\cref{chap:intro:sm:cp}.
This \namecref{chap:cpv:theory} shall describe the formalism of the measurement 
of \dACP, and shall enumerate the production and detection asymmetries that may 
enter.

The asymmetry in the decay of a \PLambdac\ baryon ($\Pup\Pdown\Pcharm$) to some 
final state $f$ is given by the parameter \ACP\ in 
\cref{eqn:cpv:introduction:acp}.
Measuring the decay rate $\Gamma$ experimentally requires counting the number 
of \PLambdac\ at production, which is hard to measure precisely, as shown in 
\cref{chap:prod}.
It is simpler to count only the number $N$ of reconstructed \PLambdac\ decays, 
and to form an asymmetry parameter as in \cref{eqn:cpv:introduction:araw}
This is the raw asymmetry as is it not necessarily equal to \ACP, being 
contaminated by possible detector and production effects.
This analysis reconstructs the \PLambdac\ decays \pKK\ and \ppipi, originating 
from \decay{\PLambdab}{\PLambdac\Pmuon X}\ decays.
The value of \ARaw\ can then include the effects of the \PLambdab/\APLambdab\ 
production asymmetry, the \Pmuon/\APmuon detection asymmetry, and the detection 
asymmetry of the \PLambdac\ final state $f$/$\bar{f}$, if they are non-zero.

The \PLambdab\ production asymmetry is predicted to be non-zero due to the \pp\ 
collisions provided by the \ac{LHC}~\cite{PhysRevD.90.014023}, and \lhcb\ has 
found evidence of such as a function of \PLambdab\ rapidity 
\lhcb~\cite{Aaij:2015fea}.
The higher `availability' of matter quarks compared to antimatter quarks 
results in higher production cross sections of matter hadrons compared to that 
of antimatter.

% TODO should make a statement like the one for the det. asyms., that more Lb 
% means more Lc means more f, looking like a non-zero ACP
The two detection asymmetries are also expected to be non-zero, due to the 
known differences between the hadron/anti-hadron and muon/anti-muon 
cross-sections with matter.
The final state detection asymmetry $f$ can be broken down into two terms: one 
due to the detection asymmetry of the proton; and another due to the detection 
asymmetry of the \KmKp pair and the \pimpip\ pair.
If the meson kinematics are equal within a given mode, for example if the 
\PKminus kinematics are identical to those of the \PKplus\ in the \pKK\ data, 
then the final state asymmetry reduces to the proton detection asymmetry.

Both the \PLambdab\ production and the muon detection asymmetries should be 
independent of the \PLambdac\ final state $f$, given that the \PLambdac\ 
kinematics are equal between \pKK and \ppipi, but different acceptance 
efficiencies for \pKK\ and \ppipi\ will likely invalidate this assumption.

The number of reconstructed \LcTof\ decays can be expressed as the product of 
several effective probabilities
\begin{equation}
  N(f) = \prob(\PLambdab)\cdot
         \Gamma(\LbToLcmuX)\cdot
         \Gamma(\LcTof)\cdot
         \eff(\Pmuon)\cdot
         \eff(f),
  \label{eqn:cpv:theory:yield}
\end{equation}
and similarly for $N(\bar{f})$, where $\prob(\PLambdab)$ is the probability of 
producing a $\PLambdab$ baryon given a \pp\ collision, and $\eff(\Pmuon)$ and 
$\eff(f)$ are the muon and $\PLambdac$ final-state detection efficiencies.
% TODO can we back up this assumption?
It is assumed that the \LbToLcmuX decay is \CP-symmetric, but that all other 
factors may not be, for example $\eff(\Pmuon) \neq \eff(\APmuon)$.
In the following, it will be shown how a variable can be constructed to 
eliminate the various background asymmetries, leaving only the asymmetry of 
interest, \ACP\@.

To express \ARaw\ in terms of all of its component asymmetries, it is useful to 
first define the notation of a general asymmetry parameter $X$, which describes 
the asymmetry between two quantities $x$ and $\bar{x}$
\begin{equation}
  X = \frac{x - \bar{x}}{x + \bar{x}}.
  \label{eqn:cpv:theory:generic_asym}
\end{equation}
This general form can be rearranged as
\begin{align}
  x &= \frac{1}{2}(x + \bar{x})(1 + X),\label{eqn:cpv:theory:asym_form_one}\\
  \bar{x} &= \frac{1}{2}(x + \bar{x})(1 - X).\label{eqn:cpv:theory:asym_form_two}
\end{align}
An asymmetry parameter like that in \cref{eqn:cpv:theory:generic_asym} can be defined for each of the relevant terms in 
\cref{eqn:cpv:theory:yield}
\begin{align*}
  \APLb(f) &= \frac{%
    \prob(\PLambdab) - \prob(\APLambdab)
  }{%
    \prob(\PLambdab) + \prob(\APLambdab)
  },\\
  \ADmu(f) &= \frac{%
    \eff(\Pmuon) - \eff(\APmuon)
  }{%
    \eff(\Pmuon) + \eff(\APmuon)
  },\\
  \ADf(f)  &= \frac{%
    \eff(f) - \eff(\bar{f})
  }{%
    \eff(f) + \eff(\bar{f})
  },
\end{align*}
and the asymmetry in $\Gamma(\LcTof)$ is \ACP, in 
\cref{eqn:cpv:introduction:acp}.
% TODO should say why
Each parameter is, at least implicitly, dependent on the detected final state 
$f$.

Substituting in Equation~\ref{eqn:cpv:theory:yield} to 
Equation~\ref{eqn:cpv:introduction:araw}
\begin{equation*}
  \ARaw(f) = \frac{%
    \prob(\PLambdab)\Gamma(f)\eff(\Pmuon)\eff(f) - 
    \prob(\APLambdab)\Gamma(\bar{f})\eff(\APmuon)\eff(\bar{f})
  }{%
    \prob(\PLambdab)\Gamma(f)\eff(\Pmuon)\eff(f) + 
    \prob(\APLambdab)\Gamma(\bar{f})\eff(\APmuon)\eff(\bar{f})
  },
\end{equation*}
and then substituting each quantity for its equivalent form as in 
\cref{eqn:cpv:theory:asym_form_one,eqn:cpv:theory:asym_form_two}, all factors 
of $\sfrac{1}{2}$ and all factors of the form $(x - \bar{x})$ cancel, leaving
% TODO format this disgustingly long equation
% \begin{equation*}
%   \ARaw(f) = \frac{%
%     \APLb\ADmu\ADf + \APLb\ADmu\ACP + \APLb\ADf\ACP + \ADmu\ADf\ACP + \APLb + 
%     \ADmu + \ADf + \ACP
%   }{%
%     1 + \APLb\ADmu + \APLb\ADf + \APLb\ACP + \ADmu\ADf + \ADmu\ACP + \ADf\ACP + 
%     \APLb\ADmu\ADf\ACP
%   }.
% \end{equation*}
\begin{equation}
\ARaw(f) = \frac{Y}{Z}.
\end{equation}
where (dropping the final state parameter temporarily for compactness)
\begin{align}
Y = \APLb\ADmu\ADf &+ \APLb\ADmu\ACP + \APLb\ADf\ACP + \ADmu\ADf\ACP \nonumber\\
                   &+ \APLb +  \ADmu + \ADf + \ACP,
\end{align}
and
\begin{align}
Z = 1 &+ \APLb\ADmu + \APLb\ADf + \APLb\ACP + \ADmu\ADf + \ADmu\ACP \nonumber\\
      &+ \ADf\ACP + \APLb\ADmu\ADf\ACP.
\end{align}
Assuming that the individual asymmetries are small, of the order of 
\SI{1}{\percent}, the product of two or more asymmetries is negligible with 
respect to the leading order, and so
\begin{equation}
  \ARaw(f) \approx \ACP(f) + \APLb(f) + \ADmu(f) + \ADf(f).
  \label{eqn:cpv:theory:araw_approx}
\end{equation}

By assuming that these background asymmetries are mode-independent, that is to 
say $\AD(f) = \AD(g)$ and $\AP(f) = \AP(g)$, the two \PLambdac\ decay modes 
\pKK\ and \ppipi\ can be considered simultaneously and the difference in their 
raw asymmetries can be measured, eliminating the background asymmetries by 
construction
\begin{align}
  \dACP &= \ARaw(\pKK) - \ARaw(\ppipi),\label{eqn:cpv:theory:dacp}\\
        &\approx \ACP(\pKK) - \ACP(\ppipi)\nonumber.
\end{align}

The assumption that the production and detection asymmetries in 
\cref{eqn:cpv:theory:araw_approx} are mode independent is not true in general.
By making a more reasonable assumption that these background asymmetries are 
dependent only on the kinematics of the representative particles, this is made 
clear by considering that different final states will in general have different 
reconstruction and selection efficiencies as a function of \PLambdac\ 
kinematics.
Two samples with different \PLambdac\ kinematics will likely also have 
different \PLambdab\ and muon kinematics, and hence there can still be net 
production and detection asymmetries in \dACP\@.

The assumption that the background asymmetries depend only on particle 
kinematics is not unreasonable.
The production asymmetry, for example, is a difference in cross sections, which 
are generally parameterised by the kinematics of the produced particle (some 
combination of momentum \ptot, transverse momentum \pT, and pseudorapidity 
\Eta), and of the properties of the colliding beams (which are independent of 
the produced particle).
Similarly, a detection asymmetry describes the differences of material 
interactions between matter and antimatter, and these are dependent on the 
momentum of the particle in question and, assuming a non-uniform material 
distribution, its flight path.
These can be described by some combination of the kinematic variables 
previously given.

Given a sample of \PLambdac\ decays with equal kinematics, it is reasonable to 
assume that the kinematics of the muon and of the \PLambdab\ are independent of 
the decay modes present in that sample.
As it is not given that the \PLambdac\ kinematics are equal between \LcTopKK 
and \LcToppipi decays, the \PLambdac\ kinematics will be weighted to match one 
another.
The background asymmetries \AP\ and \ADmu\ will cancel in the difference 
\dACP\@.
The assumption of matching \PLambdab\ and \Pmuon kinematics after the weighting 
will have to be shown explicitly.

Finally, it is required to show that the kinematics of the protons between the 
\pKK\ and \ppipi\ also agree, assuming the \hmhp\ kinematics agree within the 
modes.
It is not so obvious, as before, that these variables would a priori match, and 
indeed it seems sensible to assume that they would not, and so it must be 
demonstrated explicitly whether it is the case.

These considerations govern the analysis strategy: measure the number of 
\PLambdac\ candidates in the \pKK\ and \ppipi\ samples after weighting the 
\PLambdac\ kinematics to look alike, such that the \PLambdab\ and \Pmuon 
kinematics also agree.
Additional weighting may be required to align the proton kinematics.

\section{Decay phase space}
\label{chap:cpv:theory:phsp}

The phase space of a decay is the set of variables which fully parameterises 
all possible dynamics.
For the decay of a pseudo-scalar to three pseudo-scalars, such as 
\decay{\PDzero}{\PKshort\Ppiplus\Ppiminus}, the spin symmetry allows for the 
phase space to be fully parameterised by two variables.
These are usually taken to be two child-pair squared masses, which can 
visualised as a Dalitz plot.
In the case of \LcTophh, the spin \sfrac{1}{2} of the proton means that 
three-body system is no longer rotationally system, and three additional 
variables are required to describe the phase space, for a total of five.

For a meaningful comparison of the measurement of \dACP\ with theoretical 
predictions, which are not presently available, the efficiency of the 
\PLambdac\ selection across the five-dimensional phase space must be known.
The definition of ``selection'' includes the effects of the \lhcb\ acceptance 
and the trigger, stripping and offline requirements.
The efficiency model can either be provided to theorists so that they can apply 
the same efficiency to their predictions, or it can be used to correct the data 
before the \dACP\ measurement is made.

The five dimensions of the phase space is defined here in a similar way to the 
\LcTopKpi\ amplitude analysis performed by the \esno\ 
collaboration~\cite{Aitala:1999uq}.
This defines two child-pair square masses, of the proton and oppositely-charged 
child \msqphm\ and of the two opposite-sign pseudo-scalars \msqhh, and three 
decay angles.
The child-pair square masses are invariant under Lorentz transformations, but 
the decay angles are not and so a definition of the frame in which they are 
computed is required.

The \esno\ analysis defines a coordinate system in the \PLambdac\ rest frame, 
and this convention is followed here.
The $z$-axis, also called the quantisation axis or polarisation axis \polzlcp, 
is perpendicular to the plane of production
\begin{equation}
  z = \polzlcp = \phatbeam \times \phatlcp,
\end{equation}
where \phatbeam\ is the direction of the beam\footnotemark\ and \phatlcp\ is 
the direction of the \PLambdac\ measured in the laboratory frame.
\footnotetext{%
  E791 was a fixed-target experiment, colliding a \SI{500}{\GeVc} pion beam 
  with metal foils.
}
The $x$-axis of the \PLambdac\ rest frame is \phatlcp.
As the \PLambdac\ candidates in this analysis are not produced directly from 
the \pp\ collision but in the decays of \PLambdab\ baryons, the `beam 
direction' is defined as the direction of the \PLambdab\ momentum vector, which 
is equal to the direction vector pointing from the \pp\ primary vertex 
$v_{\pp}$ to the $\PLambdac\Pmuon$ vertex $v_{\PLambdac\Pmuon}$
\begin{equation}
  \phatbeam = \phatlbz = v_{\PLambdac\Pmuon} - v_{\pp}.
\end{equation}
With this coordinate system and inertial frame, the three decay angles are 
defined as:
\begin{enumerate}
  \item The angle \thetap\ between the proton momentum vector and the $z$-axis;
  \item The angle \phip\ between the proton momentum vector and the $x$-axis; 
    and
  \item The angle \phihh\ between the plane containing the proton momentum 
    vector and the $z$-axis and the plane containing the two pseudo-scalar 
    meson momentum vectors.
\end{enumerate}
These definition are illustrated in \cref{fig:cpv:theory:phsp:angles}

The distributions of the five phase space variables will be presented in 
\cref{chap:cpv:phsp}, along with the evaluation of the efficiency as a function 
of the position in phase space.

\begin{figure}
  \begin{subfigure}{0.5\textwidth}
    \resizebox{\textwidth}{!}{%
      % Rotation of the axes wrt the viewer
% syntax: \tdplotsetdisplay{\theta_d}{\phi_d}
\tdplotsetmaincoords{70}{110}

% Define proton coordinates
\pgfmathsetmacro{\rp}{.8}
\pgfmathsetmacro{\angthetap}{60}
\pgfmathsetmacro{\angphip}{50}

% Define Kpi coordinates
\pgfmathsetmacro{\rKpi}{.4}
\pgfmathsetmacro{\thetaKpi}{80}
\pgfmathsetmacro{\phiKpi}{340}

% Start a TikZ picture, using the tdplot_main_coords system
% tdplot provides the coordinate transformation
\begin{tikzpicture}[scale=4,tdplot_main_coords]

% Define the origin
\coordinate (O) at (0,0,0);

% Create a point P with coordinates r, theta, and phi
% I think this is similar to \coordinate, but also transform the vector
% in to the rotated coordinate system
\tdplotsetcoord{P}{\rp}{\angthetap}{\angphip}
\tdplotsetcoord{Kpi}{\rKpi}{\thetaKpi}{\phiKpi}

% Draw main coordinate axes
% The anchor key relates to the positioning of the label
% South is towards the bottom of the diagram, east is to the left
\draw[thick,->] (0,0,0) -- (1.0,0,0) node[anchor=north west]{
  $x = \hat{p}_{\PLambdac}$
};
\draw[thick,->] (0,0,0) -- (0,0.5,0) node[anchor=north west]{
  $y$
};
\draw[thick,->] (0,0,0) -- (0,0,0.5) node[anchor=south]{
  $z = \polzlcp = \phatlbz \times \phatlcp$
};

% Draw a vector from origin to point P
% The -stealth' option is the type of arrowhead
\draw[-stealth',color=red] (O) -- (P) node[anchor=west]{\Pproton};
\draw[-stealth',color=red] (O) -- (Kpi) node[anchor=east]{\hmhp};

% Draw projection of vector on to xy plane, and a connecting line
\draw[dashed, color=red] (O) -- (Pxy);
\draw[dashed, color=red] (P) -- (Pxy);

% Draw angle phi of xy projection
% {} arguments are center point, radius, angle-from, angle-to, options, label
% [] arguments are coordinate frame, draw options
\tdplotdrawarc{(O)}{0.25}{0}{\angphip}{anchor=north}{\phip}

%set the rotated coordinate system so the x'-y' plane lies within the
%"theta plane" of the main coordinate system
%syntax: \tdplotsetthetaplanecoords{\phi}
% Rotate the "theta plane" (xz) by phi around z
\tdplotsetthetaplanecoords{\angphip}

% Draw arc in rotated system
\tdplotdrawarc[tdplot_rotated_coords]{(0,0,0)}{0.35}{0}{\angthetap}{anchor=south west}{\thetap}

\end{tikzpicture}

    }
  \end{subfigure}
  \begin{subfigure}{0.5\textwidth}
    \resizebox{\textwidth}{!}{%
      % Rotation of the axes about x-axis, then about z-axis
% We align the x-axis with the negative proton momentum
\tdplotsetmaincoords{0}{0}

% Define proton coordinates
\pgfmathsetmacro{\rp}{0.7}
\pgfmathsetmacro{\angthetap}{90}
\pgfmathsetmacro{\angphip}{180}

% Polarisation axis 'coordinates'
\pgfmathsetmacro{\rpol}{0.7}
\pgfmathsetmacro{\thetapol}{90}
\pgfmathsetmacro{\phipol}{135}

% Define h+h- coordinates
\pgfmathsetmacro{\rKpi}{.4}
\pgfmathsetmacro{\thetaKpi}{80}
\pgfmathsetmacro{\phiKpi}{340}

% Define h+ coordinates
\pgfmathsetmacro{\rhp}{.4}
\pgfmathsetmacro{\thetahp}{90}
\pgfmathsetmacro{\phihp}{-20}

% Define h+ coordinates
\pgfmathsetmacro{\rhm}{.6}
\pgfmathsetmacro{\thetahm}{90}
\pgfmathsetmacro{\phihm}{30}

% Start a TikZ picture, using the tdplot_main_coords system
% tdplot provides the coordinate transformation
\begin{tikzpicture}[scale=4,tdplot_main_coords]

% Define the origin
\coordinate (O) at (0,0,0);

% Create a point P with coordinates r, theta, and phi
% I think this is similar to \coordinate, but also transform the vector
% in to the rotated coordinate system
\tdplotsetcoord{P}{\rp}{\angthetap}{\angphip}
\tdplotsetcoord{Hp}{\rhp}{\thetahp}{\phihp}
\tdplotsetcoord{Hm}{\rhm}{\thetahm}{\phihm}
\tdplotsetcoord{Pol}{\rpol}{\thetapol}{\phipol}
\tdplotsetcoord{Kpi}{\rKpi}{\thetaKpi}{\phiKpi}

% h+h- plane parameters
\pgfmathsetmacro{\hhHeight}{1}
\pgfmathsetmacro{\hhWidth}{0.7}
\pgfmathsetmacro{\hhSkew}{30}
\pgfmathsetmacro{\hhSkewWidth}{tan(\hhSkew)*\hhHeight}
\coordinate (hhOrigin) at ($(-\hhSkewWidth/2,-0.5)$);

% Draw planes before anything else, as fill would cover them
% Proton-z plane
\filldraw[semithick,black,fill=white] ($(P)-(0.1,0.6)$) rectangle (0,0.8);
% h+h- plane
\filldraw[semithick,black,fill=white] (hhOrigin)
  -- ($(hhOrigin)+(\hhSkewWidth, \hhHeight)$)
  -- ($(hhOrigin)+(\hhSkewWidth+\hhWidth,\hhHeight)$)
  -- ($(hhOrigin)+(\hhWidth, 0)$)
  % Close the path
  -- cycle;
% Angle between planes
% \draw[dotted] (hhOrigin) -- ($(hhOrigin)+(0, \hhHeight)$);
\tdplotdrawarc[dotted,very thick]{(O)}{0.5}{90}{
  90-\hhSkew
}{anchor=south west}{\Large$2\pi - \phihh$}

% Draw main coordinate axes
% The anchor key relates to the positioning of the label
% South is towards the bottom of the diagram, east is to the left
% \draw[thick,->] (O) -- (1,0,0) node[anchor=north west]{
%   $x = \hat{p}_{\PLambdac}$
% };
% \draw[thick,->] (O) -- (0,1,0) node[anchor=north west]{
%   $y$
% };

% Proton
\draw[-stealth',color=red] (O) -- (P) node[anchor=south]{\Large\Pproton};
% Polarisation axis
\draw[thick,->] (O) -- (Pol) node[anchor=south]{
  \Large$z = \polzlcp$
};

% h+
\draw[-stealth',color=red] (O) -- (Hp) node[anchor=north]{\Large\Php};
% h-
\draw[-stealth',color=red] (O) -- (Hm) node[anchor=south]{\Large\Phm};

% Draw projection of vector on to xy plane, and a connecting line
% \draw[dashed, color=red] (O) -- (Pxy);
% \draw[dashed, color=red] (P) -- (Pxy);

% Draw angle phi of xy projection
% {} arguments are center point, radius, angle-from, angle-to, options, label
% [] arguments are coordinate frame, draw options
\tdplotdrawarc{(O)}{0.25}{\angphip}{\phipol}{anchor=east}{\Large\thetap}

%set the rotated coordinate system so the x'-y' plane lies within the
%"theta plane" of the main coordinate system
%syntax: \tdplotsetthetaplanecoords{\phi}
% Rotate the "theta plane" (xz) by phi around z
% \tdplotsetthetaplanecoords{\angphip}

% Draw arc in rotated system
% \tdplotdrawarc[tdplot_rotated_coords]{(0,0,0)}{0.35}{0}{\angthetap}{anchor=south 
% west}{$\theta_{p}$}

\end{tikzpicture}

    }
  \end{subfigure}
  \caption{%
    Definition of inertial reference frame axes and \LcTophh\ phase space decay 
    angles.
    Adapted from Figure~1 in the \esno\ \LcTopKpi\ amplitude analysis 
    paper~\cite{Aitala:1999uq}.
  }
  \label{fig:cpv:theory:phsp:angles}
\end{figure}
