\chapter{Introduction}
\label{chap:cpv:introduction}

In 2011, the \lhcb\ collaboration announced the first evidence of \CP\ 
violation in charm decays, claiming a deviation from the \CP\ symmetry 
hypothesis at the
\SI{3.5}{\sigma} level~\cite{Aaij:2011in}.
The measurement was of the rate of direct \CP\ violation in the \ac{SCS} decays 
of the \PDzero meson, $\decay{\PDzero}{\KmKp}$ and \pimpip, so called as the 
weak decay changes the flavour of the up-type charm quark to either a strange 
or a down quark, which are suppressed relative to \ac{CF} decays, where the 
flavour changes to another up-type quark.

The rate of direct \CP\ violation in the decay of any particle $P$ to a final 
state $f$ is parameterised as
\begin{equation}
  \ACP(\decay{P}{f}) = \frac{%
    \Gamma(\decay{P}{f}) - \Gamma(\decay{\bar{P}}{\bar{f}})
  }{%
    \Gamma(\decay{P}{f}) + \Gamma(\decay{\bar{P}}{\bar{f}})
  },
  \label{eqn:cpv:introduction:acp}
\end{equation}
that is the relative rate of the \decay{P}{f} process to its \CP\ conjugate.
For \decay{\PDzero}{\hmhp} decays, this is expected to be of the order of 
\num{e-3} in the \ac{SM}~\cite{Grossman:2006jg}, and so it sensitive to 
contributions from physics \acl{BSM}.
% TODO should make sure I know what this actually is
Using arguments of $U$-spin symmetry between the \KmKp\ and \pimpip\ final 
states, the sign of $\ACP(\KmKp)$ is predicted to be opposite that of 
$\ACP(\pimpip)$~\cite{Grossman:2006jg}.
The difference between the two is then particular sensitive to direct \CP\ 
violation in this system
\begin{equation}
  \dACP = \ACP(\pKK) - \ACP(\ppipi).
  \label{eqn:cpv:introduction:dacp_pure}
\end{equation}
It was the value of \dACP\ that was measured to deviate significantly from 
zero.

The observation generated much interest in the \acl{HEP} theory 
community~\cite{Lenz:2013pwa}, creating a push for a better mathematical 
understanding of \CP\ violation in charm decays.
However, subsequent measurements of \dACP, by including additional data and 
using \PDzero mesons from \PB\ decays, have not confirmed the initial 
evidence~\cite{Aaij:2014gsa,Aaij:2016cfh}, and the current world average is 
consistent with the hypothesis of no direct \CP\ violation in \PDzero 
decays~\cite{Amhis:2014hma}, being\footnotemark
\begin{equation}
  \dACP = \SI{-0.252 \pm 0.104}{\percent}.
  \label{eqn:cpv:introduction:dacp_world_average}
\end{equation}
This demonstrates that the experimental precision is now at level of some 
\ac{SM} predictions.
Future measurements of \dACP\ with \PDzero decays at \lhcb\ are of top 
priority.

\footnotetext{%
  The world average is not strictly of \dACP, but of 
  $\Delta{a}_{\CP}^{\text{dir}}$, which includes effects in \emph{indirect} 
  \CP\ violation, which arise to the possibility of \CP\ violation in 
  \PDzero-\APDzero mixing~\cite{Gersabeck:2011xj}.
}

Charm baryons, the lightest of which is the \PLambdac~($\Pup\Pdown\Pcharm)$, 
are an additional system in which to search for \CP\ violation.
Measurements using baryons are particularly evocative, as they relate directly 
to the as-yet unexplained imbalance of baryons and antibaryons in the 
observable universe, and as such are a potential source of signs of \ac{BSM} 
physics.
The first evidence for \CP\ violation in baryon decays was found only 
recently~\cite{Aaij:2016cla}, using the decays of the lightest beauty baryon, 
the \PLambdab~($\Pup\Pdown\Pbottom$).
Charm baryons are then relevant not only because their study may uncover the 
first evidence of \CP\ violation in the up-type quark sector, but also because 
they can strengthen the understanding of the mechanisms that create baryon 
asymmetries.

An analogous measurement of \dACP\ using \PLambdac\ decays, to that made with 
\PDzero decays, is searching for direct \CP\ violation in the decays \LcTopKK\ 
and \LcToppipi.
The dynamics of these \ac{SCS} decays, which can only be fully parameterised by 
a five-dimensional phase space, are currently poorly 
understood~\cite{Bigi:2012ev}, and so any experimental input can be useful to 
guide and constrain the theory.
The previous generation of heavy flavour collider experiments, such as \belle\ 
and \babar, collected small samples of \PLambdac~\cite{Seuster:2005tr}, whereas 
the large charm cross-section at \lhcb\ enables precision measurements.

The large sample of \PB decays collected by \lhcb\ can also be used to study 
\CP\ violation in \PLambdac\ decays.
The \PLambdab\ has a lifetime around seven times greater than that of the 
\PLambdac, which can be exploited to provide excellent background rejection, 
and the semileptonic decay chain \LbToLcmuX\ would allow for a simple 
triggering scheme, by requiring a single, high \pT\ muon whose trajectory is 
significantly displaced from the \ac{PV}.
A related inclusive branching fraction for this decay is~\cite{PDG2014}
\begin{equation}
  \bfrac(\decay{\PLambdab}{\PLambdac\Pleptonminus\APnu_{\Plepton}}) =
    \SI{10.7 \pm 2.2}{\percent}.
  \label{eqn:cpv:introduction:lb_bfrac}
\end{equation}
An estimate for the number of \LcTophh\ produced from semileptonic \PLambdab\ 
decays at \lhcb\ during \runone\ of the \ac{LHC} can be computed as
\begin{equation}
  N(\LcTof, \zeta) =
    \xsec(\PLambdab, \zeta)\times
    \bfrac(\LbToLcmuX)\times
    \bfrac(\LcTof)\times
    \lumi(\zeta),
\end{equation}
where $\zeta$ is the year of data-taking, either 2011 or 2012; 
$\xsec(\PLambdab)$ the \PLambdab\ cross-section, measured at \sqrtseq{7} and 
\sqrtseq{8} to be \SI{20}{\micro\barn} and 
\SI{25}{\micro\barn}~\cite{Aaij:2015fea}; $\bfrac(\LbToLcmuX)$ is taken to be 
the branching fraction from \cref{eqn:cpv:introduction:lb_bfrac} divided by 3, 
\SI{3.6 \pm 0.7}{\percent},
assuming it is equally distributed between the leptons and that there are no 
other contributions of a similar magnitude; $\bfrac(\LcTof)$ is the branching 
fraction of the \PLambdac\ to either \pKK\ or \ppipi, where $\bfrac(\LcTopKK) = 
(1.61 \pm 0.26)\times 10^{-3}$ and $\bfrac(\LcToppipi) = (3.91 \pm 0.4)\times 
10^{-3}$~\cite{Ablikim:2016tze}; and \lumi\ is the integrated luminosity, 
$\lumi(2011) = \SI{1}{\per\femto\barn}$ and $\lumi(2012) = 
\SI{2}{\per\femto\barn}$.
This gives
\begin{align}
  N(\LcTopKK, 2011) &\approx (1.2 \pm 0.4) \times 10^{6},\nonumber\\
  N(\LcTopKK, 2012) &\approx (2.8 \pm 1.0) \times 10^{6},\nonumber\\
  N(\LcToppipi, 2011) &\approx (2.8 \pm 0.9) \times 10^{6},\nonumber\\
  N(\LcToppipi, 2012) &\approx (6.8 \pm 2.1) \times 10^{6}.
\end{align}
Assuming the full reconstruction and selection efficiencies are about 
\SI{1}{\percent}, and that the precision to which \dACP\ can be measured is 
dominated by the statistical uncertainty on the signal yield $N$, $\unc(\dACP) 
\approx 1/\sqrt{N}$, a measurement of \dACP\ using the \runone\ data should be 
able to achieve a precision of the order of \SI{0.1}{\percent}.

This analysis aims to measure \dACP\ using \LcTopKK\ and \LcToppipi decays, 
reconstructed in association with a high \pT\ muon assumed to originate from 
the decay of a \PLambdab, using data collected with the \lhcb\ detector in 2011 
and 2012, corresponding to an integrated luminosity of \SI{3}{\per\femto\barn}.
For the remainder of this \namecref{chap:cpv:introduction}, an overview of the 
analysis strategy will be discussed, to summarise and contextualise the 
remaining body of this \namecref{chap:cpv}.
% In addition, the phase space available to the final state is fully 
% parameterised by five dimensions, allowing for the possibility of rich 
% dynamics which vary across it.

% The breaking of the combined charge-parity~(\CP) symmetry in weak interactions, 
% as well as the definition \CP\ transformation, has been discussed in 
% \cref{chap:intro:sm}.
% \CP\ violation is necessary to generate the matter-antimatter asymmetry in the 
% universe, and in particular the process of baryogenesis which created the 
% imbalance between baryons and anti-baryons.
% The experimental study of \CP\ asymmetries is then intimately connected to a 
% wider understanding of nature.

% Only interactions of the weak force have been observed to violate \CP\ 
% symmetry.
% The strong force, while not theoretically invariant under the \CP 
% transformation, has not been experimentally observed to violate \CP\@.
% Electromagnetism, the remaining force described by the \acl{SM}, is invariant 
% in its theoretical definition.
% However, the magnitude of the asymmetry produced by \ac{SM} processes is not 
% large enough to account for the origins of the universe, and so additional 
% sources of \CP\ violation must exist.

\section{Analysis overview}
\label{chap:cpv:introduction:overview}

Rather than measuring the relative difference on decay rates in 
\cref{eqn:cpv:introduction:acp}, it is simpler to measure the asymmetry in the 
observed yields $N$
\begin{equation*}
  \ARaw(f) = \frac{N(f) - N(\bar{f})}{N(f) + N(\bar{f})},
\end{equation*}
where here, and throughout this \namecref{chap:cpv}, the origin of the final 
state $f$ from a \PLambdac\ decay is implicit.
However, this is not necessarily equal to the decay asymmetry in 
\cref{eqn:cpv:introduction:acp}.
For example, if more \PLambdab\ baryons are produced in \pp\ collisions than 
\APLambdab, \ARaw\ will contain a contribution from that asymmetry.
% The enumeration of the various production and detection asymmetries that enter 
% \ARaw\ is given in \cref{chap:cpv:theory}.
By assuming that each background asymmetry is independent of the decay mode, 
their contributions will cancel in the difference \dACP\@.
In general, however, this assumption is not true, and instead the asymmetries 
depend on the kinematics of the particles involved, such as the \PLambdab\ 
transverse momentum and pseudorapidity.
If these kinematics are equal between the modes, they will cancel in \dACP\@.
To ensure that this is the case, the kinematics are weighted between the modes.
The signal distributions for the weighting are extracted via \chisq\ fits to 
the \phh\ invariant mass distribution.

Any theoretical computation of \ACP\ or \dACP\ for \LcTophh\ decays requires 
some model for the five-dimensional \phh\ phase space, as it may be that the 
strength of \CP\ violation varies across it.
The selection and reconstruction of the \PLambdac\ candidates may sculpt this 
distribution, and so the experimental efficiency as a function of the phase 
space is computed, and the signal yields are corrected for it.
For each mode, the quantity \ARaw\ is measured by simultaneous fitting the 
\PLambdac\ and \APLambdac\ invariant mass distributions, which are weighted by 
the product of the kinematic weights and the inverse of the phase space 
efficiency.

Provided that the kinematic weighting equalises the distributions between the 
modes exactly, that the phase space efficiency is correctly modelled, and that 
the \chisq\ fit describes the signal and background components and 
normalisation correctly, the measurement of \dACP\ will be correct, within the 
statistical precision available with the data.
As it cannot be guaranteed that these assumptions are valid, they will be 
tested in the context of systematic studies.
Deviations from the nominal, measured value of \dACP\ that cannot be corrected 
for will be assigned as systematic uncertainties on the measurement.

\subsection{Document structure}
\label{chap:cpv:introduction:overview:structure}

The exact contributions of the various production and detection asymmetries to 
\ARaw will be derived and enumerated in \cref{chap:cpv:theory}.
The parameterisation of the five-dimensional phase space will also be given, 
which is used when constructing the phase space efficiency model.
The collision data and simulated \ac{MC} data used in the analysis is described 
in \cref{chap:cpv:data}, and the reconstruction and selection of the real data 
will be presented in \cref{chap:cpv:selection}.
The extraction of the signal yields and of the kinematic distributions of the 
signal from the fully selected data is presented in 
\cref{chap:cpv:prelim_fits}, and then those signal distributions are equalised 
between the two \PLambdac\ decay modes using \ac{MVA} techniques discussed in 
\cref{chap:cpv:kinematic_weighting}.
The resulting kinematic weights are combined with reconstruction and selection 
efficiencies computed as a function of \phh\ phase space, whose method of 
computation is shown is \cref{chap:cpv:phsp}, after which the combined weights 
enter the \chisq\ fits that are used to measure \ARaw, described in 
\cref{chap:cpv:araw}.
The combination of $\ARaw(\pKK)$ and $\ARaw(\ppipi)$ to measure \dACP\ is in 
\cref{chap:cpv:results}.
Methods for and the results of various systematic studies are given in 
\cref{chap:cpv:syst}.
Finally, \cref{chap:cpv:summary} gives the combination of the measurements of 
\dACP\ and the systematic studies, and concludes with a brief discussion on 
future prospects of \CP\ violation measurements at \lhcb.
