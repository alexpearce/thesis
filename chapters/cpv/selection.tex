\chapter{Selection}
\label{chap:cpv:selection}

The \PLambdac\ charm baryon has a lifetime of \SI{0.2}{\pico\second}, half that 
of the \PDzero\ and eight times less than that of the \PBzero.
Indeed, it is one of the shortest-lived weakly decaying heavy flavour hadrons, 
and as such presents particular challenges when a clean, efficient selection is 
desired as the secondary decay vertex is not so significantly displaced from 
the \ac{PV}.
Although the \bbbar\ cross-section is of the order of 100 times less than the 
\ccbar\ cross-section~\cite{LHCb-PAPER-2012-041,LHCb-PAPER-2013-004}, 
reconstructing the decay chain \LbToLcmuX\ allows for the exploitation of the 
long \PLambdab\ lifetime and the highly efficiency topological \PB decay 
triggers.
This analysis selects secondary \PLambdac\ candidates by association with a 
high \pT\ muon displaced from the \ac{PV}.
This \namecref{chap:cpv:selection} shall describe each stage of the selection.

\section{Stripping and trigger}
\label{chap:cpv:selection:stripping_trigger}

The selections in the stripping lines were not tuned specifically for this 
analysis, instead serving as general-purpose requirements suitable for a range 
of studies that use \LcTophh\ decays originating from semileptonic \PLambdab\ 
decays.
As such, they are similar to the selections of other \PB decays made at \lhcb.
They exploit the long decay times of \PB hadrons, such that they fly as 
significant, measurable distance in the \velo, and the high \pT\ and large 
\acl{IP} of the child vertices and tracks.
Additional requirements are similar in nature to those made for charm hadrons 
such as track and vertex fit quality and track \ipchisq, which are described in 
\cref{chap:prod:sel}.

The selection of final state particles by the stripping lines, namely protons, 
charged kaons and pions, and muons, is given in 
\cref{tab:cpv:selection:stripping_basic}.
One variable that has not been discussed previously is \pghost, the ghost 
probability, which is the response of an \ac{ANN} trained to discriminate 
between real and fake~(ghost) tracks~\cite{Brehmer:1478372}.
The \ac{ANN} response is transformed in such a way as to be within the range 
$[0, 1]$, and that it is linearly efficient at rejecting ghost tracks in 
simulated data in that range.

The selected protons, pions, and kaons are combined as \PLambdac\ candidate 
vertices, which are then combined with muon candidates to form \PLambdab\ 
candidates.
The requirements on the combinations and vertices are given in 
\cref{tab:cpv:selection:stripping_composite}.
As in \cref{chap:prod:sel}, the selection of composite particles proceeds in 
two stages: firstly by imposing requirements on the union of the inputs; and 
secondly on the vertex formed after the fit.

Trigger requirements are made after the \PLambdab\ candidate is created and 
selected in the stripping lines.
They are driven by the powerful discrimination provided by the muon from the 
\PLambdab.
The muon candidate is required to have fired the \lzero\ and \hltone\ single 
muon lines, and at \hlttwo the union of the tracks comprising the \PLambdab\ 
and \PLambdac\ candidates are required to have fired the topological 
semileptonic \PB lines, as described in \cref{chap:intro:lhcb:trigger}.
Distributions of the \pKK\ and \ppipi\ mass in the 2012 magnet down dataset 
after the stripping and trigger requirements are given in 
\cref{fig:cpv:data:mass}, as an illustration of the signal purity at this 
stage.

% As this table is so close to the beginning of the section, try to position it 
% at the bottom of the page so it doesn't come before the section header
\begin{table}[bp]
  \centering
  \caption{%
    Selection of tracks, with associated particle hypotheses, used in the 
    \PLambdab\ and \PLambdac\ stripping selection.
    Cuts listed under ``All'' are applied to all tracks.
  }
  \label{tab:cpv:selection:stripping_basic}
  \begin{tabular}{crl}
  Particle                 & Variable             & Cut value          \\
  \midrule
  \multirow{4}{*}{All}     & \pT                  & $> \SI{250}{\MeV}$ \\
                           & \ptot                & $> \SI{2}{\GeV}$   \\
                           & Track $\chisq/\ndof$ & $< 3$              \\
                           & \pghost              & $< 0.5$            \\
  \midrule
  \multirow{4}{*}{Muons}   & \pT                  & $> \SI{800}{\MeV}$ \\
                           & \ptot                & $> \SI{3}{\GeV}$   \\
                           & \ipchisq             & $> 4$              \\
                           & \dllmupi             & $> 0$              \\
  \midrule
  \multirow{2}{*}{Pions}   & \ipchisq             & $> 9$              \\
                           & \dllkpi              & $< 4$              \\
  \midrule
  \multirow{2}{*}{Kaons}   & \ipchisq             & $> 9$              \\
                           & \dllkpi              & $> 4$              \\
  \midrule
  \multirow{3}{*}{Protons} & \ipchisq             & $> 9$              \\
                           & \dllppi              & $> 4$              \\
                           & \dllpk               & $> \num{1e-10}$    \\
  \bottomrule
\end{tabular}

\end{table}

\begin{table}
  \centering
  \caption{%
    Stripping selection of particle combinations, vertices, and events.
  }
  \label{tab:cpv:selection:stripping_composite}
  \begin{tabular}{crl}
  Object                             & Variable                          & Cut value                    \\
  \midrule
  \multirow{2}{*}{Event}             & $N_{\text{Long tracks}}$          & $< 250$                      \\
                                     & $N_{\text{PV}}$                   & $> 0$                        \\
  \midrule
  \multirow{3}{*}{\phh}              & $|m - m_{\text{PDG}}(\PLambdac)|$ & $< \SI{100}{\MeV}$           \\
                                     & $\sum_{\text{Children}}\pT$       & $> \SI{1800}{\MeV}$          \\
                                     & DOCA \chisq                       & $< 20$                       \\
  \midrule
  \multirow{4}{*}{\PLambdac\ vertex} & $|m - m_{\text{PDG}}(\PLambdac)|$ & $< \SI{80}{\MeV}$            \\
                                     & $\sum_{\text{Children}}\pT$       & $> \SI{1800}{\MeV}$          \\
                                     & Vertex $\chisq/\ndof$             & $< 6$                        \\
                                     & Vertex distance \chisq            & $> 100$                      \\
  \midrule
  $\PLambdac\Pmuon$                  & $m$                               & $< \SI{6200}{\MeV}$          \\
  \midrule
  \multirow{4}{*}{\PLambdab\ vertex} & $m$                               & $2500 < m < \SI{6000}{\MeV}$ \\
                                     & Vertex $\chisq/\ndof$             & $< 6$                        \\
                                     & DIRA                              & $> 0.999$                    \\
                                     & $z_{\PLambdac} - z_{\PLambdab}$   & $> \SI{0}{\milli\metre}$     \\
  \bottomrule
\end{tabular}

\end{table}

\section{Offline}
\label{chap:cpv:selection:offline}

The offline selection serves three purposes: to increase the signal purity of 
the data after the stripping selection and trigger requirements; to remove 
specific physics backgrounds; and to process the data such that it is most 
suitable for offline analysis.

A mass window requirement is imposed on the \PLambdac\ candidates, which is 
equal to the range used in the mass fits (described in 
\cref{chap:cpv:prelim_fits})
\begin{equation}
  2230 < m(\PLambdac) < \SI{2350}{\GeV}.
  \label{eqn:cpv:selection:mass_window}
\end{equation}
This is the three-body \phh\ invariant mass computed in the vertex fit, rather 
than that from the decay tree fit.
Each fit made by \decaytreefitter\ has an associated \chisq, and fits that 
failed to converged are assigned a nominal negative value.
The decay tree fit is required to have converged by removing candidates with 
negative values of the fit \chisq\ and number of degrees of freedom
\begin{equation}
  \chisq_{\text{DTF}} > 0,\ N_{\text{DOF}} > 0.
  \label{eqn:selection:dtf_convergence}
\end{equation}

The \ppipi\ mode contains a number of $\Pproton\PKshort$ and $\PLambda\Ppiplus$ 
candidates, as shown in 
\cref{fig:cpv:selection:ppipi_ks_cut,fig:cpv:selection:ppipi_lambda_cut}.
These modes are non-resonant, and so the candidates are removed by applying an 
exclusion window \SI{40}{\MeV} wide centred near the nominal \PKshort 
mass~\cite{PDG2014}
\begin{equation}
  485 < m(\Ppiplus\Ppiminus) < \SI{510}{\MeV},
\end{equation}
and an exclusion window \SI{10}{\MeV} wide centred near the nominal \PLambda\ 
mass~\cite{PDG2014}
\begin{equation}
  1100 < m(\Pproton\Ppiminus) < \SI{1120}{\MeV}.
\end{equation}

\begin{figure}
  \begin{subfigure}[b]{0.5\textwidth}
    \includegraphics[width=\textwidth]{cpv/selection/LcToppipi_2012_MagDown_Lc_h1_h2_M}
    \caption{Full spectrum}
    \label{fig:cpv:selection:ppipi_ks_cut:full}
  \end{subfigure}
  \begin{subfigure}[b]{0.5\textwidth}
    \includegraphics[width=\textwidth]{cpv/selection/LcToppipi_2012_MagDown_Lc_h1_h2_M_zoom}
    \caption{Detail}
    \label{fig:cpv:selection:ppipi_ks_cut:zoom}
  \end{subfigure}
  \caption{%
    Mass spectrum of the $\Ppiplus\Ppiminus$ pair in the \ppipi\ decay mode, in 
    the magnet up 2012 data.
    The background contribution in the black spectrum has been statistically 
    subtracted, whereas the red spectrum has the signal component subtracted.
    The red lines indicate the boundaries of the requirement designed to remove 
    the $\Pproton\PKshort$ component of the sample.
    Also visible are the $\rho(770)/\omega(782)$ and $f_{0}(980)$ strong 
    resonances.
  }
  \label{fig:cpv:selection:ppipi_ks_cut}
\end{figure}

\begin{figure}
  \begin{subfigure}[b]{0.5\textwidth}
    \includegraphics[width=\textwidth]{cpv/selection/LcToppipi_2012_MagDown_Lc_p_h1_M}
    \caption{Full spectrum}
    \label{fig:cpv:selection:ppipi_lambda_cut:full}
  \end{subfigure}
  \begin{subfigure}[b]{0.5\textwidth}
    \includegraphics[width=\textwidth]{cpv/selection/LcToppipi_2012_MagDown_Lc_p_h1_M_zoom}
    \caption{Detail}
    \label{fig:cpv:selection:ppipi_lambda_cut:zoom}
  \end{subfigure}
  \caption{%
    Mass spectrum of the $\Pproton\Ppiminus$ pair in the \ppipi\ decay mode, 
    weighted by signal sWeights, in the magnet up 2012 data.
    The spectrum weighted by background sWeights is shown for reference in red.
    The red lines indicate the boundaries of the requirement designed to remove 
    the $\PLambda\Ppiplus$ component of the sample.
    The additional features in the $\Pproton\Ppiminus$ spectrum have not been 
    identified.
  }
  \label{fig:cpv:selection:ppipi_lambda_cut}
\end{figure}

The largest source of combinatorial background is due to $\PK \to \Pproton$ and 
$\Ppi \to \Pproton$ misidentification, as pions and kaons are produced much 
more numerously than protons.
The \ac{PID} requirements on the proton, kaon, and pion candidates are 
tightened offline.
The \probnn\ family of variables is used, which contains one \probnn\ variable 
each charged particle hypothesis~(\Pproton/\APproton, \PKpm, \Ppipm, \Pmupm, 
\Pepm), and is the response of an \ac{ANN} trained to discriminate each 
hypothesis from all others.
Input variables to the network include the \ac{DLL} \ac{PID} variables 
described in \cref{chap:intro:lhcb:detector:pid}, the track \ptot\ and \pT, and 
the total track fit quality and that within each tracking sub-detector.
The \ac{ANN} is trained on a set of truth-matched tracks from a sample of 
\ac{MC} generated under 2012 running conditions, and is scaled such that the 
signal probability is uniform between the extent of the variable, 
$\text{\probnn} \in [0, 1]$.

To define a starting point for the \probnn\ cut values, a simultaneous 
optimisation of all \ac{PID} requirements was performed by maximising the 
signal significance $\sigma$, defined as
\begin{equation}
  \sigma = \frac{%
    \nsig
  }{%
    \sqrt{\nsig + \nbkg}
  },
  \label{eqn:cpv:selection:signal_significance}
\end{equation}
where \nsig\ and \nbkg\ are the signal and background yields in the sample, 
measured using the fits described in \cref{chap:cpv:prelim_fits}.
% TODO why? I used to have a good reason for this
The optimisation showed that the most powerful discriminants between signal and 
background were the proton \ac{PID} information, as expected, and the \ac{PID} 
information of the kaon or pion with the same electric charge as the proton.
Applying a selection optimised for the signal significance would then result in 
the \ac{PID} requirements being asymmetric in the $\Phm\Php$ meson pair.
As it is known that the \ac{PID} efficiency for a given requirement on a given 
particle is dependent on the particle kinematics, as discussed in 
\cref{chap:prod:effs:pid}, and that the detection asymmetries also depend on 
the same kinematics, the same \ac{PID} cut is applied to both mesons in the 
\PLambdac\ final state so as not to induce any additional detection 
asymmetries.
The optimal value found for the cut on the meson with opposite sign to the 
proton was chosen as the value for the cut on the same-sign meson.
The \ac{PID} requirements made offline on the candidate protons, kaons, and 
pions are given in \cref{tab:cpv:selection:pid_cut_values}.

The \phh\ invariant mass distributions after the weak decay vetoes, \ac{PID} 
cuts, and \decaytreefitter\ convergence requirements are given in 
\cref{fig:cpv:selection:postpid} for the 2012 magnet down dataset.

\begin{table}
  \centering
  \caption{%
    Particle identification requirements made on the \PLambdac\ decay products.
  }
  \label{tab:cpv:selection:pid_cut_values}
  \begin{tabular}{ccc}
    \toprule
    Particle           & Variable  & Requirement \\
    \midrule
    \Pproton/\APproton & \probnnp  & $ > 0.3$    \\
    \PKpm              & \probnnk  & $ > 0.3$    \\
    \Ppipm             & \probnnpi & $ > 0.3$    \\
    \bottomrule
  \end{tabular}
\end{table}

\begin{figure}
  \begin{subfigure}[b]{0.5\textwidth}
    \includegraphics[width=\textwidth]{cpv/selection/fits-offline_stripping_trigger_weak_decay-selection_unweighted_no-simultaneous/LcTopKK_2012_MagDown_fit.pdf}
    \caption{\pKK}
    \label{fig:cpv:selection:postpid:pKK}
  \end{subfigure}
  \begin{subfigure}[b]{0.5\textwidth}
    \includegraphics[width=\textwidth]{cpv/selection/fits-offline_stripping_trigger_weak_decay-selection_unweighted_no-simultaneous/LcToppipi_2012_MagDown_fit.pdf}
    \caption{\ppipi}
    \label{fig:cpv:selection:postpid:ppipi}
  \end{subfigure}
  \caption{%
    Fits to the \PLambdac\ mass spectrum in the 2012 magnet down dataset for 
    \pKK\ (\subref*{fig:cpv:data:mass:pKK}) and \ppipi\ 
    (\subref*{fig:cpv:data:mass:ppipi}).
    The stripping, trigger, and \ac{PID} selection is applied, as well as the 
    \decaytreefitter\ convergence requirements, and the weak decay vetoes for 
    the \ppipi\ data.
    The fit that is overlaid is described in \cref{chap:cpv:prelim_fits}.
  }
  \label{fig:cpv:selection:postpid}
\end{figure}

\section{Background study}
\label{chap:cpv:selection:background_study}

Although \ac{PID} requirements are made on the proton candidate, it is possible 
that a true meson decay will be misidentified as a \PLambdac\ decay.
For example, a true \decay{\PDsplus}{\PKminus\PKplus\Ppiplus} decay could be 
misidentified as \LcTopKK\ if the true pion is assigned the proton mass 
hypothesis and passes the \ac{PID} cuts.
The presence of any such misidentification (or `mis-ID') can add structures to 
the \PLambdac\ mass spectrum, making the signal yield extraction more 
challenging.
Such structures may peak near the nominal \PLambdac\ mass and be incorrectly 
counted as \PLambdac\ signal.
Additional backgrounds can arise due to the partial reconstruction of 
\PLambdac\ decays, which would manifest as a `shoulder' at values of the 
\PLambdac\ mass below the signal peak.
As this feature is not seen in the \pKK\ or \ppipi\ mass distributions, shown 
in \cref{fig:cpv:selection:postpid}, it is considered insignificant and is not 
studied further.

One method of checking for the presence of misidentified decays is by 
inspecting three-body mass distributions for different mass hypothesis sets.
For example, by assigning the pion mass to the proton candidate in the \pKK\ 
sample and inspecting the invariant mass of the $\PKplus\PKminus\Ppiplus$ 
combination.
A peak near the \PDsplus mass would indicate a contamination of \PDsplus 
candidates in the data.

% TODO reference
A second way to look for meson backgrounds is to inspect the `momentum 
asymmetry' parameter of the proton
\begin{equation}
  \beta_{\Pproton} = \frac{-p_{\Pproton} + p_{\Php} + p_{\Phm}}{p_{\Pproton} + p_{\Php} + p_{\Phm}},
  \label{eqn:cpv:selection:background_study:mom_asym}
\end{equation}
where $p_{\Pproton}$ is the momentum of the proton candidate and $p_{h^{\pm}}$ 
is the momentum of the $h^{\pm}$ candidate.
Along with the mass of the \PLambdac, this fully parameterises the mass of the 
$\phh$ system under different hypotheses, and so one can infer the presence of 
meson backgrounds by inspecting the $\beta_{\Pproton}$--$m(\phh)$ plane.
As the relationship between $m(\phh)$, $\beta_{\Pproton}$, and the three-body 
mass under the background hypothesis is known, contours of the expected mis-ID 
shape can be drawn on this plane and compared with the data.

Throughout this \namecref{chap:cpv:selection:background_study}, the variables 
shown are not computed using any information from the \decaytreefitter\ 
algorithm described in \cref{chap:cpv:data}.

\subsection{Exchanging mass hypotheses}
\label{chap:cpv:selection:background_study:mass_hypo}

\Cref{fig:cpv:selection:background_study:pKK_meson} shows the three-body mass 
distributions in the 2012 magnet down \pKK\ data for several mass hypothesis 
sets, where each set does not contain the proton mass hypothesis.
\Cref{fig:cpv:selection:background_study:pKK_baryon} shows the same, but where 
each set does contain the proton mass hypothesis.
The two Figures show the charm meson and charm baryon background contributions, 
respectively.
\Cref{fig:cpv:selection:background_study:ppipi_meson,fig:cpv:selection:background_study:ppipi_baryon} 
show the same but for the 2012 magnet down \ppipi\ data.

The distributions from the \pKK\ data show contributions from 
\decay{\PDplus}{\PKplus\PKminus\Ppiplus}, 
\decay{\PDsplus}{\PKplus\PKminus\Ppiplus}, and 
\decay{\PLambdac}{\Pproton\PKminus\Ppiplus} decays.
The \ppipi\ data show \decay{\PDplus}{\PKplus\Ppiminus\Ppiplus}, 
\decay{\PDplus}{\PKplus\PKminus\Ppiplus}, 
\decay{\PDsplus}{\PKplus\Ppiminus\Ppiplus}, and 
\decay{\PDsplus}{\PKplus\PKminus\Ppiplus} contributions.
Each of these backgrounds are removed by vetoing \PLambdac\ candidates whose 
wrong-mass value falls within \SI{8}{\MeV} of the nominal \PDplus, \PDsplus, or 
\PLambdac\ mass, taken to be \SI{1869.62}{\MeV}, \SI{1968.49}{\MeV}, and 
\SI{2286.46}{\MeV}~\cite{PDG2014}.

A double misidentification of \pKK\ is present in the \pKK\ data, where the 
hypotheses of the nominal proton and nominal kaon with the same charge as the 
proton are swapped, visible in 
\cref{fig:cpv:selection:background_study:pKK_baryon:lcp_KKp}.
This is not removed from the data, as the distribution of candidates that fall 
in the signal region of the swapped hypothesis ($\PKplus\PKminus\Pproton$) have 
no irregular structures under the nominal hypothesis (\pKK), as shown in 
\cref{fig:cpv:selection:background_study:pKK_lcp_KKp_Lc_M}.

\subsection{Momentum asymmetry}
\label{chap:cpv:selection:background_study:mom_asym}

The momentum asymmetry of the proton is defined in 
\cref{eqn:cpv:selection:background_study:mom_asym}.
Along with the mass of the \PLambdac\ under the nominal three-body mass 
hypothesis, it fully parameterises alternative three-body hypotheses where the 
hypothesis of the proton is changed.
The $\beta_{\Pproton}$--$m(\PLambdac)$ plane then provides an additional handle 
on misidentification backgrounds, in addition to the technique described in 
\cref{chap:cpv:selection:background_study:mass_hypo}.

\Cref{fig:cpv:selection:background_study:mom_asym:pKK,fig:cpv:selection:background_study:mom_asym:ppipi} 
show the $\beta_{\Pproton}$--$m(\PLambdac)$ plane with 2012 magnet down data.
There are distinct contributions from \decay{\PDplus}{\PKplus\PKminus\Ppiplus} 
and \decay{\PDsplus}{\PKplus\PKminus\Ppiplus} in the \pKK\ data, and from 
\decay{\PDsplus}{\PKplus\Ppiminus\Ppiplus} in the \ppipi\ data.
Also shown are the $\beta_{\Pproton}$--$m(\PLambdac)$ planes after the 
misidentification vetoes described in 
\cref{chap:cpv:selection:background_study:mass_hypo} have been applied.
No contributions beyond those given in 
\cref{chap:cpv:selection:background_study:mass_hypo} are seen.

\begin{figure}
  \begin{subfigure}[b]{0.3\textwidth}
    \includegraphics[width=\textwidth]{cpv/selection/background_study/pKK/LcTopKK_2012_MagDown_Dp_ppTokp_km_kpTopip}
    \caption{\decay{\PDplus}{\PKplus\PKminus\Ppiplus}}
    \label{fig:cpv:selection:background_study:pKK_meson:dplus_kkpi}
  \end{subfigure}
  \begin{subfigure}[b]{0.3\textwidth}
    \includegraphics[width=\textwidth]{cpv/selection/background_study/pKK/LcTopKK_2012_MagDown_Dp_ppTopip_kmTopim_kpTopip}
    \caption{\decay{\PDplus}{\Ppiplus\Ppiminus\Ppiplus}}
    \label{fig:cpv:selection:background_study:pKK_meson:dplus_pipipi}
  \end{subfigure}
  \begin{subfigure}[b]{0.3\textwidth}
    \includegraphics[width=\textwidth]{cpv/selection/background_study/pKK/LcTopKK_2012_MagDown_Dp_ppTopip_km_kp}
    \caption{\decay{\PDplus}{\Ppiplus\PKminus\PKplus}}
    \label{fig:cpv:selection:background_study:pKK_meson:dplus_pikk}
  \end{subfigure}

  \begin{subfigure}[b]{0.3\textwidth}
    \includegraphics[width=\textwidth]{cpv/selection/background_study/pKK/LcTopKK_2012_MagDown_Dp_ppTopip_km_kpTopip}
    \caption{\decay{\PDplus}{\Ppiplus\PKminus\Ppiplus}}
    \label{fig:cpv:selection:background_study:pKK_meson:dplus_pikpi}
  \end{subfigure}
  \begin{subfigure}[b]{0.3\textwidth}
    \includegraphics[width=\textwidth]{cpv/selection/background_study/pKK/LcTopKK_2012_MagDown_Ds_ppTokp_kmTopim_kpTopip}
    \caption{\decay{\PDsplus}{\PKplus\Ppiminus\Ppiplus}}
    \label{fig:cpv:selection:background_study:pKK_meson:dsplus_Kpipi}
  \end{subfigure}
  \begin{subfigure}[b]{0.3\textwidth}
    \includegraphics[width=\textwidth]{cpv/selection/background_study/pKK/LcTopKK_2012_MagDown_Ds_ppTokp_km_kp}
    \caption{\decay{\PDsplus}{\PKplus\PKminus\PKplus}}
    \label{fig:cpv:selection:background_study:pKK_meson:dsplus_KKK}
  \end{subfigure}

  \begin{subfigure}[b]{0.3\textwidth}
    \includegraphics[width=\textwidth]{cpv/selection/background_study/pKK/LcTopKK_2012_MagDown_Ds_ppTopip_kmTopim_kpTopip}
    \caption{\decay{\PDsplus}{\Ppiplus\Ppiminus\Ppiplus}}
    \label{fig:cpv:selection:background_study:pKK_meson:dsplus_pipipi}
  \end{subfigure}
  \begin{subfigure}[b]{0.3\textwidth}
    \includegraphics[width=\textwidth]{cpv/selection/background_study/pKK/LcTopKK_2012_MagDown_Ds_ppTopip_km_kp}
    \caption{\decay{\PDsplus}{\Ppiplus\PKminus\PKplus}}
    \label{fig:cpv:selection:background_study:pKK_meson:dsplus_piKK}
  \end{subfigure}

  \caption{%
    Wrong-mass distributions obtained when changing the mass hypotheses of the 
    fully selected \PLambdac\ children in the \pKK\ mode, where no child is 
    assigned the proton mass hypothesis.
    The $x$-axis on each sub-figure shows the substitutions that have been 
    made.
    For example, the ${(\Pproton \to \Ppiplus)(\PKminus \to \Ppiminus)\PKplus}$ distribution is 
    where the nominal proton candidate has been assigned the pion mass 
    hypothesis, and the nominal kaon with opposite charge to the proton has 
    been also assigned the pion hypothesis.
    Only the 2012 magnet down data is shown.
  }
  \label{fig:cpv:selection:background_study:pKK_meson}
\end{figure}


\begin{figure}
  \begin{subfigure}[b]{0.3\textwidth}
    \includegraphics[width=\textwidth]{cpv/selection/background_study/pKK/LcTopKK_2012_MagDown_Lc_ppTokp_kmTopim_kpTopp}
    \caption{\decay{\PLambdac}{\PKplus\Ppiminus\Pproton}}
    \label{fig:cpv:selection:background_study:pKK_baryon:lcp_Kpip}
  \end{subfigure}
  \begin{subfigure}[b]{0.3\textwidth}
    \includegraphics[width=\textwidth]{cpv/selection/background_study/pKK/LcTopKK_2012_MagDown_Lc_ppTokp_km_kpTopp}
    \caption{\decay{\PLambdac}{\PKplus\PKminus\Pproton}}
    \label{fig:cpv:selection:background_study:pKK_baryon:lcp_KKp}
  \end{subfigure}
  \begin{subfigure}[b]{0.3\textwidth}
    \includegraphics[width=\textwidth]{cpv/selection/background_study/pKK/LcTopKK_2012_MagDown_Lc_ppTopip_kmTopim_kpTopp}
    \caption{\decay{\PLambdac}{\Ppiplus\Ppiminus\Pproton}}
    \label{fig:cpv:selection:background_study:pKK_baryon:lcp_pipip}
  \end{subfigure}

  \begin{subfigure}[b]{0.3\textwidth}
    \includegraphics[width=\textwidth]{cpv/selection/background_study/pKK/LcTopKK_2012_MagDown_Lc_ppTopip_km_kpTopp}
    \caption{\decay{\PLambdac}{\Ppiplus\PKminus\Pproton}}
    \label{fig:cpv:selection:background_study:pKK_baryon:lcp_pikp}
  \end{subfigure}
  \begin{subfigure}[b]{0.3\textwidth}
    \includegraphics[width=\textwidth]{cpv/selection/background_study/pKK/LcTopKK_2012_MagDown_Lc_pp_kmTopim_kp}
    \caption{\decay{\PLambdac}{\Pproton\Ppiminus\PKplus}}
    \label{fig:cpv:selection:background_study:pKK_baryon:lcp_ppik}
  \end{subfigure}
  \begin{subfigure}[b]{0.3\textwidth}
    \includegraphics[width=\textwidth]{cpv/selection/background_study/pKK/LcTopKK_2012_MagDown_Lc_pp_kmTopim_kpTopip}
    \caption{\decay{\PLambdac}{\Pproton\Ppiminus\Ppiplus}}
    \label{fig:cpv:selection:background_study:pKK_baryon:lcp_ppipi}
  \end{subfigure}

  \begin{subfigure}[b]{0.3\textwidth}
    \includegraphics[width=\textwidth]{cpv/selection/background_study/pKK/LcTopKK_2012_MagDown_Lc_pp_km_kpTopip}
    \caption{\decay{\PLambdac}{\Pproton\PKminus\Ppiplus}}
    \label{fig:cpv:selection:background_study:pKK_baryon:lcp_pkpi}
  \end{subfigure}

  \caption{%
    Wrong-mass distributions obtained when changing the mass hypotheses of the 
    fully selected \PLambdac\ children in the \pKK\ mode, where one child is 
    assigned the proton mass hypothesis.
    The $x$-axis on each sub-figure shows the substitutions that have been 
    made.
    For example, the ${\Pproton\PKminus(\PKplus \to \Ppiplus)}$ distribution is where the 
    nominal kaon with same charge as the proton has been also assigned the pion 
    hypothesis.
    Only the 2012 magnet down data is shown.
  }
  \label{fig:cpv:selection:background_study:pKK_baryon}
\end{figure}

\begin{figure}
  \centering
  \includegraphics[width=0.5\textwidth]{cpv/selection/background_study/pKK/LcTopKK_2012_MagDown_Lc_ppTokp_km_kpTopp-Lc_M}
  \caption{%
    Mass spectrum of \pKK\ candidates that fall within \SI{8}{\MeV} of the 
    nominal \PLambdac\ mass when the proton and same-sign kaon mass hypotheses 
    are swapped.
    The mass spectrum of \PLambdac\ candidates with the swapped hypotheses is 
    shown in 
    Figure~\ref{fig:cpv:selection:background_study:pKK_baryon:lcp_KKp}.
    The lack of structures other than the flat combinatorial-background-like 
    and peaking signal-like components indicate that the presence of this 
    misidentification background does not affect the nominal \pKK\ yield 
    extraction.
  }
  \label{fig:cpv:selection:background_study:pKK_lcp_KKp_Lc_M}
\end{figure}

\begin{figure}
  \begin{subfigure}[b]{0.3\textwidth}
    \includegraphics[width=\textwidth]{figures/cpv/selection/background_study/ppipi/LcToppipi_2012_MagDown_Dp_ppTokp_pimTokm_pip}
    \caption{\decay{\PDplus}{\PKplus\PKminus\Ppiplus}}
    \label{fig:cpv:selection:background_study:ppipi_meson:dplus_kkpi}
  \end{subfigure}
  \begin{subfigure}[b]{0.3\textwidth}
    \includegraphics[width=\textwidth]{figures/cpv/selection/background_study/ppipi/LcToppipi_2012_MagDown_Dp_ppTopip_pimTokm_pip}
    \caption{\decay{\PDplus}{\Ppiplus\PKminus\Ppiplus}}
    \label{fig:cpv:selection:background_study:ppipi_meson:dplus_pikpi}
  \end{subfigure}
  \begin{subfigure}[b]{0.3\textwidth}
    \includegraphics[width=\textwidth]{figures/cpv/selection/background_study/ppipi/LcToppipi_2012_MagDown_Dp_ppTopip_pimTokm_pipTokp}
    \caption{\decay{\PDplus}{\Ppiplus\PKminus\PKplus}}
    \label{fig:cpv:selection:background_study:ppipi_meson:dplus_pikk}
  \end{subfigure}
  \begin{subfigure}[b]{0.3\textwidth}
    \includegraphics[width=\textwidth]{figures/cpv/selection/background_study/ppipi/LcToppipi_2012_MagDown_Dp_ppTopip_pim_pip}
    \caption{\decay{\PDplus}{\Ppiplus\Ppiminus\Ppiplus}}
    \label{fig:cpv:selection:background_study:ppipi_meson:dplus_pipipi}
  \end{subfigure}
  \begin{subfigure}[b]{0.3\textwidth}
    \includegraphics[width=\textwidth]{figures/cpv/selection/background_study/ppipi/LcToppipi_2012_MagDown_Ds_ppTokp_pimTokm_pip}
    \caption{\decay{\PDsplus}{\PKplus\PKminus\Ppiplus}}
    \label{fig:cpv:selection:background_study:ppipi_meson:dsplus_kkpi}
  \end{subfigure}
  \begin{subfigure}[b]{0.3\textwidth}
    \includegraphics[width=\textwidth]{figures/cpv/selection/background_study/ppipi/LcToppipi_2012_MagDown_Ds_ppTokp_pimTokm_pipTokp}
    \caption{\decay{\PDsplus}{\PKplus\PKminus\PKplus}}
    \label{fig:cpv:selection:background_study:ppipi_meson:dsplus_kkk}
  \end{subfigure}
  \begin{subfigure}[b]{0.3\textwidth}
    \includegraphics[width=\textwidth]{figures/cpv/selection/background_study/ppipi/LcToppipi_2012_MagDown_Ds_ppTokp_pim_pip}
    \caption{\decay{\PDsplus}{\PKplus\Ppiminus\Ppiplus}}
    \label{fig:cpv:selection:background_study:ppipi_meson:dsplus_kpipi}
  \end{subfigure}
  \begin{subfigure}[b]{0.3\textwidth}
    \includegraphics[width=\textwidth]{figures/cpv/selection/background_study/ppipi/LcToppipi_2012_MagDown_Ds_ppTopip_pimTokm_pipTokp}
    \caption{\decay{\PDsplus}{\Ppiplus\PKminus\PKplus}}
    \label{fig:cpv:selection:background_study:ppipi_meson:dsplus_pikk}
  \end{subfigure}
  \begin{subfigure}[b]{0.3\textwidth}
    \includegraphics[width=\textwidth]{figures/cpv/selection/background_study/ppipi/LcToppipi_2012_MagDown_Ds_ppTopip_pim_pip}
    \caption{\decay{\PDsplus}{\Ppiplus\Ppiminus\Ppiplus}}
    \label{fig:cpv:selection:background_study:ppipi_meson:dsplus_pipipi}
  \end{subfigure}
  \begin{subfigure}[b]{0.3\textwidth}
    \includegraphics[width=\textwidth]{figures/cpv/selection/background_study/ppipi/LcToppipi_2012_MagDown_Ds_ppTopip_pim_pipTokp}
    \caption{\decay{\PDsplus}{\Ppiplus\Ppiminus\PKplus}}
    \label{fig:cpv:selection:background_study:ppipi_meson:dsplus_pipik}
  \end{subfigure}
  \caption{%
    Wrong-mass distributions obtained when changing the mass hypotheses of the 
    fully selected \PLambdac\ children in the \ppipi\ mode, where no child is 
    assigned the proton mass hypothesis.
    The $x$-axis on each sub-figure shows the substitutions that have been 
    made.
    For example, the ${(\Pproton \to \Ppiplus)(\Ppiminus \to \PKminus)\Ppiplus}$ distribution is 
    where the nominal proton candidate has been assigned the pion mass 
    hypothesis, and the nominal pion with opposite charge to the proton has 
    been assigned the kaon hypothesis.
    Only the 2012 magnet down data is shown.
  }
  \label{fig:cpv:selection:background_study:ppipi_meson}
\end{figure}

\begin{figure}
  \begin{subfigure}[b]{0.3\textwidth}
    \includegraphics[width=\textwidth]{figures/cpv/selection/background_study/ppipi/LcToppipi_2012_MagDown_Lc_ppTokp_pimTokm_pipTopp}
    \caption{\decay{\PLambdac}{\PKplus\PKminus\Pproton}}
    \label{fig:cpv:selection:background_study:ppipi_baryon:kkp}
  \end{subfigure}
  \begin{subfigure}[b]{0.3\textwidth}
    \includegraphics[width=\textwidth]{figures/cpv/selection/background_study/ppipi/LcToppipi_2012_MagDown_Lc_ppTokp_pim_pipTopp}
    \caption{\decay{\PLambdac}{\PKplus\Ppiminus\Pproton}}
    \label{fig:cpv:selection:background_study:ppipi_baryon:kpip}
  \end{subfigure}
  \begin{subfigure}[b]{0.3\textwidth}
    \includegraphics[width=\textwidth]{figures/cpv/selection/background_study/ppipi/LcToppipi_2012_MagDown_Lc_ppTopip_pimTokm_pipTopp}
    \caption{\decay{\PLambdac}{\Ppiplus\PKminus\Pproton}}
    \label{fig:cpv:selection:background_study:ppipi_baryon:pikp}
  \end{subfigure}
  \begin{subfigure}[b]{0.3\textwidth}
    \includegraphics[width=\textwidth]{figures/cpv/selection/background_study/ppipi/LcToppipi_2012_MagDown_Lc_ppTopip_pim_pipTopp}
    \caption{\decay{\PLambdac}{\Ppiplus\Ppiminus\Pproton}}
    \label{fig:cpv:selection:background_study:ppipi_baryon:pipip}
  \end{subfigure}
  \begin{subfigure}[b]{0.3\textwidth}
    \includegraphics[width=\textwidth]{figures/cpv/selection/background_study/ppipi/LcToppipi_2012_MagDown_Lc_pp_pimTokm_pip}
    \caption{\decay{\PLambdac}{\Pproton\PKminus\Ppiplus}}
    \label{fig:cpv:selection:background_study:ppipi_baryon:pkpi}
  \end{subfigure}
  \begin{subfigure}[b]{0.3\textwidth}
    \includegraphics[width=\textwidth]{figures/cpv/selection/background_study/ppipi/LcToppipi_2012_MagDown_Lc_pp_pimTokm_pipTokp}
    \caption{\decay{\PLambdac}{\Pproton\PKminus\PKplus}}
    \label{fig:cpv:selection:background_study:ppipi_baryon:pkk}
  \end{subfigure}
  \begin{subfigure}[b]{0.3\textwidth}
    \includegraphics[width=\textwidth]{figures/cpv/selection/background_study/ppipi/LcToppipi_2012_MagDown_Lc_pp_pim_pipTokp}
    \caption{\decay{\PLambdac}{\Pproton\Ppiminus\PKplus}}
    \label{fig:cpv:selection:background_study:ppipi_baryon:ppik}
  \end{subfigure}
  \caption{%
    Wrong-mass distributions obtained when changing the mass hypotheses of the 
    fully selected \PLambdac\ children in the \ppipi\ mode, where one child is 
    assigned the proton mass hypothesis.
    The $x$-axis on each sub-figure shows the substitutions that have been 
    made.
    For example, the ${\Pproton(\Ppiminus \to \PKminus)\Ppiplus}$ distribution is where the 
    nominal proton candidate has been assigned the pion mass hypothesis, and 
    the nominal pion with opposite charge to the proton has been also assigned 
    the kaon hypothesis.
    Only the 2012 magnet down data is shown.
  }
  \label{fig:cpv:selection:background_study:ppipi_baryon}
\end{figure}

\begin{figure}
  \begin{subfigure}[b]{0.5\textwidth}
    \includegraphics[width=\textwidth]{cpv/selection/background_study/pKK/LcTopKK_2012_MagDown_beta_p-Lb_DTF_Lc_M}
    \caption{Before mis-ID vetoes}
    \label{fig:cpv:selection:background_study:mom_asym:pKK:before}
  \end{subfigure}
  \begin{subfigure}[b]{0.5\textwidth}
    \includegraphics[width=\textwidth]{cpv/selection/background_study/pKK/LcTopKK_2012_MagDown_beta_p-Lb_DTF_Lc_M_contours}
    \caption{With theoretical contours}
    \label{fig:cpv:selection:background_study:mom_asym:pKK:contours}
  \end{subfigure}
  \begin{subfigure}[b]{0.5\textwidth}
    \includegraphics[width=\textwidth]{cpv/selection/background_study/pKK/LcTopKK_2012_MagDown_beta_p-Lb_DTF_Lc_M-vetoes}
    \caption{After mis-ID vetoes}
    \label{fig:cpv:selection:background_study:mom_asym:pKK:after}
  \end{subfigure}
  \caption{%
    Plane of $\beta_{\Pproton}$--$m(\PLambdac)$ for the \pKK\ mode in the 2012 
    magnet down dataset.
    The ``theoretical contours'' shown on 
    \cref{fig:cpv:selection:background_study:mom_asym:pKK:contours} are the 
    solution of the parametric equation that defines the relationship between 
    $m(\PLambdac)$, $\beta_{\Pproton}$, and the mass of the misidentified 
    particle decaying to a particular final state.
    The plane after the misidentification vetoes, defined in 
    \cref{chap:cpv:selection:background_study:mass_hypo}, is shown in 
    \cref{fig:cpv:selection:background_study:mom_asym:pKK:after}.
  }
  \label{fig:cpv:selection:background_study:mom_asym:pKK}
\end{figure}

\begin{figure}
  \begin{subfigure}[b]{0.5\textwidth}
    \includegraphics[width=\textwidth]{cpv/selection/background_study/ppipi/LcToppipi_2012_MagDown_beta_p-Lb_DTF_Lc_M}
    \caption{Before mis-ID vetoes}
    \label{fig:cpv:selection:background_study:mom_asym:ppipi:before}
  \end{subfigure}
  \begin{subfigure}[b]{0.5\textwidth}
    \includegraphics[width=\textwidth]{cpv/selection/background_study/ppipi/LcToppipi_2012_MagDown_beta_p-Lb_DTF_Lc_M_contours}
    \caption{With theoretical contours}
    \label{fig:cpv:selection:background_study:mom_asym:ppipi:contours}
  \end{subfigure}
  \begin{subfigure}[b]{0.5\textwidth}
    \includegraphics[width=\textwidth]{cpv/selection/background_study/ppipi/LcToppipi_2012_MagDown_beta_p-Lb_DTF_Lc_M-vetoes}
    \caption{After mis-ID vetoes}
    \label{fig:cpv:selection:background_study:mom_asym:ppipi:after}
  \end{subfigure}
  \caption{%
    Plane of $\beta_{\Pproton}$--$m(\PLambdac)$ for the \ppipi\ mode in the 
    2012 magnet down dataset.
    The ``theoretical contours'' shown on 
    \cref{fig:cpv:selection:background_study:mom_asym:ppipi:contours} are the 
    solution of the parametric equation that defines the relationship between 
    $m(\PLambdac)$, $\beta_{\Pproton}$, and the mass of the misidentified 
    particle decaying to a particular final state.
    The plane after the misidentification vetoes, defined in 
    \cref{chap:cpv:selection:background_study:mass_hypo}, is shown in 
    \cref{fig:cpv:selection:background_study:mom_asym:ppipi:after}.
  }
  \label{fig:cpv:selection:background_study:mom_asym:ppipi}
\end{figure}

\section{Multiple candidates}
\label{chap:cpv:selection:multiple_candidates}

It is possible that two or more true \LcTophh\ decays are reconstructed in a 
single \ac{LHC} bunch crossing, but it is unlikely.
It is more likely that such multiple candidates are caused by unphysical means, 
such as cloned tracks~\cite{LHCb-INT-2011-009}.
In the case an event has multiple \PLambdac\ candidates, one is selected at 
random for further analysis and the rest are discarded.
The fraction of events containing multiple \PLambdac\ candidates in given in 
\cref{tab:cpv:selection:multiple_candidates:pKK,tab:cpv:selection:multiple_candidates:ppipi}.

\begin{table}
  \caption{%
    Statistics on multiple candidates in the \pKK\ data.
    The total number of events $N_{\text{Events}}$ is the number of 
    proton-proton collisions that make up the data sub-sample, and 
    $N_{\text{Candidates}}$ is the number of \PLambdac\ candidates.
    The fraction of events containing at least two candidates is 
    $f_{\text{Multiple}}$, and the fraction of those events that fall within 
    the signal window in the \PLambdac\ mass distribution is 
    $f_{\text{Multiple, sig.}}$.
  }
  \label{tab:cpv:selection:multiple_candidates:pKK}
  \begin{tabular}{ccccccc}
  \toprule
  Year & Polarity & $N_{\text{Events}}$ & $N_{\text{Candidates}}$ & $f_{\text{Multiple}}$ [\si{\percent}] & $f_{\text{Multiple, sig.}}$ [\si{\percent}] \\
  \midrule
2011   & Down     & $11600 \pm 100$     & $11800 \pm 100$         & $1.39 \pm 0.11$                       & $0.16 \pm 0.06$                             \\
2011   & Up       & $8300 \pm 100$      & $8400 \pm 100$          & $1.62 \pm 0.14$                       & $0.39 \pm 0.10$                             \\
2012   & Down     & $26900 \pm 200$     & $27300 \pm 200$         & $1.51 \pm 0.08$                       & $0.23 \pm 0.04$                             \\
2012   & Up       & $26900 \pm 200$     & $27400 \pm 200$         & $1.65 \pm 0.08$                       & $0.28 \pm 0.05$                             \\
  \bottomrule
\end{tabular}

\end{table}

\begin{table}
  \caption{%
    Multiple candidates for the \ppipi\ data.
    The total number of events $N_{\text{Events}}$ is the number of 
    proton-proton collisions that make up the data sub-sample, and 
    $N_{\text{Candidates}}$ is the number of \PLambdac\ candidates.
    The fraction of events containing at least two candidates is 
    $f_{\text{Multiple}}$, and the fraction of those events that fall within 
    the signal window in the \PLambdac\ mass distribution is 
    $f_{\text{Multiple, sig.}}$.
  }
  \label{tab:cpv:selection:multiple_candidates:ppipi}
  \begin{tabular}{ccccccc}
  \toprule
  Year & Polarity & $N_{\text{Events}}$ & $N_{\text{Candidates}}$ & $f_{\text{Multiple}}$ [\si{\percent}] & $f_{\text{Multiple, sig.}}$ [\si{\percent}] \\
  \midrule
2011   & Down     & $64700 \pm 300$     & $65800 \pm 300$         & $1.62 \pm 0.05$                       & $0.23 \pm 0.03$                             \\
2011   & Up       & $46800 \pm 200$     & $47600 \pm 200$         & $1.80 \pm 0.06$                       & $0.32 \pm 0.04$                             \\
2012   & Down     & $163100 \pm 400$    & $166100 \pm 400$        & $1.75 \pm 0.03$                       & $0.29 \pm 0.02$                             \\
2012   & Up       & $161900 \pm 400$    & $165000 \pm 400$        & $1.80 \pm 0.03$                       & $0.32 \pm 0.02$                             \\
  \bottomrule
\end{tabular}

\end{table}
