\chapter{Selection}
\label{chap:cpv:selection}

The \lhcb\ experiment is optimised for efficiently selecting heavy flavour 
hadron decays and rejecting the relatively huge levels of combinatorial 
background.
The \PLambdac\ charm baryon has a lifetime of \SI{0.2}{\pico\second}, half that 
of the \PDzero\ and eight times less than that of the \PBzero.
Indeed, it is one of the shortest-lived weakly decaying heavy flavour hadrons, 
and as such presents particular challenges when a clean, efficient selection is 
desired, as in this analysis, as the secondary decay vertex is not so 
significantly displaced from the \ac{PV}.
Although the \bbbar\ cross-section is of the order of 100 times less than the 
\ccbar\ cross-section, reconstructing the decay chain \LbToLcmuX\ can allow for 
the exploitation of the excellent secondary vertex resolution and high \PB 
trigger efficiency provided by the \lhcb\ detector.
This analysis takes that approach, selecting \emph{secondary} \PLambdac\ 
candidates by association with a muon.
This \namecref{chap:cpv:selection} shall describe each stage of the selection.

The selection of the data described in \cref{chap:cpv:data} proceeds in two 
steps: firstly the stripping is run, and then an `offline selection' is made on 
the candidates passing the stripping.
All events passing the trigger are available to the stripping, and so the 
trigger `requirements' can be made at any stage.
In this analysis, these are made in the stripping lines themselves.
These two steps shall be described in turn.

\section{Stripping and trigger}
\label{chap:cpv:selection:stripping_trigger}

The selections in the stripping lines were not tuned specifically for this 
analysis, instead serving as general-purpose requirements suitable for a range 
of studies that use \LcTophh\ decays originating from semileptonic \PLambdab\ 
decays.
As such, they are similar to the selections of other \PB decays made at \lhcb\ 
in that they exploit the long decay times of \PB hadrons, such that they fly as 
significant, measurable distance in the \velo, and the high \pT of the child 
vertices and tracks.
The additional requirements are similar in nature to those made for charm 
hadrons such as track and vertex fit quality and track \ipchisq, which are 
described in \cref{chap:prod:sel}.

The selection of final state particles in the stripping lines, that is protons, 
charged kaons and pions, and muons, is given in 
\cref{tab:cpv:selection:stripping_basic}.
One variable given has not been discussed previously is \pghost, the 
\emph{ghost probability}, which is the response of an \ac{ANN} trained to 
discriminate between real and fake (ghost) tracks~\cite{Brehmer:1478372}.
The \ac{ANN} response is transformed in such a way as to be within the range 
$[0, 1]$, and that it is linearly efficient at rejecting ghost tracks in 
simulated data in that range.
The requirement $\pghost < 0.5$ made on real data is made with the intent to 
reject \SI{50}{\percent} of the ghost tracks in the sample.

The selected final state particles are combined into a \LcTophh\ candidates, 
which is then combined with muon candidates to form candidate \LbToLcmuX\ 
decays.
The requirements on the combinations and vertices are given in 
\cref{tab:cpv:selection:stripping_composite}.
As in \cref{chap:prod:sel}, the selection of composite particles proceeds in 
two stages: firstly by imposing requirements on the union of the inputs; and 
secondly on the vertex formed after the fit.

Trigger requirements are made after the \PLambdab\ candidate is created and 
selected in the stripping lines.
They are driven by the powerful discrimination provided by the muon from the 
\PLambdab which, for a signal particle, will have a high \pT\ and \ipchisq.
At \lzero\ the muon candidate is required to have fired the muon trigger, which 
itself requires the presence of a track in the muon stations with $\pT > 
\SI{1.8}{\GeVc}$, and likewise at \hltone, which requires the \lzero muon 
candidate to be reconstructed as a good-quality track.
At \hlttwo, the tracks comprising the \PLambdab\ candidate are required to have 
fired the topological semileptonic \PB lines, which were briefly described in 
\cref{chap:cpv:data}.
These topological lines cut on the output of a \ac{BDT} trained to discriminate 
sets of 2, 3, and 4 charged particle tracks originating in the decays of \PB 
hadrons to those randomly selected from the event.
The inputs to the \ac{BDT} include the \emph{corrected mass}, which accounts 
for missing particles in the candidate \PB decay and peaks near the \PB mass 
for signal, the sum of the \pT\ of the input objects, the minimum \pT\ amongst 
those objects, and the \PB candidate flight distance \chisq.
The topological trigger lines are typically 80--\SI{90}{\percent} efficient in 
accepting signal \PB decays, with a high background rejection 
rate~\cite{Gligorov:2011qxa}.
Distributions of the \pKK\ and \ppipi\ mass in the 2012 magnet down dataset 
after the stripping and trigger requirements are given in 
\cref{fig:cpv:data:mass}, as an illustration of the signal purity at this 
stage.

% As this table is so close to the beginning of the section, try to position it 
% at the bottom of the page so it doesn't come before the section header
\begin{table}[bp]
  \centering
  \caption{%
    Selection of tracks, with associated particle hypotheses, used in the 
    \PLambdab\ and \PLambdac\ stripping selection.
    Cuts listed under ``All'' are applied to all tracks.
  }
  \label{tab:cpv:selection:stripping_basic}
  \begin{tabular}{crl}
  Particle                 & Variable             & Cut value          \\
  \midrule
  \multirow{4}{*}{All}     & \pT                  & $> \SI{250}{\MeV}$ \\
                           & \ptot                & $> \SI{2}{\GeV}$   \\
                           & Track $\chisq/\ndof$ & $< 3$              \\
                           & \pghost              & $< 0.5$            \\
  \midrule
  \multirow{4}{*}{Muons}   & \pT                  & $> \SI{800}{\MeV}$ \\
                           & \ptot                & $> \SI{3}{\GeV}$   \\
                           & \ipchisq             & $> 4$              \\
                           & \dllmupi             & $> 0$              \\
  \midrule
  \multirow{2}{*}{Pions}   & \ipchisq             & $> 9$              \\
                           & \dllkpi              & $< 4$              \\
  \midrule
  \multirow{2}{*}{Kaons}   & \ipchisq             & $> 9$              \\
                           & \dllkpi              & $> 4$              \\
  \midrule
  \multirow{3}{*}{Protons} & \ipchisq             & $> 9$              \\
                           & \dllppi              & $> 4$              \\
                           & \dllpk               & $> \num{1e-10}$    \\
  \bottomrule
\end{tabular}

\end{table}

\begin{table}
  \centering
  \caption{%
    Stripping selection of particle combinations, vertices, and events.
  }
  \label{tab:cpv:selection:stripping_composite}
  \begin{tabular}{crl}
  Object                             & Variable                          & Cut value                    \\
  \midrule
  \multirow{2}{*}{Event}             & $N_{\text{Long tracks}}$          & $< 250$                      \\
                                     & $N_{\text{PV}}$                   & $> 0$                        \\
  \midrule
  \multirow{3}{*}{\phh}              & $|m - m_{\text{PDG}}(\PLambdac)|$ & $< \SI{100}{\MeV}$           \\
                                     & $\sum_{\text{Children}}\pT$       & $> \SI{1800}{\MeV}$          \\
                                     & DOCA \chisq                       & $< 20$                       \\
  \midrule
  \multirow{4}{*}{\PLambdac\ vertex} & $|m - m_{\text{PDG}}(\PLambdac)|$ & $< \SI{80}{\MeV}$            \\
                                     & $\sum_{\text{Children}}\pT$       & $> \SI{1800}{\MeV}$          \\
                                     & Vertex $\chisq/\ndof$             & $< 6$                        \\
                                     & Vertex distance \chisq            & $> 100$                      \\
  \midrule
  $\PLambdac\Pmuon$                  & $m$                               & $< \SI{6200}{\MeV}$          \\
  \midrule
  \multirow{4}{*}{\PLambdab\ vertex} & $m$                               & $2500 < m < \SI{6000}{\MeV}$ \\
                                     & Vertex $\chisq/\ndof$             & $< 6$                        \\
                                     & DIRA                              & $> 0.999$                    \\
                                     & $z_{\PLambdac} - z_{\PLambdab}$   & $> \SI{0}{\milli\metre}$     \\
  \bottomrule
\end{tabular}

\end{table}

\section{Offline}
\label{chap:cpv:selection:offline}
