\chapter{Kinematic weighting}
\label{chap:cpv:kinematic_weighting}

For the assumptions in \cref{eqn:cpv:theory:dacp} to hold, the
kinematics of the particles contributing to the background asymmetries must be
equal between the \pKK\ and \ppipi\ final states.
These are the kinematics of the \PLambdab, the muon from the \PLambdab, and the
proton from the \PLambdac.
Within a particular \phh\ final state, it must also be the case that the \Php\
and \Phm\ kinematics are equal in order that $\ADf(f)$ only contains
contributions from the proton detection asymmetry $\AD(\Pproton)$.
A comparison of these distributions between the 2012 magnet down \pKK\ and
\ppipi\ datasets is given in
\cref{fig:cpv:kinematic_weighting:pre:Lb,fig:cpv:kinematic_weighting:pre:Lb_mu,fig:cpv:kinematic_weighting:pre:Lc,fig:cpv:kinematic_weighting:pre:Lc_p,fig:cpv:kinematic_weighting:pre:pKK_h1h2,fig:cpv:kinematic_weighting:pre:ppipi_h1h2}.
The signal distributions are shown, where the contents of each bin is defined
as the sum of signal sWeights within that bin, as computed from the result of
the mass fit to the charge-combined samples described in
\cref{chap:cpv:prelim_fits}.
The correlation of the variables with the \PLambdac\ mass is seen to be small,
with a correlation coefficient less than \SI{2}{\percent} for all variables.
Only the muon kinematics across the modes and the \hmhp\ kinematics within the
modes agree, and so a weighting is necessary.

A \acf{BDT} is used to equalise the relevant particle kinematics between the
two modes.
This is an uncommon technique in physics analyses, with ratios of
multi-dimensional histograms being used more often, and so it will be briefly
summarised in the following.
(An introduction to decisions trees and boosting is presented in
\cref{chap:decision_trees}.)
The creation of the specific models using for weighting will then be presented,
followed by a discussion on weighting validation.

\section{Weighting with decision trees}
\label{chap:cpv:kinematic_weighting:bdt_method}

It is common in \acl{HEP} to want to be able to bring two distributions into
agreement, as is the case in this analysis where particle kinematics should be
equal between the \pKK\ and \ppipi\ data.
Call one possibly multi-dimensional distribution the \emph{target} $f_{t}$ and
the other the \emph{original} distribution $f_{o}$.
To transform $f_{o}$ to have the same form as $f_{t}$, there is some
transformation function $W$
\begin{equation}
  f_{t}(x) = W(x)f_{o}(x),
\end{equation}
where $x$ is a possibly multi-dimensional parameter.
As the true distributions of $f_{t}$ and $f_{o}$ are not known in the data,
they are approximated as histograms, partitioning the data in $x$ and counting
the number of events that fall in each bin.
The transformation function is quantised as $w$ in bins, where in the $i$th bin
\begin{equation}
  f_{t}(x_{i}) = w(x_{i})f_{o}(x_{i}),
\end{equation}
and where $f_{t(o)}(x_{i})$ is the count of the target (original) distribution
in the $i$th bin.
The histogram of $f_{o}$ can be transformed to that of $f_{t}$ by computing the
per-bin \emph{weights}
\begin{equation}
  w(x_{i}) = \frac{f_{t}(x_{i})}{f_{o}(x_{i})}.
\end{equation}
This is valid assuming there are no bins where $f_{o}(x_{i}) = 0$.

To model the true \aclp{PDF} in $x$ as closely as possible, the histogram must
have a fine enough binning that all structures are resolved.
Specifically, as in the determination of the \ac{PID} efficiency in
\cref{chap:prod:effs:pid}, the binning must be fine enough such that difference
between the true \acp{PDF} of the two distribution within any given bin are
small.
This must be balanced with the limited statistical precision available in each
bin: too fine a binning will result in the original distribution being weighted
to match the statistical fluctuations in the target, rather than the physical
features.
Limited sample sizes become particularly problematic for higher-dimensional
weighting, when even 10 equally spaced bins per dimension in a
three-dimensional binning requires $1,000$ bins.
Given physical distributions such as momentum and pseudorapidity, which have
long tails, it can be difficult to invent a binning that avoids sparsely
populated or empty bins whilst still accounting for differences in the
distributions.

To overcome the limitations of histogram-based weighting, this analysis defers
the computation of the per-event weights $w_{i}$ to a decision tree with
gradient boosting~\cite{Rogozhnikov:2016bdp}.
The classifier is trained with the loss function
\begin{equation}
  L = \sum_{l \in \text{Leaves}} \frac{%
    {(w_{l,o} - w_{l,t})}^{2}
  }{%
    w_{l,o} + w_{l,t}
  },
\end{equation}
where $w_{l,t(o)}$ is the sum of target (original) weights in the $l$th leaf
node of the regression tree.
For the initial iteration, and an unweighted set of training events, this is
equal to the number of target (original) events in the leaf node.
This metric has the tree split the samples into regions where the differences
between the sample sizes is maximal, such that the classifier focuses on
weighting the entries in those regions.
Predictions $p_{l}$ within a leaf are computed as
\begin{equation}
  p_{l} = \log{\frac{w_{l,t}}{w_{l,o}}}.
\end{equation}
At the end of the $i$th iteration, each event from the `original' sample is
weighted by $w_{i} = w_{i - 1}e^{p_{l}}$, where $l$ is the leaf node the event
falls in.
The target distribution is assigned unit weights $w_{i} = w_{0}$.
This weighting has the same effect as the weighting in the 
\adaboost~\cite{Freund1997119} algorithm, increasing the importance of events 
in the original distribution that are in regions with a large original/target 
difference.

The actual weighting of events is identical to that in the histogram weighting:
it is the ratio of target to original events in some region.
The use of gradient boosting allows for a more intelligent way of finding the
`bins' in which the weights are computed, and iteratively improves the weights
based on that method.

\section{Evaluation}
\label{chap:cpv:kinematic_weighting:evaluation}

The decision tree with gradient boosting is trained with the \ppipi\ data as
the `original' sample and the \pKK\ data as the `target'.
The weighting procedure is performed separately for each data sub-sample in
year and magnet polarity.
As it is expected that the \PLambdab\ and muon kinematics are correlated with
that of the \PLambdac, the \PLambdac\ transverse momentum \pT\ and
pseudorapidity \Eta\ are included as inputs to the \ac{BDT}.
Also included are the proton \pT\ and \Eta, as these distributions are seen to
disagree after weighting when only the \PLambdac\ kinematics are used.
To avoid biases due to fluctuations in the data, two classifiers $A$ and $B$
are trained, and the \pKK\ and \ppipi\ data are each randomly split into two
sub-samples 1 and 2: classifier $A$ is trained using sub-sample 1, and the
weights for sub-sample 2 are computed using $A$; likewise, classifier $B$ is
trained using sub-sample 2, and the weights for sub-sample 1 are computed using
$B$.
Sub-samples 1 and 2 are then combined for the remainder of the analysis.
The input training data are weighted by signal sWeights, as computed from the
result of the mass fits described in \cref{chap:cpv:prelim_fits}.
Each classifier is boosted in 300 iterations, the maximum number of leaf nodes
allowed in the regression tree is four, and the criteria for splitting includes
the requirement that any new nodes must contain at least 200 events.
The choice of these hyper-parameters will be discussed in
\cref{chap:cpv:kinematic_weighting:validation}, as well as the specific choice
of original and target samples.

\Cref{fig:cpv:kinematic_weighting:post:Lb,fig:cpv:kinematic_weighting:post:Lb_mu,fig:cpv:kinematic_weighting:post:Lc,fig:cpv:kinematic_weighting:post:Lc_p}
show the \PLambdab, muon, \PLambdac, and proton kinematics after weighting with
the product of signal sWeights and the kinematic weights.
\Cref{fig:cpv:kinematic_weighting:post:pKK_h1h2,fig:cpv:kinematic_weighting:post:ppipi_h1h2}
show the $\Phm\Php$ kinematics, \emph{within} the modes, after the same
weighting.
There is a considerable improvement in the agreement between the \pKK\ and
\ppipi\ data in the \PLambdab\ and proton kinematics.
The good agreement in the muon kinematics between the datasets and the
$\Phm\Phm$ kinematics within the datasets does not change.
It is noted that the \PLambdac\ kinematic agreement also improves, although
this does not affect the measurement of \dACP\ as there is no associated
asymmetry.

\section{Validation}
\label{chap:cpv:kinematic_weighting:validation}

The purpose of the kinematic weighting is to make the particle kinematics the
same between the \pKK\ and \ppipi\ modes.
\Cref{fig:cpv:kinematic_weighting:post:Lb,fig:cpv:kinematic_weighting:post:Lb_mu,fig:cpv:kinematic_weighting:post:Lc,fig:cpv:kinematic_weighting:post:Lc_p}
show good agreement `by eye', but a quantitative assessment is needed.
What affects the measurement of \dACP\ is not so much the absolute level of
disagreement between the two histograms, for example, but the disagreement in a
particular region combined with both the size of the respective asymmetries
involved and the density of the data in that region.
The effect of the remaining differences on \dACP\ is discussed in the context
of systematic uncertainties in \cref{chap:cpv:syst}.

To provide some quantitative estimate of the agreement, a \ac{BDT} is employed.
If the two samples, the target weighted with signal sWeights and the original
weighted with the product of signal sWeights and kinematic weights, are truly
identical, then a \ac{BDT} designed to \emph{discriminate} between them should
perform no better than random guessing.
The performance of a \ac{BDT} trained for binary classification can be
evaluated by computing the area under the \ac{ROC} curve, or just \ac{AUC}.
The \ac{ROC} curve compares the \ac{FPR} of the classifier with the \ac{TPR} as
a function of the classifier output probability.
The \ac{TPR} is defined as the fraction of true signal events that are
correctly classified as such, whilst the \ac{FPR} is the fraction of background
events that are classified as signal.
Plotting the \ac{FPR} increasing along the $x$-axis and the \ac{TPR} increasing
along the $y$-axis, the \ac{ROC} curve for an optimal classifier passes through
the top-left corner of the plot, where a large fraction of the signal is
accepted and a correspondingly large background fraction is rejected.
A classifier that cannot distinguish between signal and background any better
than random guessing will have a \ac{ROC} curve of the form $x = y$.
In these two extreme cases, the \ac{AUC} is 1 and 0.5, respectively.

A \ac{BDT}, with gradient boosting, is trained to discriminate between the
weighted \pKK\ and \ppipi\ data, using three-quarters of the available 2012
magnet down data as input.
The \PLambdab, muon, and proton \ptot, \pT, \Eta, and azimuthal angle $\phi$
are used as features.
The \PLambdac\ kinematics are not included as there is no asymmetry associated
to any residual differences there might be.
The maximum number of leaf nodes for any one regression tree is 11, the number
of boosting iterations performed is 400, and the minimum number of samples
allowed in a leaf node is 200.
The performance of the \ac{BDT} is evaluated using the remaining quarter of the
dataset, and the resulting \ac{ROC} curve is shown in
\cref{fig:cpv:kinematic_weighting:post:roc}.
For comparison, it is shown along with a curve obtained when the same
classifier is trained using data with no kinematic weighting, as well as a
curve when a rudimentary, two-dimensional, $10\times10$ histogram weighting in
\PLambdac\ \pT\ and \Eta\ is used.
It is seen that the inclusion of the kinematic weights significantly reduces
the \ac{AUC}, from 0.61 with no kinematic weights to 0.52 for the \ac{BDT}
weighting.
This shows that the \ac{BDT} weighting does increase the similarity between the
kinematic distributions across the modes, and that it can be significantly more
powerful than a two-dimensional histogram weighting.

It is the \ac{AUC} metric evaluated on the training sample that was maximised
when choosing the hyper-parameters to use for the weighting.
A grid search was performed where the following hyper-parameter values were
tested in all combinations: 100, 200, 300, and 400 boosting iterations; a
minimum of 50, 100, 200, 300, 500, and 1000 events required to create a new
node; and a maximum number of allowed leaf nodes per regression tree of 2, 3,
4, 5, 6, 7, and 8.
The hyper-parameters used for the weighting (300 iterations, a minimum of 200
events per node, and a maximum of 4 leaf nodes per tree) were those for which
the \ac{AUC} was smallest.

\section{Weight statistics}
\label{chap:cpv:kinematic_weighting:stats}

Weighting can increase the statistical contribution of individual entries
within a dataset, but it cannot increase the overall statistical power.
The number of entries $N'$ in a weighted dataset can be computed as the sum of
the per-entry weights $w_{i}$, and the variance on this quantity is given as
the sum of the squares of the weights
\begin{equation}
  N' = \sum_{i}^{N} w_{i},\quad (\unc{N'})^{2} = \sum_{i}^{N} w_{i}^{2}
\end{equation}
For certain values of $w_{i}$, it can be that $\unc{N'} < \unc{N}$.
To prevent this, the number of `effective' entries in the weighted dataset is
computed, under the relation that the relative variance on the effective
entries is equal to the relative variance on the weights
\begin{equation}
  \frac{\neff}{\unc{\neff}} =
    \frac{\sum_{i}^{N} w_{i}}{\sqrt{\sum_{i}^{N} w_{i}^{2}}},
\end{equation}
where \neff\ is assumed to be Poisson-distributed, and so $(\unc{\neff})^{2} =
\neff$, giving
\begin{equation}
  \neff = \frac{%
    {\left(\sum_{i}^{N}{w_{i}}\right)}^{2}
  }{%
    \sum_{i}^{N}{w_{i}^{2}},
  }.
  \label{eqn:cpv:kinematic_weighting:neff}
\end{equation}
This relation can be enforced by redefining the weights $w_{i}$ by a factor $W
= \sum_{i} w_{i}/\sum_{i} w_{i}^{2}$.
For unit weights, $\neff = N$ and $\unc{\neff} = \sqrt{N}$.
Otherwise, \neff\ is the size of the hypothetical dataset that has the same
statistical power as the weighted dataset.
For this analysis, it provides a handle on how well the two data samples agree
in the weighting variables before the weighting, as, in general, a higher level
of disagreement requires a larger fraction of candidates to be `thrown away' by
the weighting, reducing the statistical power.

\Cref{tab:cpv:kinematic_weighting:validation:stats} gives several statistics
related to the weighting procedure for the \ppipi\ data.
For all data sub-samples, the effective number of candidates is no less than
\SI{80}{\percent} the number of candidates entering the weighting procedure.
As there are around five times more \ppipi\ candidates than \pKK, the
statistical uncertainty of the \dACP\ measurement will still be dominated by
the \pKK\ sample size, even with an effective \SI{20}{\percent} reduction in
the \ppipi\ sample size.
It is this fact that motivates the choice of the \ppipi\ mode as the source of
the `original' distributions in the weighting and the \pKK\ mode as the
`target'.

\begin{sidewaystable}
  \centering
  \caption{%
    Statistics computed on the weighted \ppipi\ data for all data sub-samples
    used in the analysis.
    The quantities are defined in \cref{chap:cpv:kinematic_weighting:stats}.
  }
  \label{tab:cpv:kinematic_weighting:validation:stats}
  \begin{tabular}{ccccccccc}
  \toprule
  Year & Polarity & $N$    & $\sum{w_{i}}$ & $\sum{w_{i}^{2}}$ & $\sqrt{\sum{w_{i}^{2}}}$ & \neff  & $\unc{\neff}$ & $\frac{\neff}{N}$ \\
  \midrule
  2011   & Up       & 42688  & 39408         & 44801             & 211                      & 34663  & 186           & 0.812             \\
  2011   & Down     & 59044  & 53627         & 58867             & 242                      & 48853  & 221           & 0.827             \\
  2012   & Up       & 147064 & 137461        & 155696            & 394                      & 121362 & 348           & 0.825             \\
  2012   & Down     & 148218 & 138168        & 153675            & 392                      & 124227 & 352           & 0.838             \\
  \bottomrule
\end{tabular}

\end{sidewaystable}

\clearpage

\begin{figure}
  \begin{subfigure}[b]{0.4\textwidth}
    \includegraphics[width=\textwidth]{cpv/kinematic_weighting/preweighting_kinematics/LcToppipi_2012_MagDown_Lb_P}
    \label{fig:cpv:kinematic_weighting:pre:Lb:P}
  \end{subfigure}
  \begin{subfigure}[b]{0.4\textwidth}
    \includegraphics[width=\textwidth]{cpv/kinematic_weighting/preweighting_kinematics/LcToppipi_2012_MagDown_Lb_PT}
    \label{fig:cpv:kinematic_weighting:pre:Lb:PT}
  \end{subfigure}\\
  \begin{subfigure}[b]{0.4\textwidth}
    \includegraphics[width=\textwidth]{cpv/kinematic_weighting/preweighting_kinematics/LcToppipi_2012_MagDown_Lb_ETA}
    \label{fig:cpv:kinematic_weighting:pre:Lb:ETA}
  \end{subfigure}
  \begin{subfigure}[b]{0.4\textwidth}
    \includegraphics[width=\textwidth]{cpv/kinematic_weighting/preweighting_kinematics/LcToppipi_2012_MagDown_Lb_PHI}
    \label{fig:cpv:kinematic_weighting:pre:Lb:PHI}
  \end{subfigure}
  \caption{%
    Clockwise from the top left: total momentum, transverse momentum, angle
    $\phi$, and pseudorapidity of the \PLambdab, weighted by signal sWeights.
    The 2012 magnet down data are shown.
  }
  \label{fig:cpv:kinematic_weighting:pre:Lb}
\end{figure}

\begin{figure}
  \begin{subfigure}[b]{0.4\textwidth}
    \includegraphics[width=\textwidth]{cpv/kinematic_weighting/preweighting_kinematics/LcToppipi_2012_MagDown_Lb_mu_P}
    \label{fig:cpv:kinematic_weighting:pre:Lb_mu:P}
  \end{subfigure}
  \begin{subfigure}[b]{0.4\textwidth}
    \includegraphics[width=\textwidth]{cpv/kinematic_weighting/preweighting_kinematics/LcToppipi_2012_MagDown_Lb_mu_PT}
    \label{fig:cpv:kinematic_weighting:pre:Lb_mu:PT}
  \end{subfigure}\\
  \begin{subfigure}[b]{0.4\textwidth}
    \includegraphics[width=\textwidth]{cpv/kinematic_weighting/preweighting_kinematics/LcToppipi_2012_MagDown_Lb_mu_ETA}
    \label{fig:cpv:kinematic_weighting:pre:Lb_mu:ETA}
  \end{subfigure}
  \begin{subfigure}[b]{0.4\textwidth}
    \includegraphics[width=\textwidth]{cpv/kinematic_weighting/preweighting_kinematics/LcToppipi_2012_MagDown_Lb_mu_PHI}
    \label{fig:cpv:kinematic_weighting:pre:Lb_mu:PHI}
  \end{subfigure}
  \caption{%
    Clockwise from the top left: total momentum, transverse momentum, angle
    $\phi$, and pseudorapidity of the muon from the \PLambdab, weighted by
    signal sWeights.
    The 2012 magnet down data are shown.
  }
  \label{fig:cpv:kinematic_weighting:pre:Lb_mu}
\end{figure}

\begin{figure}
  \begin{subfigure}[b]{0.4\textwidth}
    \includegraphics[width=\textwidth]{cpv/kinematic_weighting/preweighting_kinematics/LcToppipi_2012_MagDown_Lc_P}
    \label{fig:cpv:kinematic_weighting:pre:Lc:P}
  \end{subfigure}
  \begin{subfigure}[b]{0.4\textwidth}
    \includegraphics[width=\textwidth]{cpv/kinematic_weighting/preweighting_kinematics/LcToppipi_2012_MagDown_Lc_PT}
    \label{fig:cpv:kinematic_weighting:pre:Lc:PT}
  \end{subfigure}\\
  \begin{subfigure}[b]{0.4\textwidth}
    \includegraphics[width=\textwidth]{cpv/kinematic_weighting/preweighting_kinematics/LcToppipi_2012_MagDown_Lc_ETA}
    \label{fig:cpv:kinematic_weighting:pre:Lc:ETA}
  \end{subfigure}
  \begin{subfigure}[b]{0.4\textwidth}
    \includegraphics[width=\textwidth]{cpv/kinematic_weighting/preweighting_kinematics/LcToppipi_2012_MagDown_Lc_PHI}
    \label{fig:cpv:kinematic_weighting:pre:Lc:PHI}
  \end{subfigure}
  \caption{%
    Clockwise from the top left: total momentum, transverse momentum, angle
    $\phi$, and pseudorapidity of the \PLambdac, weighted by signal sWeights.
    The 2012 magnet down data are shown.
  }
  \label{fig:cpv:kinematic_weighting:pre:Lc}
\end{figure}

\begin{figure}
  \begin{subfigure}[b]{0.4\textwidth}
    \includegraphics[width=\textwidth]{cpv/kinematic_weighting/preweighting_kinematics/LcToppipi_2012_MagDown_Lc_p_P}
    \label{fig:cpv:kinematic_weighting:pre:Lc_p:P}
  \end{subfigure}
  \begin{subfigure}[b]{0.4\textwidth}
    \includegraphics[width=\textwidth]{cpv/kinematic_weighting/preweighting_kinematics/LcToppipi_2012_MagDown_Lc_p_PT}
    \label{fig:cpv:kinematic_weighting:pre:Lc_p:PT}
  \end{subfigure}\\
  \begin{subfigure}[b]{0.4\textwidth}
    \includegraphics[width=\textwidth]{cpv/kinematic_weighting/preweighting_kinematics/LcToppipi_2012_MagDown_Lc_p_ETA}
    \label{fig:cpv:kinematic_weighting:pre:Lc_p:ETA}
  \end{subfigure}
  \begin{subfigure}[b]{0.4\textwidth}
    \includegraphics[width=\textwidth]{cpv/kinematic_weighting/preweighting_kinematics/LcToppipi_2012_MagDown_Lc_p_PHI}
    \label{fig:cpv:kinematic_weighting:pre:Lc_p:PHI}
  \end{subfigure}
  \caption{%
    Clockwise from the top left: total momentum, transverse momentum, angle
    $\phi$, and pseudorapidity of the proton from the \PLambdac, weighted by
    signal sWeights.
    The 2012 magnet down data are shown.
  }
  \label{fig:cpv:kinematic_weighting:pre:Lc_p}
\end{figure}

\begin{figure}
  \begin{subfigure}[b]{0.5\textwidth}
    \centering
    \includegraphics[width=0.8\textwidth]{cpv/kinematic_weighting/preweighting_kinematics/LcToppipi_2012_MagDown_h1_h2_P_LcTopKK}
    \label{fig:cpv:kinematic_weighting:pre:pKK_h1h2:P}
  \end{subfigure}
  \begin{subfigure}[b]{0.5\textwidth}
    \centering
    \includegraphics[width=0.8\textwidth]{cpv/kinematic_weighting/preweighting_kinematics/LcToppipi_2012_MagDown_h1_h2_PT_LcTopKK}
    \label{fig:cpv:kinematic_weighting:pre:pKK_h1h2:PT}
  \end{subfigure}\\
  \begin{subfigure}[b]{\textwidth}
    \centering
    \includegraphics[width=0.4\textwidth]{cpv/kinematic_weighting/preweighting_kinematics/LcToppipi_2012_MagDown_h1_h2_ETA_LcTopKK}
    \label{fig:cpv:kinematic_weighting:pre:pKK_h1h2:ETA}
  \end{subfigure}
  \caption{%
    Clockwise from the top left: total momentum, transverse momentum, and
    pseudorapidity of the \PKminus\ and \PKplus\ \PLambdac\ children in the
    \pKK\ data, weighted by signal sWeights.
    The 2012 magnet down data are shown.
  }
  \label{fig:cpv:kinematic_weighting:pre:pKK_h1h2}
\end{figure}

\begin{figure}
  \begin{subfigure}[b]{0.5\textwidth}
    \centering
    \includegraphics[width=0.8\textwidth]{cpv/kinematic_weighting/preweighting_kinematics/LcToppipi_2012_MagDown_h1_h2_P_LcToppipi}
    \label{fig:cpv:kinematic_weighting:pre:ppipi_h1h2:P}
  \end{subfigure}
  \begin{subfigure}[b]{0.5\textwidth}
    \centering
    \includegraphics[width=0.8\textwidth]{cpv/kinematic_weighting/preweighting_kinematics/LcToppipi_2012_MagDown_h1_h2_PT_LcToppipi}
    \label{fig:cpv:kinematic_weighting:pre:ppipi_h1h2:PT}
  \end{subfigure}\\
  \begin{subfigure}[b]{\textwidth}
    \centering
    \includegraphics[width=0.4\textwidth]{cpv/kinematic_weighting/preweighting_kinematics/LcToppipi_2012_MagDown_h1_h2_ETA_LcToppipi}
    \label{fig:cpv:kinematic_weighting:pre:ppipi_h1h2:ETA}
  \end{subfigure}
  \caption{%
    Clockwise from the top left: total momentum, transverse momentum, and
    pseudorapidity of the \Ppiminus\ and \Ppiplus\ \PLambdac\ children in the
    \ppipi\ data, weighted by signal sWeights.
    The 2012 magnet down data are shown.
  }
  \label{fig:cpv:kinematic_weighting:pre:ppipi_h1h2}
\end{figure}

\begin{figure}
  \begin{subfigure}[b]{0.4\textwidth}
    \includegraphics[width=\textwidth]{cpv/kinematic_weighting/postweighting_kinematics/LcToppipi_2012_MagDown_Lb_P-weighted}
    \label{fig:cpv:kinematic_weighting:post:Lb:P}
  \end{subfigure}
  \begin{subfigure}[b]{0.4\textwidth}
    \includegraphics[width=\textwidth]{cpv/kinematic_weighting/postweighting_kinematics/LcToppipi_2012_MagDown_Lb_PT-weighted}
    \label{fig:cpv:kinematic_weighting:post:Lb:PT}
  \end{subfigure}\\
  \begin{subfigure}[b]{0.4\textwidth}
    \includegraphics[width=\textwidth]{cpv/kinematic_weighting/postweighting_kinematics/LcToppipi_2012_MagDown_Lb_ETA-weighted}
    \label{fig:cpv:kinematic_weighting:post:Lb:ETA}
  \end{subfigure}
  \begin{subfigure}[b]{0.4\textwidth}
    \includegraphics[width=\textwidth]{cpv/kinematic_weighting/postweighting_kinematics/LcToppipi_2012_MagDown_Lb_PHI-weighted}
    \label{fig:cpv:kinematic_weighting:post:Lb:PHI}
  \end{subfigure}
  \caption{%
    Clockwise from the top left: total momentum, transverse momentum, angle
    $\phi$, and pseudorapidity of the \PLambdab, weighted by the product of
    signal sWeights and kinematic weights.
    The 2012 magnet down data are shown.
  }
  \label{fig:cpv:kinematic_weighting:post:Lb}
\end{figure}

\begin{figure}
  \begin{subfigure}[b]{0.4\textwidth}
    \includegraphics[width=\textwidth]{cpv/kinematic_weighting/postweighting_kinematics/LcToppipi_2012_MagDown_Lb_mu_P-weighted}
    \label{fig:cpv:kinematic_weighting:post:Lb_mu:P}
  \end{subfigure}
  \begin{subfigure}[b]{0.4\textwidth}
    \includegraphics[width=\textwidth]{cpv/kinematic_weighting/postweighting_kinematics/LcToppipi_2012_MagDown_Lb_mu_PT-weighted}
    \label{fig:cpv:kinematic_weighting:post:Lb_mu:PT}
  \end{subfigure}\\
  \begin{subfigure}[b]{0.4\textwidth}
    \includegraphics[width=\textwidth]{cpv/kinematic_weighting/postweighting_kinematics/LcToppipi_2012_MagDown_Lb_mu_ETA-weighted}
    \label{fig:cpv:kinematic_weighting:post:Lb_mu:ETA}
  \end{subfigure}
  \begin{subfigure}[b]{0.4\textwidth}
    \includegraphics[width=\textwidth]{cpv/kinematic_weighting/postweighting_kinematics/LcToppipi_2012_MagDown_Lb_mu_PHI-weighted}
    \label{fig:cpv:kinematic_weighting:post:Lb_mu:PHI}
  \end{subfigure}
  \caption{%
    Clockwise from the top left: total momentum, transverse momentum, angle
    $\phi$, and pseudorapidity of the muon from the \PLambdab, weighted by the
    product of signal sWeights and kinematic weights.
    The 2012 magnet down data are shown.
  }
  \label{fig:cpv:kinematic_weighting:post:Lb_mu}
\end{figure}

\begin{figure}
  \begin{subfigure}[b]{0.4\textwidth}
    \includegraphics[width=\textwidth]{cpv/kinematic_weighting/postweighting_kinematics/LcToppipi_2012_MagDown_Lc_P-weighted}
    \label{fig:cpv:kinematic_weighting:post:Lc:P}
  \end{subfigure}
  \begin{subfigure}[b]{0.4\textwidth}
    \includegraphics[width=\textwidth]{cpv/kinematic_weighting/postweighting_kinematics/LcToppipi_2012_MagDown_Lc_PT-weighted}
    \label{fig:cpv:kinematic_weighting:post:Lc:PT}
  \end{subfigure}\\
  \begin{subfigure}[b]{0.4\textwidth}
    \includegraphics[width=\textwidth]{cpv/kinematic_weighting/postweighting_kinematics/LcToppipi_2012_MagDown_Lc_ETA-weighted}
    \label{fig:cpv:kinematic_weighting:post:Lc:ETA}
  \end{subfigure}
  \begin{subfigure}[b]{0.4\textwidth}
    \includegraphics[width=\textwidth]{cpv/kinematic_weighting/postweighting_kinematics/LcToppipi_2012_MagDown_Lc_PHI-weighted}
    \label{fig:cpv:kinematic_weighting:post:Lc:PHI}
  \end{subfigure}
  \caption{%
    Clockwise from the top left: total momentum, transverse momentum, angle
    $\phi$, and pseudorapidity of the \PLambdac, weighted by the product of
    signal sWeights and kinematic weights.
    The 2012 magnet down data are shown.
  }
  \label{fig:cpv:kinematic_weighting:post:Lc}
\end{figure}

\begin{figure}
  \begin{subfigure}[b]{0.4\textwidth}
    \includegraphics[width=\textwidth]{cpv/kinematic_weighting/postweighting_kinematics/LcToppipi_2012_MagDown_Lc_p_P-weighted}
    \label{fig:cpv:kinematic_weighting:post:Lc_p:P}
  \end{subfigure}
  \begin{subfigure}[b]{0.4\textwidth}
    \includegraphics[width=\textwidth]{cpv/kinematic_weighting/postweighting_kinematics/LcToppipi_2012_MagDown_Lc_p_PT-weighted}
    \label{fig:cpv:kinematic_weighting:post:Lc_p:PT}
  \end{subfigure}\\
  \begin{subfigure}[b]{0.4\textwidth}
    \includegraphics[width=\textwidth]{cpv/kinematic_weighting/postweighting_kinematics/LcToppipi_2012_MagDown_Lc_p_ETA-weighted}
    \label{fig:cpv:kinematic_weighting:post:Lc_p:ETA}
  \end{subfigure}
  \begin{subfigure}[b]{0.4\textwidth}
    \includegraphics[width=\textwidth]{cpv/kinematic_weighting/postweighting_kinematics/LcToppipi_2012_MagDown_Lc_p_PHI-weighted}
    \label{fig:cpv:kinematic_weighting:post:Lc_p:PHI}
  \end{subfigure}
  \caption{%
    Clockwise from the top left: total momentum, transverse momentum, angle
    $\phi$, and pseudorapidity of the proton from the \PLambdac, weighted by
    the product of signal sWeights and kinematic weights.
    The 2012 magnet down data are shown.
  }
  \label{fig:cpv:kinematic_weighting:post:Lc_p}
\end{figure}

\begin{figure}
  \begin{subfigure}[b]{0.5\textwidth}
    \centering
    \includegraphics[width=0.8\textwidth]{cpv/kinematic_weighting/postweighting_kinematics/LcToppipi_2012_MagDown_h1_h2_P_LcTopKK-weighted}
    \label{fig:cpv:kinematic_weighting:post:pKK_h1h2:P}
  \end{subfigure}
  \begin{subfigure}[b]{0.5\textwidth}
    \centering
    \includegraphics[width=0.8\textwidth]{cpv/kinematic_weighting/postweighting_kinematics/LcToppipi_2012_MagDown_h1_h2_PT_LcTopKK-weighted}
    \label{fig:cpv:kinematic_weighting:post:pKK_h1h2:PT}
  \end{subfigure}\\
  \begin{subfigure}[b]{\textwidth}
    \centering
    \includegraphics[width=0.4\textwidth]{cpv/kinematic_weighting/postweighting_kinematics/LcToppipi_2012_MagDown_h1_h2_ETA_LcTopKK-weighted}
    \label{fig:cpv:kinematic_weighting:post:pKK_h1h2:ETA}
  \end{subfigure}
  \caption{%
    Clockwise from the top left: total momentum, transverse momentum, and
    pseudorapidity of the \PKminus\ and \PKplus\ \PLambdac\ children in the
    \pKK\ data, weighted by the product of signal sWeights and kinematic
    weights.
    The 2012 magnet down data are shown.
  }
  \label{fig:cpv:kinematic_weighting:post:pKK_h1h2}
\end{figure}

\begin{figure}
  \begin{subfigure}[b]{0.5\textwidth}
    \centering
    \includegraphics[width=0.8\textwidth]{cpv/kinematic_weighting/postweighting_kinematics/LcToppipi_2012_MagDown_h1_h2_P_LcToppipi-weighted}
    \label{fig:cpv:kinematic_weighting:post:ppipi_h1h2:P}
  \end{subfigure}
  \begin{subfigure}[b]{0.5\textwidth}
    \centering
    \includegraphics[width=0.8\textwidth]{cpv/kinematic_weighting/postweighting_kinematics/LcToppipi_2012_MagDown_h1_h2_PT_LcToppipi-weighted}
    \label{fig:cpv:kinematic_weighting:post:ppipi_h1h2:PT}
  \end{subfigure}\\
  \begin{subfigure}[b]{\textwidth}
    \centering
    \includegraphics[width=0.4\textwidth]{cpv/kinematic_weighting/postweighting_kinematics/LcToppipi_2012_MagDown_h1_h2_ETA_LcToppipi-weighted}
    \label{fig:cpv:kinematic_weighting:post:ppipi_h1h2:ETA}
  \end{subfigure}
  \caption{%
    Clockwise from the top left: total momentum, transverse momentum, and
    pseudorapidity of the \Ppiminus\ and \Ppiplus\ \PLambdac\ children in the
    \ppipi\ data, weighted by the product of signal sWeights and kinematic
    weights.
    The 2012 magnet down data are shown.
  }
  \label{fig:cpv:kinematic_weighting:post:ppipi_h1h2}
\end{figure}

\begin{figure}
  \includegraphics[width=\textwidth]{cpv/kinematic_weighting/postweighting_kinematics/LcToppipi_2012_MagDown_roc_curves}
  \caption{%
    ROC curves for different kinematic weighting techniques.
    The ``no weights'' data has no \emph{kinematic} weights applied, only
    signal sWeights.
    The ``2D binned'' and ``\ac{BDT}'' data uses the product of the respective
    kinematic weights and the signal sWeights for the \ppipi\ data, and uses
    signal sWeights for the \pKK\ data.
    The area under each \ac{ROC} curve, the \acs{AUC}, is given in the legend.
  }
  \label{fig:cpv:kinematic_weighting:post:roc}
\end{figure}
