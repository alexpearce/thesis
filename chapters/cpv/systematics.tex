\chapter{Systematic uncertainties}
\label{chap:cpv:syst}

The measurement of \dACP\ in \cref{chap:cpv:results} carries a statistical 
uncertainty due to finite \pKK\ and \ppipi\ sample sizes.
Three sources of systematic uncertainty are considered in this analysis: the 
arbitrary choice of model used to extract the \PLambdac\ and \APLambdac\ signal 
yields; the failure of the kinematic weighting procedure to fully equalise the 
\PLambdab, muon, and proton kinematics between the modes; and the breaking of 
the assumption that the \PLambdac\ candidates originate from the \LbToLcmuX\ 
decay.
The effect of these on the measurement of \dACP\ will be discussed in this 
\namecref{chap:cpv:syst} and, where appropriate, systematic uncertainties will 
be computed and assigned.

\section{Fit model}
\label{chap:cpv:syst:fit}

The models used for the signal and combinatorial background components in the 
mass fits, presented in \cref{chap:cpv:prelim_fits,chap:cpv:araw}, do describe 
the data well.
However, the choice of functional forms is somewhat arbitrary, and other forms 
may also describe the data well but give different yields, which may change the 
value of \dACP\@.
To assess the extent to which different models can affect the measurement, the 
yields are extracted using the method of \emph{sideband subtraction}.
In this \namecref{chap:cpv:syst:fit}, this method shall first be defined and 
described, and then employed on the data.

\subsection{Sideband subtraction}
\label{chap:cpv:syst:fit:sb}

The sideband subtraction is used to measure the signal yield within some 
\emph{signal window} in some distribution, say of $x$.
The signal window is defined as being centred at $x = X$, and has a width $S$.
The upper and lower bounds of the signal window are then at $x = X \pm S/2$.
The fraction $f_{\text{S}}$ of the data in this region is given by
\begin{equation}
  f_{\text{S}} = \int_{X - \frac{S}{2}}^{X + \frac{S}{2}} h(x) \dif{x},
  \label{eqn:cpv:syst:fit:sig_win_frac}
\end{equation}
where $h(x)$ is the normalised, true \ac{PDF} that generates the observed 
distribution.
It is the sum of the normalised signal and background \acp{PDF} $f(x)$ and 
$g(x)$, having respective yields \nsig\ and \nbkg
\begin{equation}
  h(x) = \frac{1}{\nsig + \nbkg}(\nsig f(x) + \nbkg g(x)).
\end{equation}
Substituting this into \cref{eqn:cpv:syst:fit:sig_win_frac}
\begin{equation}
  f_{\text{S}} = \frac{1}{\nsig + \nbkg}
    \int_{X - \frac{S}{2}}^{X + \frac{S}{2}}
    \nsig f(x) + \nbkg g(x)
    \dif{x}.
  \label{eqn:cpv:syst:fit:sig_win_frac_full}
\end{equation}
Two additional regions are further defined called the lower and upper 
sidebands, collectively the sidebands.
These are centred $D$ either side of the centre $X$ of the signal window, and 
each have a width half of that of the signal window.
The fraction $f_{\text{SB}}$ of the total \ac{PDF} in these regions is
\begin{equation}
  f_{\text{SB}} = \int_{X - D - \frac{S}{4}}^{X - D + \frac{S}{4}} h(x) \dif{x}
    + \int_{X + D - \frac{S}{4}}^{X + D + \frac{S}{4}} h(x) \dif{x}.
  \label{eqn:cpv:syst:fit:sb_frac}
\end{equation}
The sideband subtraction method assumes that the contribution of the signal 
model $f(x)$ within the sidebands is negligible, such that it's contribution to 
the integrals in \cref{eqn:cpv:syst:fit:sb_frac} can be ignored, giving
\begin{equation}
  f_{\text{SB}} = \frac{\nbkg}{\nsig + \nbkg}\left(
    \int_{X - D - \frac{S}{4}}^{X - D + \frac{S}{4}} g(x) \dif{x} +
    \int_{X + D - \frac{S}{4}}^{X + D + \frac{S}{4}} g(x) \dif{x}
  \right).
  \label{eqn:cpv:syst:fit:sb_frac_full}
\end{equation}
It is then desired to subtract \cref{eqn:cpv:syst:fit:sb_frac_full} from
\cref{eqn:cpv:syst:fit:sig_win_frac_full} such that only the signal component 
remains in $f_{\text{S}}$.
This is equivalent to requiring
\begin{equation}
    \int_{X - \frac{S}{2}}^{X + \frac{S}{2}} g(x) \dif{x}
    =
    \int_{X - D - \frac{S}{4}}^{X - D + \frac{S}{4}} g(x) \dif{x} +
    \int_{X + D - \frac{S}{4}}^{X + D + \frac{S}{4}} g(x) \dif{x},
  \label{eqn:cpv:syst:fit:bkg_frac_equiv}
\end{equation}
and is satisfied when $g(x)$ is a linear function
\begin{equation*}
  g(x) = 1 + a_{0}x,
\end{equation*}
in which case the integrals either side of 
\cref{eqn:cpv:syst:fit:bkg_frac_equiv} are equal to
\begin{equation}
    \int_{X - \frac{S}{2}}^{X + \frac{S}{2}} g(x) \dif{x} =
      S(1 + a_{0}X),
\end{equation}
where, as before, $X$ is the centre of the signal window and $S$ is its width.
With a linear form for $g(x)$, it is then the case that
\begin{equation}
  f_{\text{S}} - f_{\text{SB}} = \frac{\nsig}{\nsig + \nbkg}
    \int_{X - \frac{S}{2}}^{X + \frac{S}{2}} f(x) \dif{x}.
  \label{eqn:cpv:syst:fit:frac_diff}
\end{equation}

The quantities $f_{\text{S}}$ and $f_{\text{SB}}$ are the fractions of the 
total yield, in the extent of $h(x)$, in the signal window and sidebands, 
respectively.
In the data, the sum of the signal and background yields is taken to be the 
observed candidate count $N = \nsig + \nbkg$, and so the yield fractions can be 
converted into yields by multiplying them by $N$
\begin{equation}
  N_{\text{S}} = f_{\text{S}}N,\quad
  N_{\text{SB}} = f_{\text{SB}}N.
\end{equation}
Multiplying \cref{eqn:cpv:syst:fit:frac_diff} by $N$ gives
\begin{equation}
  f_{\text{S}}N - f_{\text{SB}}N =
    N_{\text{S}} - N_{\text{SB}} =
    \nsig
    \int_{X - \frac{S}{2}}^{X + \frac{S}{2}} f(x) \dif{x}
  \label{eqn:cpv:syst:fit:yield_diff}
\end{equation}
That is, the difference between the \emph{observed number} of candidates in the 
signal window and the sidebands gives the number of signal candidate in the 
signal window.
The assumptions entering this result are that the contribution of the signal 
model in the sideband regions is small enough to be neglected, that the 
background \ac{PDF} is linear, and, implicitly, that there are no contributions 
to the data in the signal and sideband regions other than signal and 
combinatorial background.

\subsection{Region definitions and method validity}
\label{chap:cpv:syst:fit:defs}

The sideband subtraction is applied to the data as a alternative method to 
measure the \pKK\ and \ppipi\ signal yields in the charge-separated samples.

The signal region is defined as a window \SI{20}{\MeVcc} wide for the \pKK\ 
data and \SI{30}{\MeVcc} wide for \ppipi, centred on the nominal value of the 
\PLambdac\ mass of \SI{2286.46}{\MeVcc}~\cite{PDG2014}.
The wider \ppipi\ window accounts for the wider signal shape, which is itself 
due to the larger energy release, or $Q$ value, of the \ppipi\ decay.
The two sideband regions, each \SI{10}{\MeVcc} wide for \pKK\ and 
\SI{15}{\MeVcc} wide for \ppipi, are chosen such that the centres are 
$\pm\sfrac{3}{2}$ signal window widths away from the signal window centre, $D = 
3S/2$.
The signal and sideband regions are shown in \cref{fig:cpv:syst:mass_windows}, 
with the 2012 magnet down data overlaid for reference.

The assumption of the sideband subtraction technique that the background 
\ac{PDF} is linear is justified by the good description of the background by a 
linear function in the nominal fits.
The lack of any visible physics backgrounds after the full selection described 
in \cref{chap:cpv:selection} suggests that the sidebands are purely 
combinatorial, and that no peaking structures are present in the signal region 
other than the that of the signal.
Finally, the assumption that the signal contribution to the data in the 
sidebands is negligible is checked using \ac{MC}, where \SI{0.7}{\percent} of 
the \ppipi\ truth-matched signal data is in the sidebands, and for \pKK the 
fraction of \SI{0.3}{\percent}.
These fractions are much smaller than the relative statistical uncertainties on 
the signal yields shown in 
\cref{tab:cpv:prelim_fits:yields:pKK,tab:cpv:prelim_fits:yields:ppipi}, and so 
the level of contamination is considered to be negligible.

\begin{figure}
  \begin{subfigure}[b]{0.5\textwidth}
    \includegraphics[width=\textwidth]{cpv/systematics/LcTopKK_2012_MagDown_Lb_DTF_Lc_M.pdf}
    \caption{\pKK}
    \label{fig:cpv:syst:mass_windows:pKK}
  \end{subfigure}
  \begin{subfigure}[b]{0.5\textwidth}
    \includegraphics[width=\textwidth]{cpv/systematics/LcToppipi_2012_MagDown_Lb_DTF_Lc_M.pdf}
    \caption{\ppipi}
    \label{fig:cpv:syst:mass_windows:ppipi}
  \end{subfigure}
  \caption{%
    Definitions of the signal and sideband regions in the \PLambdac\ mass 
    spectrum for \pKK~(\subref*{fig:cpv:syst:mass_windows:pKK}) and 
    \ppipi~(\subref*{fig:cpv:syst:mass_windows:ppipi}).
    The 2012 magnet down data is overlaid for reference.
  }
  \label{fig:cpv:syst:mass_windows}
\end{figure}

\subsection{Results}
\label{chap:cpv:syst:fit:results}

For each data sub-sample, the \pKK\ and \ppipi\ samples are split by the charge 
of the \PLambdac\ and the sideband subtraction technique is used to measure the 
yields.
% TODO This should be the sum of the weights, rather than the candidate count
The yield asymmetry is then
\begin{equation}
  \ARaw(f) = \frac{%
    (N_{\text{S}}(\PLambdac) - N_{\text{SB}}(\PLambdac)) -
    (N_{\text{S}}(\APLambdac) - N_{\text{SB}}(\APLambdac))
  }{%
    (N_{\text{S}}(\PLambdac) - N_{\text{SB}}(\PLambdac)) +
    (N_{\text{S}}(\APLambdac) - N_{\text{SB}}(\APLambdac))
  },
  \label{eqn:cpv:syst:fit:araw_sb}
\end{equation}
where $N_{\text{S(SB)}}(\PLambdac)$ is the number of candidates in the signal 
(sideband) region in the \PLambdac\ sample, and $N_{\text{S(SB)}}(\APLambdac)$ 
is the same but in the \APLambdac\ sample.
The counts are assumed to be Poisson-distributed, and so the uncertainty 
$\unc{N}$ on each count is taken to be $\unc{N} = \sqrt{N}$.

The difference \dACP\ between \ARaw\ for \pKK\ and \ppipi\ is then computed 
using the values obtain with \cref{eqn:cpv:syst:fit:araw_sb}, and the 
difference is computed between those values and the nominal measurements given 
in \cref{tab:results:asymmetries}.
The uncertainty on \ARaw\ is found by propagating the uncertainties on the 
candidate counts using an \ac{MC} error propagation, and the values are assumed 
to be fully correlated with those from the nominal procedure.
The differences for the measurements in all data sub-samples, and their 
averages across years and magnet polarities, are presented in 
\cref{tab:syst:sbs_differences}.
Almost all of the deviations from the nominal values are significant, and so 
they taken as a systematic uncertainties on the measurements of \dACP\@.
On the nominal result quoted in \cref{eqn:cpv:results:dacp}, the systematic 
uncertainty is \SI{0.26}{\percent}.

\begin{sidewaystable}
  \centering
  \caption{%
    Differences between asymmetries measured with the nominal fit method and 
    the sideband subtraction, measured for each data sub-sample and combination 
    of sub-samples.
    The uncertainties quoted assume the values found by the two methods are 
    fully correlated.
    The computation of the combinations, ``2011 + 2012'' and ``Average'', is 
    defined in \cref{chap:cpv:results:combination}.
  }
  \label{tab:syst:sbs_differences}
  \begin{tabular}{ccccc}
  \toprule
  Year & Polarity & \ARaw(\pKK) (\si{\percent}) & \ARaw(\ppipi) (\si{\percent}) & \dACP (\si{\percent}) \\
  \midrule
2011 & Up & $0.73 \pm 0.06$ & $-0.95 \pm 0.21$ & $1.68 \pm 0.03$ \\
2011 & Down & $-0.71 \pm 0.09$ & $-0.14 \pm 0.16$ & $-0.57 \pm 0.02$ \\
2011 & Average & $0.01 \pm 0.05$ & $-0.55 \pm 0.13$ & $0.56 \pm 0.01$ \\
\midrule
2012 & Up & $-0.00 \pm 0.06$ & $-0.29 \pm 0.14$ & $0.28 \pm 0.01$ \\
2012 & Down & $0.43 \pm 0.05$ & $0.48 \pm 0.14$ & $-0.05 \pm 0.02$ \\
2012 & Average & $0.21 \pm 0.04$ & $0.10 \pm 0.10$ & $0.12 \pm 0.01$ \\
\midrule
$2011 + 2012$ & Up & $0.18 \pm 0.05$ & $-0.48 \pm 0.12$ & $0.66 \pm 0.01$ \\
$2011 + 2012$ & Down & $0.10 \pm 0.04$ & $0.32 \pm 0.11$ & $-0.22 \pm 0.01$ \\
$2011 + 2012$ & Average & $0.16 \pm 0.03$ & $-0.10 \pm 0.08$ & $0.26 \pm 0.01$ \\
  \bottomrule
\end{tabular}
\end{sidewaystable}

\section{Residual background asymmetries}
\label{chap:cpv:syst:asym}

The kinematic weighting discussed in \cref{chap:cpv:kinematic_weighting} does 
completely remove all differences between the \PLambdab, muon, and proton 
kinematics between the \pKK\ and \ppipi\ datasets, nor does it remove all 
\PKminus/\PKplus and \Ppiminus/\Ppiplus differences within the decay modes.
Therefore, contributions from the respective background asymmetries will still 
contribute to \dACP\@.
A study is performed to quantity this contamination.
Firstly, the available measurements of the detection and production asymmetries 
are enumerated, and then their application to this analysis is described.

The asymmetries considered are those that arise from the differences in the 
matter and antimatter cross-sections with the detector, called detection 
asymmetries.
The trigger asymmetry for muons from \PB decays is known to be 
small~\cite{Aaij:2016yze}, and is neglected here.

The pion detection asymmetry is measured using \sqrtseq{7} 
data~\cite{Aaij:2012cy} by a method of partial reconstruction of 
\PDstarp-tagged \decay{\PDzero}{\PKminus\Ppiplus\Ppiminus\Ppiplus} decays.
In this, the delta mass variable, defined similarly to that in 
\cref{eqn:prod:fitting:delta_mass}, provides a good signal-to-background 
discrimination even if the pion is not required to have been reconstructed.
The ratio of the signal yields before and after the explicit reconstruction 
gives the pion detection efficiency, and this is performed separately for each 
pion charge in bins of pion kinematics.
The asymmetries are shown in \cref{fig:cpv:syst:asym:pion} as a function of 
pion momentum.
The central values of the relative asymmetries in bins of pion momentum are 
small, generally less than \SI{2}{\percent}, and are very close to zero when 
averaged across the magnet polarities.
The integrated ratios are measured to the $0.9914 \pm 0.0040$ and $1.0045 \pm 
0.0034$ for magnet up and magnet down configurations, respectively.
Here, these values are considered to be compatible with zero, and so 
contributions from a pion detection asymmetry are neglected.

The kaon detection asymmetry is measured using the full \runone\ dataset with 
\DpToKpipi\ and \decay{\PDplus}{\APKzero\Ppiplus} decays~\cite{Aaij:2014gsa}.
The raw asymmetries between the two decays are subtracted, as is the 
contribution from the $\APKzero$ interaction asymmetry, to measure \ADKpi.
The kaon material cross-section asymmetry is known to be much larger than that 
for pions~\cite{PDG2014}, as shown in \cref{fig:cpv:syst:asym:xsecs}, and so 
\ADKpi\ is assumed to be dominated by the kaon detection asymmetry.
The asymmetry as a function of kaon momentum, averaged across magnet 
polarities, is shown in \cref{fig:cpv:syst:asym:kaon} and tabulated in 
\cref{tab:cpv:syst:asym:kaon}, where it is seen to be significant larger in 
magnitude than the pion asymmetry in \cref{fig:cpv:syst:asym:pion} and 
significantly different from zero.

The muon detection asymmetry is measured using the tag-and-probe technique described in \cref{chap:prod:effs:tracking} with \JpsiTomumu~\cite{Stahl:2010261}.
The track reconstruction efficiency is computed separately for \Pmuon and \APmuon and an asymmetry is formed, shown in \cref{fig:cpv:syst:asym:muon} and tabulated in \cref{tab:cpv:syst:asym:muon}.
The asymmetries are small, of the order of \SI{0.1}{\percent}.

Measurements of the production asymmetry from unequal \PLambdab\ and 
\APLambdab\ cross-sections have been made by \lhcb~\cite{Aaij:2015fea}, where 
the asymmetry as a function of \pT\ was found to be negligible over both 2011 
and 2012, but as a function of rapidity $\rapidity$ was found to be
\begin{equation}
  \APLb = (0.058 \pm 0.014)(y - \langle{y}\rangle),
\end{equation}
where $\langle{y}\rangle = 3.1$ is the average \PLambdab\ rapidity in the 
sample used.
These values are compatible between the 2011 and 2012 data.
The same analysis also measured the proton detection asymmetry using simulated 
data, assigning a systematic uncertainty to the result based on the relative 
disagreement between the data and the \ac{MC} when evaluating the \emph{kaon} 
detection asymmetry.
The results are shown in \cref{tab:cpv:syst:asym:proton}.
Simulations using both the \pKK\ and the \ppipi\ samples defined in 
\cref{chap:cpv:data:mc} agree with these results, and are shown in 
\cref{fig:cpv:syst:asym:proton_this}.

\subsection{Evaluation}
\label{chap:cpv:syst:asym:eval}

The effect of the residual background asymmetry for a single asymmetry $A$ due to some particle $X$ on the value of \dACP\ is given by a shift $\Delta{A}$, defined as
\begin{equation}
  \Delta{A}^{X} = \frac{1}{\nsig^{\pKK}}\sum_{i} \nsig^{\pKK,i} A_{i} -
    \frac{1}{\nsig^{\ppipi}}\sum_{i} \nsig^{\ppipi,i} A_{i},
\end{equation}
where $\nsig^{f}$ is the total number of signal candidates in the \LcTof\ sample, $i$ is the bin index of the asymmetry parameter, such as \PLambdab\ rapidity, and $\nsig^{f,i}$ is the number of signal candidates in the \LcTof\ sample that fall in the $i$th asymmetry bin.
The number of signal candidates is computed as the sum of signal sWeights, weighted by kinematic weights and phase space efficiency corrections as in \cref{eqn:cpv:araw:total_weights}.
One such shift is defined for the muon and proton detection asymmetries and the \PLambdab\ production asymmetry, $\Delta\APLb$, $\Delta\ADp$, and $\Delta\ADmu$.
The shift due to the residual kaon detection asymmetry is similarly defined
\begin{equation}
  \Delta{\ADK} = \frac{1}{\nsig^{\pKK}}\left(%
    \sum_{i} \nsig^{\pKK,\PKminus} A_{i} -
    \sum_{i} \nsig^{\pKK,\PKplus} A_{i}
  \right),
\end{equation}
where $\nsig^{\pKK,\PKpm}$ is the number of signal \PKplus/\PKminus particles in the $i$th asymmetry bin.

The total shift of \dACP, $\Delta{A}$ is given by the sum of the individual shifts
\begin{equation}
  \Delta{A} = \Delta\APLb + \Delta\ADp + \Delta\ADmu + \Delta\ADK.
\end{equation}
These shifts for each year and data-taking condition are given in \cref{tab:cpv:syst:asym:results}, along with the shifts for the polarity-integrated datasets.
All of the values found are below one per mille, with the proton detection asymmetry providing the largest contribution.
Conservatively, a \SI{0.1}{\percent} systematic uncertainty is assigned to the measurement of \dACP\ based on the residual background uncertainty.

\begin{figure}
  \centering
  \includegraphics[width=0.75\textwidth]{cpv/systematics/pion_detection_asymmetry_p}
  \caption{%
    Ratio of pion detection asymmetries as a function of pion 
    momentum~\cite{Aaij:2012cy}.
    The uncertainties on the data points are statistical.
  }
  \label{fig:cpv:syst:asym:pion}
\end{figure}

\begin{figure}
  \centering
  \includegraphics[width=\textwidth]{cpv/systematics/kaon_detection_asymmetry_p}
  \caption{%
    Kaon detection asymmetry \ADKpi\ as a function of kaon momentum averaged 
    across magnet polarities~\cite{Aaij:2014gsa}.
    The average of the measurements is shown as a grey band.
    The error bars on the data points include statistical and systematic 
    uncertainties.
  }
  \label{fig:cpv:syst:asym:kaon}
\end{figure}

\begin{table}
  \centering
  \caption{%
    Kaon detection asymmetries \ADKpi\ in bins of kaon 
    momentum~\cite{Aaij:2014gsa}.
  }
  \label{tab:cpv:syst:asym:kaon}
  \begin{tabular}{rclc}
  \toprule
  \multicolumn{3}{c}{Momentum bin [\si{GeVc}]} & Asymmetry  [\si{\percent}] \\
  \midrule
  $0    $ & $< \ptot <$ & $10$   & $-1.37 \pm 0.11$ \\
  $10   $ & $< \ptot <$ & $17.5$ & $-1.20 \pm 0.10$   \\
  $17.5 $ & $< \ptot <$ & $22.5$ & $-1.15 \pm 0.11$ \\
  $22.5 $ & $< \ptot <$ & $30$   & $-1.09 \pm 0.12$ \\
  $30   $ & $< \ptot <$ & $50$   & $-0.88 \pm 0.16$ \\
  $50   $ & $< \ptot <$ & $70$   & $-0.71 \pm 0.29$ \\
  $70   $ & $< \ptot <$ & $100$  & $-0.33 \pm 0.30$  \\
  $100  $ & $< \ptot <$ & $150$  & $0.18 \pm 0.45$  \\
  \bottomrule
\end{tabular}

\end{table}

\begin{figure}
  \centering
  \includegraphics[width=0.75\textwidth]{cpv/systematics/muon_detection_asymmetry_pt}
  \caption{%
    Muon detection asymmetry as a function of muon \pT\ averaged 
    across magnet polarities in \sqrtseq{8} data~\cite{Stahl:2010261}.
    The average of the measurements is shown as a blue band.
    The uncertainties on the data points are statistical.
  }
  \label{fig:cpv:syst:asym:muon}
\end{figure}

\begin{table}
  \centering
  \caption{%
    Muon detection asymmetries in bins of muon \pT measured using \sqrtseq{7} 
    (2011) and \SI{8}{\TeV} (2012)~\cite{Stahl:2010261}.
  }
  \label{tab:cpv:syst:asym:muon}
  \begin{tabular}{rclcc}
  \toprule
  \multicolumn{3}{c}{Momentum bin [\si{\GeVc}]} & 2011 Asymmetry [\si{\percent}] & 2012 Asymmetry [\si{\percent}] \\
  \midrule
  $0$    & $< \ptot <$ & $1830$   & $0.040 \pm 0.159$  & $0.265 \pm 0.117$  \\
  $1830$ & $< \ptot <$ & $2090$   & $0.025 \pm 0.206$  & $-0.085 \pm 0.145$ \\
  $2090$ & $< \ptot <$ & $2460$   & $0.115 \pm 0.184$  & $0.115 \pm 0.131$  \\
  $2460$ & $< \ptot <$ & $2990$   & $-0.085 \pm 0.170$ & $0.035 \pm 0.124$  \\
  $2990$ & $< \ptot <$ & $3860$   & $0.060 \pm 0.177$  & $0.000 \pm 0.127$  \\
  $3860$ & $< \ptot <$ & \num{e5} & $0.275 \pm 0.489$  & $0.190 \pm 0.284$  \\
  \bottomrule
\end{tabular}

\end{table}

\begin{figure}
  \begin{subfigure}{0.5\textwidth}
    \centering
    \includegraphics[width=\textwidth]{cpv/systematics/pion_proton_cross_sections}
    \caption{\decay{\Ppi\Pproton}{X}}
    \label{fig:cpv:syst:asym:xsecs:pion}
  \end{subfigure}
  \begin{subfigure}{0.5\textwidth}
    \centering
    \includegraphics[width=\textwidth]{cpv/systematics/kaon_proton_cross_sections}
    \caption{\decay{\PK\Pproton}{X}}
    \label{fig:cpv:syst:asym:xsecs:kaon}
  \end{subfigure}
  \begin{subfigure}{0.5\textwidth}
    \centering
    \includegraphics[width=\textwidth]{cpv/systematics/proton_proton_cross_sections}
    \caption{\decay{\Pproton\Pproton}{X}}
    \label{fig:cpv:syst:asym:xsecs:proton}
  \end{subfigure}
  \caption{%
    Cross-sections of $h^{\mp}\Pproton \to \text{anything}$~\cite{PDG2014}.
    Each curve shows the hadronic cross-section with proton targets as a 
    function of \sqrts, with the red curve showing the cross-section for the 
    `matter' hadron (\PKplus, \Ppiplus, \Pproton) and the black curve showing 
    that for the `antimatter' hadron (\PKminus, \Ppiminus, \APproton).
  }
  \label{fig:cpv:syst:asym:xsecs}
\end{figure}

\begin{table}
  \centering
  \caption{%
    Proton detection asymmetries measured using simulated data in bins of 
    proton momentum~\cite{Aaij:2015fea}.
  }
  \label{tab:cpv:syst:asym:proton}
  \begin{tabular}{rclc}
  \toprule
  \multicolumn{3}{c}{Momentum bin [\si{\GeVc}]} & Asymmetry  [\si{\percent}] \\
  \midrule
  $0    $ & $< \ptot <$ & $10$   & 4.38 \pm 0.29  \\
  $10   $ & $< \ptot <$ & $17.5$ & 2.46 \pm 0.31  \\
  $17.5 $ & $< \ptot <$ & $22.5$ & 1.90 \pm 0.47  \\
  $22.5 $ & $< \ptot <$ & $30$   & 1.12 \pm 0.47  \\
  $30   $ & $< \ptot <$ & $50$   & 1.20 \pm 0.42  \\
  $50   $ & $< \ptot <$ & $70$   & 0.60 \pm 0.68  \\
  $70   $ & $< \ptot <$ & $100$  & -0.23 \pm 0.91 \\
  $100  $ & $< \ptot <$ & $150$  & 0.4 \pm 1.4    \\
  \bottomrule
\end{tabular}

\end{table}

\begin{figure}
  \centering
  \includegraphics[width=0.75\textwidth]{cpv/systematics/proton_detection_asymmetry_p_this}
  \caption{%
    Asymmetry of proton detection asymmetries as a function of proton momentum 
    in \si{\GeVc}.
    The uncertainties on the data points are statistical.
  }
  \label{fig:cpv:syst:asym:proton_this}
\end{figure}

\begin{sidewaystable}
  \centering
  \caption{%
    Shifts in the measurement of \dACP\ due to background asymmetries not 
    complete removed during the kinematic weighting.
    The total shift $\Delta{A}$ is the sum of the individual ones, and all 
    shifts are given in \num{e-4}.
  }
  \label{tab:cpv:syst:asym:results}
  \begin{tabular}{ccccccc}
  \toprule
  Year & Polarity & \APLb & \ADmu & \ADp & \ADK & $\Delta{A}$\\
  \midrule
  2011 & Up & $0.06 \pm 1.51$ & $0.35 \pm 0.44$ & $-7.08 \pm 1.13$ & $-0.00 \pm 0.00$ & $-6.68 \pm 1.93$ \\
  2011 & Down & $-0.97 \pm 1.62$ & $0.31 \pm 0.59$ & $-8.53 \pm 2.05$ & $-0.00 \pm 0.00$ & $-9.20 \pm 2.67$ \\
  2011 & Up+Down & $-0.68 \pm 1.07$ & $0.33 \pm 0.52$ & $-7.80 \pm 1.44$ & $-0.00 \pm 0.00$ & $-8.14 \pm 1.87$ \\
  \midrule
  2012 & Up & $0.56 \pm 0.82$ & $-0.05 \pm 0.16$ & $-6.84 \pm 1.19$ & $-0.00 \pm 0.00$ & $-6.33 \pm 1.45$ \\
  2012 & Down & $9.41 \pm 1.55$ & $-0.15 \pm 0.22$ & $-11.55 \pm 1.56$ & $-0.00 \pm 0.00$ & $-2.29 \pm 2.21$ \\
  2012 & Up+Down & $5.06 \pm 0.96$ & $-0.10 \pm 0.16$ & $-9.26 \pm 1.34$ & $-0.00 \pm 0.00$ & $-4.30 \pm 1.66$ \\
  \bottomrule
\end{tabular}

\end{sidewaystable}

\section{Signal decays from other sources}
\label{chap:cpv:syst:bkg}

The mass fits count the number of real \PLambdac\ decays in the samples.
However, backgrounds to \LbToLcmuX\ can peak in the \PLambdac\ mass spectrum, 
such as baryonic \PB\ decays or promptly produced \PLambdac\ decays associated 
with random muons.
These backgrounds will have different asymmetries, contaminating \dACP\@.

To assess the effect of prompt \PLambdac, a requirement on the \ipchisq\ of the 
\PLambdac\ candidate is made of $\lnipchisq > 3$.
This should remove almost all prompt decays, given the prompt \lnipchisq\ 
distributions studied in \cref{chap:prod:fitting}.
Distributions of \lnipchisq\ in the 2012 magnet down sub-samples are shown in 
\cref{fig:cpv:syst:bkg:ipchisq}
% TODO fill these in with actual values
This cut reduces the \pKK\ signal yield by SOMEPERCENT the \ppipi\ signal yield 
by SOMEPERCENT\@.
The value of \dACP\ does not change.
The kinematic weighting in \PLambdac\ $(\pT, \Eta)$ would suppress any 
asymmetries arising from prompt \PLambdac\ candidates.

Known alternative sources of \PLambdac\ from beauty decays are considered.
These include \decay{\PBplus}{\APLambdac\Pproton\Ppiplus} decays and
\decay{\PBplus}{\APSigmac\Pproton\Ppi\Ppi} with 
\decay{\PSigmac}{\PLambdac\Ppi}.
In total, the inclusive \decay{\PB}{\PLambdac X} branching fractions are, 
conservatively, less than \SI{10}{\percent} for both \PBplus and \PBzero, and 
the \PBs contribution is negligible~\cite{PDG2014}.
The combined \PB cross-section is around four times that of the 
\PLambdab~\cite{LHCb-PAPER-2013-004,Aaij:2015fea}, and the relative rate of 
\decay{\PB}{\PLambdac X} is around 3 times that of \LbToLcmuX, for a total 
relative rate of 12 times.
However, the muon triggers and $\PLambdac\Pmuon$ vertex requirements strongly 
suppress these alternate sources of \PLambdac\ decays.
Using a simulated sample of one million generated 
\decay{\PBzero}{\APLambdac\Pproton\Ppiminus\Ppiplus} decays, of the order of 
100 \PLambdac\ candidates pass the selection requirements defined in 
\cref{chap:cpv:selection}.
Given this, and that at least the \PBzero production asymmetry is compatible 
with zero below the percent level~\cite{Aaij:2014bba}, these backgrounds are 
neglected.

\section{Summary and combination}
\label{chap:cpv:syst:summary}

Sources of systematic uncertainties on the measurement of \dACP\ have been 
studied.
The effect of the arbitrary choice of fit model is assessed with a sideband 
subtraction method, with the deviation from the nominal result found to be 
\SI{0.26}{\percent}.
Residual asymmetries arising from the failure of the kinematic weighting to 
completely equalise all kinematic distributions are found to shift the value of 
\dACP\ by \SI{0.1}{\percent}.
The contribution of \PLambdac\ candidates from non-\PLambdab\ decays is 
considered to be negligible.

The two systematic uncertainties are added in quadrature in order to assign a 
single systematic to \dACP, giving \SI{0.28}{\percent}.
This is small compared with the statistical uncertainty of \SI{0.89}{\percent} 
given in \cref{tab:cpv:results:asymmetries}.
