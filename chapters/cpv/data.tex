\chapter{Data}
\label{chap:cpv:data}

This analysis exploits the full dataset collected with the \lhcb\ detector 
during \runone\ of the \ac{LHC}.
The data were collected over a period of two years, 2011 and 2012, over which 
the polarity of the \lhcb\ dipole magnet was periodically switched between two 
configurations, referred to here as `magnet up' and `magnet down'.
As the beam conditions changed significantly between 2011 and 2012, from 
\sqrtseq{7} to \SI{8}{\TeV}, the analysis is performed separately on the 2011 
and 2012 datasets.
Furthermore, the analysis is split into magnet up and magnet down sub-samples 
with the years, as the sign of particle detection asymmetries changes with the 
magnet polarity~\cite{Vesterinen:1642153}.
The individual measurements made on these sub-samples is later combined in an 
average.
The total integrated luminosity corresponding to the dataset is \totlumi, the 
breakdown of which for the different sub-samples is given in 
\cref{tab:cpv:data:luminosity}.
This section shall describe the generalities of the data, such as the 
processing workflow and application of calibrations, as well as the simulated 
\ac{MC} data that shall be used.

% TODO this will be described in more detail in the general LHCb intro I guess
The data are first processed by the trigger, and the passing events are saved 
to disk.
They are then reconstructed offline, and filtered through a set of selections 
in a process called the \emph{stripping}.
Like the trigger, the stripping is arranged as a series of parallel 
\emph{lines}, each of which is designed to select a particular physics process.

For this analysis, two stripping lines select the \LcTopKK\ and \LcToppipi\ 
decays of interest.
Sets of tracks are assigned a particular particle hypothesis based on the 
combined response of the \ac{PID} detectors, and then protons and kaons are 
combined to form candidate \pKK\ decays, and protons and pions are combined to 
form candidate \ppipi\ decays.
The resulting \PLambdac\ vertices are combined with muon candidates to form 
candidate \PLambdab\ vertices.
The muon candidate is required to have fired the muon trigger in \lzero, as 
well as the muon trigger in \hltone.
The tracks comprising the \PLambdab\ candidate are required to have been used 
in a positive decision of the \emph{topological semileptonic \PB\ decay} 
trigger line in \hlttwo, which is designed to inclusively select events which 
contain particles topologies characteristic of those in semileptonic \PB 
decays~\cite{Gligorov:2011qxa}.
Candidates passing these trigger requirements are then refined by a tighter 
offline selection to further reduce the fraction of combinatorial background in 
the sample, which is dominant at this stage.
Distributions of the \pKK\ and \ppipi\ mass in the 2012 magnet down dataset 
after the stripping and trigger requirements are given in 
\cref{fig:cpv:data:mass}, as an illustration of the signal purity.
Details of the trigger, stripping, and offline selection will be given in 
\cref{chap:cpv:selection}.

To improve the \PLambdac\ mass resolution in the datasets, and the consistency 
of that between them, two algorithms are run on the data offline.
The first of these is the momentum scale calibration.
This corrects the track momenta by a linear factor $\alpha$, which is found by 
comparing mass measurements made with \JpsiTomumu\ and 
\decay{\PBplus}{\PJpsi\PKplus} decays to the corresponding world 
averages~\cite{Aaij:2014jba}.
As the momentum resolution of the detector varies with the running conditions, 
correcting the track momenta with the calibration factor $\alpha$, which also 
varies, creates a more uniform dataset.
The second algorithm is \decaytreefitter~\cite{Hulsbergen:2005pu}.
This fits the entire \LbToLcmuX, \LcTophh\ decay chain in a single step, rather 
than the back-to-front (or leaf-by-leaf) reconstruction described previously.
The advantage of such a method is that all the information known about the 
decay can influence the fit parameters, whereas the `traditional' 
reconstruction can only propagate information in one direction.
This global treatment of information in the fit is particularly useful when 
applying \emph{constraints}, such as requiring the mass of a vertex is exactly 
some value, or that the head of the decay originated from the \ac{PV}.
In this analysis, \decaytreefitter\ is applied to the \PLambdab\ decay with the 
constraint that the \phh\ vertex has the nominal \PLambdac\ mass of 
\SI{2286}{\MeVcc}~\cite{PDG2014}.
The principle benefit of this is that it improves the resolution of \PLambdac\ 
child pair masses, such as $m(\phm)$ and $m(\hmhp)$, and restricts the 
\PLambdac\ phase space to physical values.
Unless stated otherwise, kinematic quantities used in this analysis are those 
computed by the \decaytreefitter\ algorithm, with the notable exception of the 
\phh\ invariant mass entering the mass fit.

\section{Monte Carlo}
\label{sec:data:mc}

The only information made available for offline analysis is that passing the 
stripping.
Simulated data is used to infer properties of the data before that stage.
There are three different sets of Monte Carlo~(MC) data that are used in this 
analysis.

\subsection{Pseudo-experiments}
\label{chap:cpv:data:mc:toy}

The \rapidsim\ package\footnotemark\ to generate `toy' \ac{MC}.
\rapidsim\ is a phase space \Pbottom-decay generator, which uses a fixed-order
next-to-leading logarithm model to boost the generated \Pbottom-decays in to an 
\lhcb-like laboratory frame, and smears the momenta of visible final state 
particles in order to more closely match the distributions seen in data.
The smearing is based on measurements of the difference between the momentum
resolution between data and \ac{MC}, and is applied in bins of track momentum 
\ptot.

\footnotetext{\url{https://github.com/gcowan/RapidSim}}

One million \LcTopKK\ and one million \LcToppipi\ decays per
centre-of-mass energy are generated with \rapidsim.
The boosting parameters depend on the proton-proton collision centre-of-mass
energy \sqrts, and so samples are generated individually for both \sqrtseq{7} 
and \SI{8}{\TeV}.
The \PLambda\ and \PKshort vetoes, described in 
Section~\ref{chap:cpvselection:offline}, are
applied to the \LcToppipi\ sample, and the \PLambdab\ mass requirement shown in 
Table~\ref{tab:cpv:selection:stripping_composite} is also applied.
The $\PLambdac\Pmuon$ mass, that is the \PLambdab\ mass missing the 
contribution from the neutrino, is required to be in the range $2.5 < 
m(\PLambdac\Pmuon) < \SI{6}{\GeV}$, to match the requirements made in the 
stripping selection shown in Table~\ref{tab:cpv:selection:stripping_composite}.
No acceptance requirements are made.

\subsection{Full detector simulation}
\label{chap:cpv:data:mc:full}

The full \lhcb\ Monte Carlo is used to obtain distributions of the phase space 
variables after the \lhcb\ acceptance, stripping, trigger, and offline 
requirements.
The general procedure of the \lhcb\ \ac{MC} generation is described in 
\cref{chap:prod:data:mc}, and the acceptance requirements are identical to 
those given in \cref{eqn:prod:data:lhcb_acceptance}.

The samples contain around \num{2.5e6} \pKK\ and \num{2.5e6} \ppipi\ generated 
candidates per magnet polarity, simulated using the most representative 
data-taking conditions for 2012.
These conditions correspond to an average number of \pp\ interactions per bunch 
crossing of $\nu = 2.5$, a beam energy of \SI{4}{\TeV}, and a trigger 
configuration representative of the average trigger conditions used in 2012.
The simulated data is reconstructed with the same offline reconstruction as the 
real data, and is then passed through a near-identical version of the 
stripping, with the only difference being the omission of any \ac{PID} 
selection, as the related variables are known to be poorly modelled in the 
\ac{MC}.
The \decaytreefitter\ algorithm is run on the candidates passing the stripping, 
but the momentum scale calibration is not run as the conditions in the \ac{MC} 
are stable.

The decay chain is generated in such a way that the \PLambdab\ is forced to 
decay to $\PLambdac\Pmuon\APnum$.
In one sample, the \PLambdac\ is forced to decay to the \pKK\ final state, with 
a \SI{52}{\percent} probability of proceeding through the resonant 
$\Pproton\Pphi$ channel.
% The hepnames \Pfz macro shows f_0(975), which is unconventional
The \ppipi\ sample is generated similarly, but with \SI{44}{\percent} 
probability of the \LcToppipi\ decay proceeding through the resonant $\Pproton 
f_{0}(980)$ channel.
In each case, the intermediate resonance is simulated incoherently to the 
non-resonant component, and is added to coarsely mimic the dominant resonant 
structure seen in data.

\subsection{Generator-level simulation}
\label{chap:cpv:data:mc:gen}

The Monte Carlo sample described in Section~\ref{chap:cpv:data:mc:full} has 
requirements made on it at the generator level, specifically the \lhcb\ 
acceptance requirements in \cref{eqn:prod:data:lhcb_acceptance}.
This selection do not necessarily have a flat efficiency across the \LcTophh\ 
phase space, and so additional samples are required to study these effects.
A sample of \num{5e5} \ac{MC} events are generated for each \LcTophh\ mode 
before the acceptance cut, for each 2012 magnet polarity.
These samples are not propagated through the detector simulation, and so only 
generator-level, or `truth level', information is available.

\begin{table}
  \centering
  \caption{%
    Integrated luminosity for each data sample used in the analysis.
  }
  \label{tab:cpv:data:luminosity}
  \begin{tabular}{ccS}
  \toprule
  Year & Polarity & {Integrated luminosity (\si{\per\pico\barn})} \\
  \midrule
  2011 & Up       & 422 \pm 7                                   \\
  2011 & Down     & 564 \pm 10                                  \\
  2012 & Up       & 1001 \pm 12                                 \\
  2012 & Down     & 993 \pm 12                                  \\
  \bottomrule
\end{tabular}

\end{table}

\begin{figure}
  \begin{subfigure}[b]{0.5\textwidth}
    \includegraphics[width=\textwidth]{cpv/data/fits-stripping_trigger-selection_unweighted_no-simultaneous/LcTopKK_2012_MagDown_fit.pdf}
    \caption{\pKK}
    \label{fig:cpv:data:mass:pKK}
  \end{subfigure}
  \begin{subfigure}[b]{0.5\textwidth}
    \includegraphics[width=\textwidth]{cpv/data/fits-stripping_trigger-selection_unweighted_no-simultaneous/LcToppipi_2012_MagDown_fit.pdf}
    \caption{\ppipi}
    \label{fig:cpv:data:mass:ppipi}
  \end{subfigure}
  \caption{%
    Fits to the \PLambdac\ mass spectrum in the 2012 magnet down dataset for 
    \pKK\ (\subref*{fig:cpv:data:mass:pKK}) and \ppipi\ 
    (\subref*{fig:cpv:data:mass:ppipi}).
    Only the stripping and trigger selection is applied.
    The fit that is overlaid is described in \cref{}.
  }
  \label{fig:cpv:data:mass}
\end{figure}
