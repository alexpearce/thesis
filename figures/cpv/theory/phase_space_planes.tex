% Rotation of the axes about x-axis, then about z-axis
% We align the x-axis with the negative proton momentum
\tdplotsetmaincoords{0}{0}

% Define proton coordinates
\pgfmathsetmacro{\rp}{0.7}
\pgfmathsetmacro{\angthetap}{90}
\pgfmathsetmacro{\angphip}{180}

% Polarisation axis 'coordinates'
\pgfmathsetmacro{\rpol}{0.7}
\pgfmathsetmacro{\thetapol}{90}
\pgfmathsetmacro{\phipol}{135}

% Define h+h- coordinates
\pgfmathsetmacro{\rKpi}{.4}
\pgfmathsetmacro{\thetaKpi}{80}
\pgfmathsetmacro{\phiKpi}{340}

% Define h+ coordinates
\pgfmathsetmacro{\rhp}{.4}
\pgfmathsetmacro{\thetahp}{90}
\pgfmathsetmacro{\phihp}{-20}

% Define h+ coordinates
\pgfmathsetmacro{\rhm}{.6}
\pgfmathsetmacro{\thetahm}{90}
\pgfmathsetmacro{\phihm}{30}

% Start a TikZ picture, using the tdplot_main_coords system
% tdplot provides the coordinate transformation
\begin{tikzpicture}[scale=4,tdplot_main_coords]

% Define the origin
\coordinate (O) at (0,0,0);

% Create a point P with coordinates r, theta, and phi
% I think this is similar to \coordinate, but also transform the vector
% in to the rotated coordinate system
\tdplotsetcoord{P}{\rp}{\angthetap}{\angphip}
\tdplotsetcoord{Hp}{\rhp}{\thetahp}{\phihp}
\tdplotsetcoord{Hm}{\rhm}{\thetahm}{\phihm}
\tdplotsetcoord{Pol}{\rpol}{\thetapol}{\phipol}
\tdplotsetcoord{Kpi}{\rKpi}{\thetaKpi}{\phiKpi}

% h+h- plane parameters
\pgfmathsetmacro{\hhHeight}{1}
\pgfmathsetmacro{\hhWidth}{0.7}
\pgfmathsetmacro{\hhSkew}{30}
\pgfmathsetmacro{\hhSkewWidth}{tan(\hhSkew)*\hhHeight}
\coordinate (hhOrigin) at ($(-\hhSkewWidth/2,-0.5)$);

% Draw planes before anything else, as fill would cover them
% Proton-z plane
\filldraw[semithick,black,fill=white] ($(P)-(0.1,0.6)$) rectangle (0,0.8);
% h+h- plane
\filldraw[semithick,black,fill=white] (hhOrigin)
  -- ($(hhOrigin)+(\hhSkewWidth, \hhHeight)$)
  -- ($(hhOrigin)+(\hhSkewWidth+\hhWidth,\hhHeight)$)
  -- ($(hhOrigin)+(\hhWidth, 0)$)
  % Close the path
  -- cycle;
% Angle between planes
% \draw[dotted] (hhOrigin) -- ($(hhOrigin)+(0, \hhHeight)$);
\tdplotdrawarc[dotted,very thick]{(O)}{0.5}{90}{
  90-\hhSkew
}{anchor=south west}{\Large$2\pi - \phihh$}

% Draw main coordinate axes
% The anchor key relates to the positioning of the label
% South is towards the bottom of the diagram, east is to the left
% \draw[thick,->] (O) -- (1,0,0) node[anchor=north west]{
%   $x = \hat{p}_{\PLambdac}$
% };
% \draw[thick,->] (O) -- (0,1,0) node[anchor=north west]{
%   $y$
% };

% Proton
\draw[-stealth',color=red] (O) -- (P) node[anchor=south]{\Large\Pproton};
% Polarisation axis
\draw[thick,->] (O) -- (Pol) node[anchor=south]{
  \Large$z = \polzlcp$
};

% h+
\draw[-stealth',color=red] (O) -- (Hp) node[anchor=north]{\Large\Php};
% h-
\draw[-stealth',color=red] (O) -- (Hm) node[anchor=south]{\Large\Phm};

% Draw projection of vector on to xy plane, and a connecting line
% \draw[dashed, color=red] (O) -- (Pxy);
% \draw[dashed, color=red] (P) -- (Pxy);

% Draw angle phi of xy projection
% {} arguments are center point, radius, angle-from, angle-to, options, label
% [] arguments are coordinate frame, draw options
\tdplotdrawarc{(O)}{0.25}{\angphip}{\phipol}{anchor=east}{\Large\thetap}

%set the rotated coordinate system so the x'-y' plane lies within the
%"theta plane" of the main coordinate system
%syntax: \tdplotsetthetaplanecoords{\phi}
% Rotate the "theta plane" (xz) by phi around z
% \tdplotsetthetaplanecoords{\angphip}

% Draw arc in rotated system
% \tdplotdrawarc[tdplot_rotated_coords]{(0,0,0)}{0.35}{0}{\angthetap}{anchor=south 
% west}{$\theta_{p}$}

\end{tikzpicture}
